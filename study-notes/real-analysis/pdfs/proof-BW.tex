\documentclass[12pt, letterpaper]{article}
\usepackage[utf8]{inputenc}
\usepackage[framemethod=TikZ]{mdframed}
\usepackage[hidelinks]{hyperref}
\usepackage{mathtools, amssymb, amsmath, cleveref, fancyhdr, geometry, tcolorbox, graphicx, float, subfigure, arydshln, url, setspace, framed, pifont, physics, ntheorem, cancel}
%%% for coding %%%
\usepackage{listings}
\usepackage[ruled, vlined, linesnumbered]{algorithm2e}
\SetKwComment{Comment}{/* }{ */}
\newcommand\mycommfont[1]{\small\ttfamily\textcolor{mygreen}{#1}}
\SetCommentSty{mycommfont}

\geometry{letterpaper, left=2cm, right=2cm, bottom=2cm, top=2cm}

\pagestyle{fancy}
\fancyhead{}
\fancyhead[L]{\leftmark}
\fancyhead[R]{\rightmark}
\fancyfoot{}
\fancyfoot[C]{\thepage}
%\renewcommand{\headrulewidth}{0pt}
\renewcommand{\footrulewidth}{0pt}

\hypersetup{
	colorlinks = true,
	bookmarks = true,
	bookmarksnumbered = true,
	pdfborder = 001,
	linkcolor = blue
}

\definecolor{emoryblue}{RGB}{1, 33, 105} 
\definecolor{lightblue}{RGB}{0, 125, 186}
\definecolor{mediumblue}{RGB}{ 0, 51, 160}
\definecolor{darkblue}{RGB}{12, 35, 64}
\definecolor{red}{RGB}{185, 58, 38}
\definecolor{green}{RGB}{72, 127, 132}
\definecolor{gray1}{RGB}{217, 217, 214}
\definecolor{gray5}{RGB}{177, 179, 179}
\definecolor{gray3}{RGB}{208, 208, 206}

\definecolor{grey}{rgb}{0.49,0.38,0.29}
\definecolor{mygreen}{rgb}{0,0.6,0}

%%% for coding %%%
\lstset{basicstyle = \ttfamily\small,commentstyle = \color{mygreen}\textit, deletekeywords = {...}, escapeinside = {\%*}{*)}, frame = single, framesep = 0.5em, keywordstyle = \bfseries\color{blue}, morekeywords = {*}, emph = {self}, emphstyle=\bfseries\color{red}, numbers = left, numbersep = 1.5em, numberstyle = \ttfamily\small\color{grey},  rulecolor = \color{black}, showstringspaces = false, stringstyle = \ttfamily\color{purple}, tabsize = 4, columns = flexible}

\newcounter{index}[section]
\renewcommand{\theindex}{\thesection.\arabic{index}}

\newenvironment*{df}[1]{%
  \stepcounter{index}%
  \noindent\textbf{Definition \theindex\ (#1).}%
}{\\}

\newenvironment*{eg}[1]{%
  \stepcounter{index}%
  \begin{framed}\noindent\textbf{Example \theindex\ #1}\\ 
}{%
  \end{framed}
}

\newenvironment*{thm}[1]{%
  \stepcounter{index}%
  \begin{tcolorbox}[colback=gray!5, colframe=gray!40!black,
    title=Theorem \theindex\ (#1)]%
}{%
  \end{tcolorbox}
}

\newenvironment*{cor}[1]{\stepcounter{index}\noindent\textbf{Corollary \theindex\ (#1):}}{\\}
\newenvironment*{lem}[1]{\stepcounter{index}\noindent\textbf{Lemma \theindex\ (#1):}}{\\}
\newenvironment*{ax}[1]{\stepcounter{index}\noindent\textbf{Axiom \theindex\ (#1):}}{\\}
\newenvironment*{prop}[1]{\stepcounter{index}\noindent\textbf{Proposition \theindex\ (#1):}}{\\}
\newenvironment*{conj}[1]{\stepcounter{index}\noindent\textbf{Conjecture \theindex\ (#1):}}{\\}
\newenvironment*{nota}{\stepcounter{index}\noindent\textbf{Notation \theindex.}}{\\}
\newenvironment*{clm}{\stepcounter{index}\noindent\textbf{Claim \theindex}}{\\}

\newcounter{nprf}[section]
\setcounter{nprf}{0}
\newenvironment*{prf}{\noindent\textbf{\textit{Proof \stepcounter{nprf}\thenprf.}}}{\hfill$\blacksquare$\\}
\newenvironment*{dis}{\indent\textbf{\textit{Disproof \stepcounter{nprf}\thenprf.}}}{\hfill$\blacksquare$\\}
\newenvironment*{sol}{\indent\textbf{\textit{Solution \stepcounter{nprf}\thenprf.}}\\}{\hfill{$\square$}\\}

\newenvironment*{prf*}{\noindent\textit{Proof.}\ }{$\qquad\square$\\}
\newenvironment*{dis*}{\indent\textit{Disproof.}\ }{$\qquad\square$\\}
\newenvironment*{sol*}{\indent\textit{Solution.}\ }{$\qquad\square$\\}

\newtheorem{hint}{Hint}[section]
\newtheorem{rmk}{Remark}[section]
\newtheorem{ext}{Extension}[section]

\newtheorem*{df*}{Definition}
\newtheorem*{thm*}{Theorem}
\newtheorem*{clm*}{Claim}
\newtheorem*{cor*}{Corollary}
\newtheorem*{lem*}{Lemma}
\newtheorem*{ax*}{Axiom}
\newtheorem*{prop*}{Proposition}
\newtheorem*{conj*}{Conjecture}
\newtheorem*{nota*}{Notation}

\linespread{1.25}

\newcommand{\inprod}[2]{\left\langle #1, #2 \right\rangle}

\def\Z{{\mathbb{Z}}}
\def\R{{\mathbb{R}}}
\def\C{{\mathbb{C}}}
\def\Q{{\mathbb{Q}}}
\def\d{{\mathrm{d}}}
\def\i{{\mathrm{i}}}
\def\ep{{\varepsilon}}
\def\N{\mathbb{N}}
\def\1{\mathds{1}}
\def\bigO{\mathcal{O}}
\def\sp{\operatorname{span}}
\def\epsilon{\varepsilon}
\def\emptyset{\varnothing}
\def\phi{\varphi}
\def\dsst{\displaystyle}
\def\st{\ s.t.\ }
\def\wrt{\ w.r.t.\ }
\def\bar{\overline}
\def\tilde{\widetilde}
\def\E{\mathbb{E}}
\def\B{\vb{B}}
\def\L{\vb{L}}
\def\I{\vb{I}}
\def\Var{\vb{Var}}
\def\V{\vb{Var}}
\def\Cov{\vb{Cov}}
\def\MSE{\vb{MSE}}
\def\P{\vb{P}}
\def\M{\vb{M}}
\def\iid{i.i.d.}
\def\argmax{\arg\max}
\def\argmin{\arg\min}
\def\l{\ell}
\def\hat{\widehat}
\def\independ{\perp\!\!\!\perp}
\def\depend{\leftrightsquigarrow}
\def\residual{\varepsilon}
\def\sd{\mathrm{sd}}
\def\LI{\mathrm{L.I.}}
\def\range{\operatorname{range}}
\def\Null{\operatorname{null}}
\def\nullity{\operatorname{nullity}}
\def\A{A^{-1}}
\def\alg{\operatorname{alg}}
\def\fl{\operatorname{fl}}
\def\algmult{\operatorname{alg. mult.}}
\def\geomult{\operatorname{geo. mult.}}
\def\diag{\operatorname{diag}}
\def\gap{\operatorname{gap}}
\def\pqde{\quad\square}
\def\lub{\operatorname{lub}}
\def\Int{\operatorname{int}}
\def\ac{\operatorname{ac}}
\def\cl{\operatorname{cl}}
\def\bd{\operatorname{bd}}
\def\upint{\mathchoice%
    {\mkern13mu\overline{\vphantom{\intop}\mkern7mu}\mkern-20mu}%
    {\mkern7mu\overline{\vphantom{\intop}\mkern7mu}\mkern-14mu}%
    {\mkern7mu\overline{\vphantom{\intop}\mkern7mu}\mkern-14mu}%
    {\mkern7mu\overline{\vphantom{\intop}\mkern7mu}\mkern-14mu}%
  \int}
\def\lowint{\mkern3mu\underline{\vphantom{\intop}\mkern7mu}\mkern-10mu\int}


\title{Proof: Bolzano-Weierstrass Theorem\\
Real Analysis} 
\date{Apr. 17th 2025}
\begin{document}

\maketitle
\begin{thm}{Bolzano-Weierstrass}
In a metric space $\M$, $\forall A\subset M$, $A$ is compact if and only if $A$ is sequentially compact
\end{thm}


\begin{prf*}
To begin with, we start by marking some necessary lemmas: \\

\section{Lemmas}
\begin{lem}{Let $A \subset M$ be a compact subset of a metric space $M$. Then $A$ is closed} 
\end{lem}

We will show that the complement $A^C = M \setminus A$ is open.

Fix any point $x \in A^C$. For each $n \in \N$, define the open set
$$U_n = \qty{ y \in M : d(y,x) > \frac{1}{n} }$$
Each $U_n$ is open in $M$, and we claim that $\{ U_n \}{n=1}^\infty$ forms an open cover of $A$. To see this, take any point $a \in A$. Since $a \neq x$ (because $x \in A^C$), we have $d(a,x) = \delta > 0$. Then for any $n$ such that $\frac{1}{n} < \delta$, it follows that $d(a,x) > \frac{1}{n}$, i.e., $a \in U_n$. Therefore, every point $a \in A$ lies in some $U_n$, and hence
$$A \subset \bigcup_{n=1}^\infty U_n$$

Since $A$ is compact, there exists a finite subcover, say $A \subset \bigcup_{i=1}^N U_i$. But since $U_i \subset U_N$ for all $i \leq N$ (because $d(y,x) > 1/i$ implies $d(y,x) > 1/N$ when $i \leq N$), we have
$$A \subset U_N$$
Taking complements, we obtain
$$U_N^C \subset A^C$$
but $U_N^C = \qty{ y \in M : d(y,x) \leq \frac{1}{N} }$ is a closed ball centered at $x$, and in particular, it contains $x$. This shows that for every $x \in A^C$, there exists an open neighborhood $U = M \setminus U_N^C$ of $x$ contained in $A^C$. Hence, $A^C$ is open, so $A$ is closed.\\


\begin{lem}{Let $M$ be a compact metric space, and let $B \subset M$ be closed. Then $B$ is compact.}\\

Let $U_i, i\in I$ be an open cover for $B$, that is
$$B\subset \bigcup_{i\in I} U_i$$
Since $B$ is closed, the complement $B^C$ is open. Then consider the union; 
$$\bigcup_{i\in I} U_i\cup B^C$$
is also an open set and moreover an open cover for $M$. Since $M$ is compact, there exists a finite subcover $\bigcup_{i=1}^N U_i\cup B^C\supset M$. Then we take $B^C$ away on both sides since $B\cap B^C = \varnothing$ meaning $B^C$ does not cover anything for $B$. That makes $\bigcup_{i=1}^N U_i$ a finite subcover for $B$. Therefore, $B$ is compact. 
\end{lem}

\section{Sufficiency (compact $\Rightarrow$ sequentally compact)}
\textbf{Step 1: Set up and reduction}\\

Suppose that $A$ is compact, we need to show that for any sequence $\qty{x_k}$, there exists a converging subsequence whose limit is in the set. \textcolor{red}{Suppose to the contrary} that $A$ is not sequentially compact, i.e. $x_k$ has no converging subsequence. Without loss of generality, assume all points are distinct in $\{x_k\}$.\\


\noindent \textbf{Step 2: For all $k = 1, 2, \ldots, \exists r>0$ such that $x_j\notin D(x_k, r) \ \forall j\neq k$}. \textcolor{teal}{Intuition: if points are not accumulating, then they have to be isolated.}\\


Suppose otherwise that $\exists k$ such that $\forall \ep>0, \exists x_j\in D(x_k, \ep)$. We fix $\ep = \frac{1}{m}$ and obtain $\{x_{j_m}\}$ such that $x_{j_m} \in D(x_k, \frac{1}{m})$. Then $x_{j_m}\rightarrow x_k$ which is a contradiction to the assumption that $x_k$ has no converging subsequence. \\

\textbf{Step 3: Finish up}\\

Consider $B = \{x_k|k = 1, 2, \ldots\}\subset A$. By Step 2, there should not be any accumulation point, therefore, $\mathrm{AC}(B) = \varnothing$. Therefore, $B$ contains all its accumulation points trivially which implies that $B$ is closed, and lemma 2 implies that $B$ is compact.

Now consider the collection of open sets $\mathcal{U} = \{ \{x_k\} \}_{k \in \mathbb{N}}$. Since each point $x_k$ is isolated (by construction in Step 2), the singleton set $\{x_k\}$ is open in the subspace topology on B. Thus, $\mathcal{U}$ is an open cover of B in the subspace topology.

However, $\mathcal U$ does not have a finite subcover for $B$. Therefore, this is a contradiction, and $\exists$ a converging subsequence for $x_k$, which implies that $A$ is sequentially compact. 


\section{Necessity (sequentially compact $\Rightarrow$ compact)}
Suppose that $A\subset M$ is sequentially compact. Show that $A$ is compact. Let $\{U_i\}_{i\in I}$ be an open cover for $A$. 

\begin{clm}{ $\exists r>0$ such that $\forall y\in A, B(y, r)\subset U_i$ for some $i$.}
\textcolor{teal}{Intuition: there’s a uniform ball size that fits entirely within at least one set in the open cover at every point.}
\end{clm}

Suppose not. Then for every $n \in \mathbb{N}$, let $r_n = \frac{1}{n}$. By assumption, for each $r_n$, there exists a point $y_n \in A$ such that for all $i \in I$,
\[
B(y_n, r_n) \nsubseteq U_i.
\]

Since A is sequentially compact, the sequence $\{y_n\} \subset A$ has a convergent subsequence $y_{n_k} \to z \in A$. Since $\{U_i\}{i \in I}$ is an open cover of $A$, there exists some $i_0 \in I$ such that $z \in U{i_0}$.

Because $U_{i_0}$ is open, there exists $\varepsilon > 0$ such that $B(z, \varepsilon) \subset U_{i_0}$. By convergence, there exists $K \in \mathbb{N}$ such that for all $k \geq K$, we have
$$d(y_{n_k}, z) < \frac{\varepsilon}{2}, \quad \text{and} \quad r_{n_k} = \frac{1}{n_k} < \frac{\varepsilon}{2}$$
Then for such $k$, we have
$$B(y_{n_k}, r_{n_k}) \subset B(z, \varepsilon) \subset U_{i_0}$$
which contradicts our assumption that \( B(y_n, r_n) \nsubseteq U_i \) for any $i \in I$. Therefore, the claim is proven. \\

\begin{clm}{ $A$ is totally bounded.}
\end{clm}

Suppose not, then $\exists \ep>0$ such that $A$ cannot be covered by finite number of balls radius $\ep$. Then we can choose $y_1\in A, y_2\in A\backslash B(y_1, \ep), \cdots, y_{k+1} \in A \backslash \bigcup_{i=1}^{k}B(y_i, \ep)$.

Observe that $d(y_i, y_j)\geq \ep \ \forall i\neq j$ for this chosen $\ep$. Then $y_n$ is not Cauchy which implies that $y_n$ does not have a converging subsequence. This is a contradiction to the previous assumption that $A$ is sequentially compact. The claim is proven. 


To wrap up the proof, by Claim 3.1, $\exists r$ such that $\forall y\in A, B(y, r)\subset U_i$ for some $i$. By Claim 3.2, $\exists \{y_1, y_2, \cdots, y_m\}\in A$ such that $A\subset \bigcup_{i=1}^{m} B(y_i, r)$. Also by Claim 3.1, $B(y_{i_j}, r) \subset U_{i_j}$ for some $i_j$. Therefore, $\{U_{i_j}\}_{j=1}^{m}$is a finite subcover for $A$. $A$ is compact. 

Now apply total boundedness with $\varepsilon = r$. There exist points $\{x_1, \dots, x_m\} \subset A$ such that
$$A \subset \bigcup_{j=1}^{m} B(x_j, r)$$

By claim 3.1, each ball $B(x_j, r) \subset U_{i_j}$ for some $i_j \in I$. Then
$$A \subset \bigcup_{j=1}^{m} B(x_j, r) \subset \bigcup_{j=1}^{m} U_{i_j}$$

Thus, $\{U_{i_j}\}_{j=1}^{m}$ is a finite subcover of $A$. Therefore, $A$ is compact.


\end{prf*}


\begin{rmk}{B-W Theorem:}\\
The Bolzano–Weierstrass theorem is a foundational result in real analysis and topology to characterize compactness in metric spaces.

This highlights a central philosophy in analysis: Compactness enables global control using only local data.

One intuitive interpretation is that a sequentially compact set “does not let sequences escape”: every sequence has a cluster point that remains within the set. This is particularly important in function spaces later, where we often work with approximating sequences and need to ensure their limits stay within a prescribed domain.
\end{rmk}




\end{document}

