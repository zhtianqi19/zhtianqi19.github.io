\usepackage[utf8]{inputenc}
\usepackage[framemethod=TikZ]{mdframed}
\usepackage[hidelinks]{hyperref}
\usepackage{mathtools, amssymb, amsmath, cleveref, fancyhdr, geometry, tcolorbox, graphicx, float, subfigure, arydshln, url, setspace, framed, pifont, physics, ntheorem, cancel}
%%% for coding %%%
\usepackage{listings}
\usepackage[ruled, vlined, linesnumbered]{algorithm2e}
\SetKwComment{Comment}{/* }{ */}
\newcommand\mycommfont[1]{\small\ttfamily\textcolor{mygreen}{#1}}
\SetCommentSty{mycommfont}

\geometry{letterpaper, left=2cm, right=2cm, bottom=2cm, top=2cm}

\pagestyle{fancy}
\fancyhead{}
\fancyhead[L]{\leftmark}
\fancyhead[R]{\rightmark}
\fancyfoot{}
\fancyfoot[C]{\thepage}
%\renewcommand{\headrulewidth}{0pt}
\renewcommand{\footrulewidth}{0pt}

\hypersetup{
	colorlinks = true,
	bookmarks = true,
	bookmarksnumbered = true,
	pdfborder = 001,
	linkcolor = blue
}

\definecolor{emoryblue}{RGB}{1, 33, 105} 
\definecolor{lightblue}{RGB}{0, 125, 186}
\definecolor{mediumblue}{RGB}{ 0, 51, 160}
\definecolor{darkblue}{RGB}{12, 35, 64}
\definecolor{red}{RGB}{185, 58, 38}
\definecolor{green}{RGB}{72, 127, 132}
\definecolor{gray1}{RGB}{217, 217, 214}
\definecolor{gray5}{RGB}{177, 179, 179}
\definecolor{gray3}{RGB}{208, 208, 206}

\definecolor{grey}{rgb}{0.49,0.38,0.29}
\definecolor{mygreen}{rgb}{0,0.6,0}

%%% for coding %%%
\lstset{basicstyle = \ttfamily\small,commentstyle = \color{mygreen}\textit, deletekeywords = {...}, escapeinside = {\%*}{*)}, frame = single, framesep = 0.5em, keywordstyle = \bfseries\color{blue}, morekeywords = {*}, emph = {self}, emphstyle=\bfseries\color{red}, numbers = left, numbersep = 1.5em, numberstyle = \ttfamily\small\color{grey},  rulecolor = \color{black}, showstringspaces = false, stringstyle = \ttfamily\color{purple}, tabsize = 4, columns = flexible}

\newcounter{index}[section]
\renewcommand{\theindex}{\thesection.\arabic{index}}

\newenvironment*{df}[1]{%
  \stepcounter{index}%
  \noindent\textbf{Definition \theindex\ (#1).}%
}{\\}

\newenvironment*{eg}[1]{%
  \stepcounter{index}%
  \begin{framed}\noindent\textbf{Example \theindex\ #1}\\ 
}{%
  \end{framed}
}

\newenvironment*{thm}[1]{%
  \stepcounter{index}%
  \begin{tcolorbox}[colback=gray!5, colframe=gray!40!black,
    title=Theorem \theindex\ (#1)]%
}{%
  \end{tcolorbox}
}

\newenvironment*{cor}[1]{\stepcounter{index}\noindent\textbf{Corollary \theindex\ (#1):}}{\\}
\newenvironment*{lem}[1]{\stepcounter{index}\noindent\textbf{Lemma \theindex\ (#1):}}{\\}
\newenvironment*{ax}[1]{\stepcounter{index}\noindent\textbf{Axiom \theindex\ (#1):}}{\\}
\newenvironment*{prop}[1]{\stepcounter{index}\noindent\textbf{Proposition \theindex\ (#1):}}{\\}
\newenvironment*{conj}[1]{\stepcounter{index}\noindent\textbf{Conjecture \theindex\ (#1):}}{\\}
\newenvironment*{nota}{\stepcounter{index}\noindent\textbf{Notation \theindex.}}{\\}
\newenvironment*{clm}{\stepcounter{index}\noindent\textbf{Claim \theindex}}{\\}

\newcounter{nprf}[section]
\setcounter{nprf}{0}
\newenvironment*{prf}{\noindent\textbf{\textit{Proof \stepcounter{nprf}\thenprf.}}}{\hfill$\blacksquare$\\}
\newenvironment*{dis}{\indent\textbf{\textit{Disproof \stepcounter{nprf}\thenprf.}}}{\hfill$\blacksquare$\\}
\newenvironment*{sol}{\indent\textbf{\textit{Solution \stepcounter{nprf}\thenprf.}}\\}{\hfill{$\square$}\\}

\newenvironment*{prf*}{\noindent\textit{Proof.}\ }{$\qquad\square$\\}
\newenvironment*{dis*}{\indent\textit{Disproof.}\ }{$\qquad\square$\\}
\newenvironment*{sol*}{\indent\textit{Solution.}\ }{$\qquad\square$\\}

\newtheorem{hint}{Hint}[section]
\newtheorem{rmk}{Remark}[section]
\newtheorem{ext}{Extension}[section]

\newtheorem*{df*}{Definition}
\newtheorem*{thm*}{Theorem}
\newtheorem*{clm*}{Claim}
\newtheorem*{cor*}{Corollary}
\newtheorem*{lem*}{Lemma}
\newtheorem*{ax*}{Axiom}
\newtheorem*{prop*}{Proposition}
\newtheorem*{conj*}{Conjecture}
\newtheorem*{nota*}{Notation}

\linespread{1.25}

\newcommand{\inprod}[2]{\left\langle #1, #2 \right\rangle}

\def\Z{{\mathbb{Z}}}
\def\R{{\mathbb{R}}}
\def\C{{\mathbb{C}}}
\def\Q{{\mathbb{Q}}}
\def\d{{\mathrm{d}}}
\def\i{{\mathrm{i}}}
\def\ep{{\varepsilon}}
\def\N{\mathbb{N}}
\def\1{\mathds{1}}
\def\bigO{\mathcal{O}}
\def\sp{\operatorname{span}}
\def\epsilon{\varepsilon}
\def\emptyset{\varnothing}
\def\phi{\varphi}
\def\dsst{\displaystyle}
\def\st{\ s.t.\ }
\def\wrt{\ w.r.t.\ }
\def\bar{\overline}
\def\tilde{\widetilde}
\def\E{\mathbb{E}}
\def\B{\vb{B}}
\def\L{\vb{L}}
\def\I{\vb{I}}
\def\Var{\vb{Var}}
\def\V{\vb{Var}}
\def\Cov{\vb{Cov}}
\def\MSE{\vb{MSE}}
\def\P{\vb{P}}
\def\M{\vb{M}}
\def\iid{i.i.d.}
\def\argmax{\arg\max}
\def\argmin{\arg\min}
\def\l{\ell}
\def\hat{\widehat}
\def\independ{\perp\!\!\!\perp}
\def\depend{\leftrightsquigarrow}
\def\residual{\varepsilon}
\def\sd{\mathrm{sd}}
\def\LI{\mathrm{L.I.}}
\def\range{\operatorname{range}}
\def\Null{\operatorname{null}}
\def\nullity{\operatorname{nullity}}
\def\A{A^{-1}}
\def\alg{\operatorname{alg}}
\def\fl{\operatorname{fl}}
\def\algmult{\operatorname{alg. mult.}}
\def\geomult{\operatorname{geo. mult.}}
\def\diag{\operatorname{diag}}
\def\gap{\operatorname{gap}}
\def\pqde{\quad\square}
\def\lub{\operatorname{lub}}
\def\Int{\operatorname{int}}
\def\ac{\operatorname{ac}}
\def\cl{\operatorname{cl}}
\def\bd{\operatorname{bd}}
\def\upint{\mathchoice%
    {\mkern13mu\overline{\vphantom{\intop}\mkern7mu}\mkern-20mu}%
    {\mkern7mu\overline{\vphantom{\intop}\mkern7mu}\mkern-14mu}%
    {\mkern7mu\overline{\vphantom{\intop}\mkern7mu}\mkern-14mu}%
    {\mkern7mu\overline{\vphantom{\intop}\mkern7mu}\mkern-14mu}%
  \int}
\def\lowint{\mkern3mu\underline{\vphantom{\intop}\mkern7mu}\mkern-10mu\int}


\title{\textbf{% ECON 620\\
               Probability and Statistical Inference}}
\author{Tianqi Zhang\\
Emory University}
\date{Apr 17th 2025}

\begin{document}
\maketitle
\setcounter{tocdepth}{1} % Only show sections in the table of contents
\setcounter{section}{5}


\section{Measurable maps}
\subsection{Measurable maps}
\begin{df}{Measurable map}\\
Let \( (\Omega_1, \mathscr{F}_1) \), \( (\Omega_2, \mathscr{F}_2) \) be measurable spaces. \\
A map \( f: \Omega_1 \to \Omega_2 \) is measurable with respect to \( (\mathscr{F}_1, \mathscr{F}_2) \) if:
\[
f^{-1}(A) \in \mathscr{F}_1, \ \forall A \in \mathscr{F}_2.
\]
\noindent \textit{iff Pre-image of measurable sets are measurable.}
\end{df}

\begin{eg}{Indicator Function}\\
\noindent Consider \( \chi_A: (\Omega, \mathscr{F}) \to (\mathbb{R}, \mathscr{B}) \), where
\[
\chi_A(\omega) =
\begin{cases}
1, & \text{if } \omega \in A, \\
0, & \text{else}.
\end{cases}
\]
To show \( \chi_A \) is measurable, check all pre-images:
\[
\chi_A^{-1}(\emptyset) = \emptyset, \quad \chi_A^{-1}(\mathbb{R}) = \Omega, \quad \chi_A^{-1}(\{1\}) = A, \quad \chi_A^{-1}(\{0\}) = A^c.
\]
Since all are in \( \mathscr{F} \), \( \chi_A \) is measurable.

\end{eg}

\begin{eg}{}
\noindent Suppose \( f: (\Omega_1,\mathscr{F}_1)  \to (\Omega_2, \mathscr{F}_2) \) and \( g: (\Omega_2, \mathscr{F}_2) \to (\Omega_3, \mathscr{F}_3) \). If \( f \) and \( g \) are measurable, then \( g \circ f \) is measurable.

\noindent For any \( A \in \mathscr{F}_3 \),
\[
(g \circ f)^{-1}(A) = \underbrace{f^{-1}\underbrace{(g^{-1}(A)}_{\in \mathscr{F}_2})}_{\in\mathscr{F}_1}.
\]
Since \( g^{-1}(A) \in \mathscr{F}_2 \) and \( f^{-1} \) preserves measurability, \( (g \circ f)^{-1}(A) \in \mathscr{F}_1 \).
\end{eg}


\begin{prop}
\noindent Let \( (\Omega, \mathscr{F}) \), \( (\mathbb{R}, \mathscr{B}) \), and \( f, g: \Omega \to \mathbb{R} \) be measurable. Then:
\begin{enumerate}
    \item \( f+g, f-g \) are measurable,
    \item \( |f| \) is measurable.
\end{enumerate}	
\end{prop}

\begin{prop}{Measurable and Continuity}
\noindent Let \( f: X \to \mathbb{R} \) be a continuous function. Then \( f \) is measurable with respect to the Borel sigma algebra.
\end{prop}

\subsection{Probability measure to a random variable (Push-forward)}

\begin{df}{Push-forward measure}
Push-forward measure \( P_X \) is the measure induced on \( \mathbb{R}^d \) via the function \( X \), such that:
\[
\forall A \in \mathscr{B}^d, \; P_X(A) = P(X^{-1}(A)).
\]
\end{df}

\noindent For a probability (measure) space \((\Omega, \mathscr{F}, P)\) and a function (random variable) 
\[
X: (\Omega, \mathscr{F}, P) \to (\mathbb{R}^d, \mathscr{B}^d, \cdot),
\]
There are generally two ways to make $X$ measurable map: 
\begin{enumerate}
	\item Fix $\mathscr{B}^d$ and $\mathscr{F}$, ask if $X$ is measurable. 
	\item Alter $\mathscr{F}$ according to $X$ such that $X$ is measurable. 
\end{enumerate}

\subsubsection{Method 1: Validation}

Ask whether \( X \) is measurable with respect to the fixed \(\sigma\)-algebras:
\[
\forall A \in \mathscr{B}^d, \quad X^{-1}(A) \in \mathscr{F}.
\]
\noindent This directly checks the measurability condition. If satisfied, we say then \( X \) as a function is measurable.

\begin{itemize}
    \item \textit{Non-Measurability} shows by the \textit{same set having 2 different sizes (measures)}
\end{itemize}

\textbf{Criterion}
\textbf{Check (1):}
\begin{itemize}
    \item For \( \Omega \xrightarrow{X} \mathbb{R}^d \), \( \forall A \in \mathscr{B}^d \), \( X^{-1}(A) \in \mathscr{F} \), then \( X \) is measurable.
    \item \textcolor{red}{BUT: Too hard to validate \( \forall A \in \mathscr{B}^d \)}.
\end{itemize}

\subsubsection{Method 2: Generation}
Fix \(\mathscr{B}^d\), then construct a \(\sigma\)-algebra \(\mathscr{F}_X\) (generated by \(X\)) that ensures measurability:
\[
X^{-1}(\mathscr{B}^d) \coloneqq \{X^{-1}(A) \mid A \in \mathscr{B}^d\}.
\]
Define:
\[
\mathscr{F}_X \equiv \sigma(X) \quad \text{(the \(\sigma\)-algebra generated by \(X\))}.
\]
\noindent This is the smallest \(\sigma\)-algebra that makes \(X\) measurable:
\begin{itemize}
    \item \textit{Actually, \(\sigma(X)\) is the smallest \(\sigma\)-algebra for \(X\) to be measurable.}
    \item \(\sigma(X)\) contains all pre-images of \(\mathscr{B}^d\) under \(X\).
\end{itemize}

\begin{eg}{}\\
Let \( X(\omega) = \mathbb{I}_A(\omega) \) for some \( A \subseteq \Omega \), where \( \mathbb{I}_A \) is the indicator function of \( A \). Then the \(\sigma\)-algebra generated by \( X \) is:
\[
\sigma(X) = \{\emptyset, A, A^c, \Omega \}.
\]
This \(\sigma\)-algebra usually represents \textit{information sets} in practice.
\end{eg}

\begin{thm}{Measurability Validation}
\textbf{Solution:} It is sufficient to check the generators \( B \) of $\mathscr{B}$ such that \( X^{-1}(B)\subset \mathscr{F}\).\\
\end{thm}

\begin{eg}\\
 Let \( \mathcal{L} = \left\{ \prod_{i=1}^d (-\infty, x_i] \mid x_i \in \mathbb{R} \right\} \) be a generator of \( \mathscr{B}^d \).
 We have \( X^{-1}(\mathcal{L}) \subseteq \mathscr{F} \), then we generate the smallest \( \sigma \)-algebra:
    \[
    \sigma(X^{-1}(\mathcal{L})) \]
We need to show that
   \[\sigma(X) \subseteq \sigma(X^{-1}(\mathcal{L})) \subseteq \mathscr{F}.
    \]
which makes \( X \) measurable.
\end{eg}
\end{document}