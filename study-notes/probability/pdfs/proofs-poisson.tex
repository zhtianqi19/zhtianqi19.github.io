\documentclass[12pt, letterpaper]{article}

\usepackage[utf8]{inputenc}
\usepackage[framemethod=TikZ]{mdframed}
\usepackage[hidelinks]{hyperref}
\usepackage{mathtools, amssymb, amsmath, cleveref, fancyhdr, geometry, tcolorbox, graphicx, float, subfigure, arydshln, url, setspace, framed, pifont, physics, ntheorem, cancel, mathrsfs}


%%% for coding %%%
\usepackage{listings}
\usepackage[ruled, vlined, linesnumbered]{algorithm2e}
\SetKwComment{Comment}{/* }{ */}
\newcommand\mycommfont[1]{\small\ttfamily\textcolor{mygreen}{#1}}
\SetCommentSty{mycommfont}

\geometry{letterpaper, left=2cm, right=2cm, bottom=2cm, top=2cm}

\pagestyle{fancy}
\fancyhead{}
\fancyhead[L]{\leftmark}
\fancyhead[R]{\rightmark}
\fancyfoot{}
\fancyfoot[C]{\thepage}
%\rfoot{\footnotesize  Tianqi Zhang}


%\renewcommand{\headrulewidth}{0pt}
\renewcommand{\footrulewidth}{0pt}

\hypersetup{
	colorlinks = true,
	bookmarks = true,
	bookmarksnumbered = true,
	pdfborder = 001,
	linkcolor = blue
}

\definecolor{emoryblue}{RGB}{1, 33, 105} 
\definecolor{lightblue}{RGB}{0, 125, 186}
\definecolor{mediumblue}{RGB}{ 0, 51, 160}
\definecolor{darkblue}{RGB}{12, 35, 64}
\definecolor{red}{RGB}{185, 58, 38}
\definecolor{green}{RGB}{72, 127, 132}
\definecolor{gray1}{RGB}{217, 217, 214}
\definecolor{gray5}{RGB}{177, 179, 179}
\definecolor{gray3}{RGB}{208, 208, 206}

\definecolor{grey}{rgb}{0.49,0.38,0.29}
\definecolor{mygreen}{rgb}{0,0.6,0}
\definecolor{grey}{rgb}{0.49,0.38,0.29}
\definecolor{mygreen}{rgb}{0,0.6,0}


%%% for coding %%%
\lstset{basicstyle = \ttfamily\small,commentstyle = \color{mygreen}\textit, deletekeywords = {...}, escapeinside = {\%*}{*)}, frame = single, framesep = 0.5em, keywordstyle = \bfseries\color{blue}, morekeywords = {*}, emph = {self}, emphstyle=\bfseries\color{red}, numbers = left, numbersep = 1.5em, numberstyle = \ttfamily\small\color{grey},  rulecolor = \color{black}, showstringspaces = false, stringstyle = \ttfamily\color{purple}, tabsize = 4, columns = flexible}


\newcounter{index}[subsection]
\setcounter{index}{0}
\newenvironment*{df}[1]{\noindent\textbf{Definition \thesubsection.\stepcounter{index}\theindex\ (#1).}}{\\}

%\newenvironment*{eg}[1]{\begin{framed}\\\noindent\textbf{Example \thesubsection.\stepcounter{index}\theindex\ #1}\\ }{\\\end{framed}}

\newenvironment*{eg}[1]{
    \refstepcounter{index} % Increment the example counter
    \begin{framed}
    \noindent\textbf{Example \thesubsection.\theindex\ #1}
}{
    \end{framed}
}
%\newenvironment*{thm}[1]{\begin{tcolorbox}{\textbf{Theorem \thesubsection.\stepcounter{index}\theindex\ {#1}}}\\}{\\\end{tcolorbox}}
%\newenvironment*{cor}[1]{\noindent\textbf{Corollary \thesubsection.\stepcounter{index}\theindex\ #1:}}{\\}
%\newenvironment*{lem}[1]{\noindent\textbf{Lemma \thesubsection.\stepcounter{index}\theindex\ #1:}}{\\}
%\newenvironment*{ax}[1]{\noindent\textbf{Axiom \thesubsection.\stepcounter{index}\theindex\ #1:}}{\\}
%\newenvironment*{prop}[1]{\noindent\textbf{Proposition \thesubsection.\stepcounter{index}\theindex\ #1:}}{\\}
%\newenvironment*{conj}[1]{\noindent\textbf{Conjecture \thesubsection.\stepcounter{index}\theindex\ #1:}}{\\}
%\newenvironment*{nota}{\noindent\textbf{Notation \thesubsection.\stepcounter{index}\theindex.}}{\\}
%\newenvironment*{clm}{\noindent\textbf{Claim \thesubsection.\stepcounter{index}\theindex}}{\\}

% Ensure proper grouping and formatting for compatibility with lists
\newenvironment*{thm}[1]{%
  \begin{tcolorbox}%
  \textbf{Theorem \thesubsection.\stepcounter{index}\theindex\ {#1}}%
  \par\noindent%
}{%
  \end{tcolorbox}%
}

\newenvironment*{cor}[1]{%
  \par\noindent\textbf{Corollary \thesubsection.\stepcounter{index}\theindex\ {#1}:}%
  \par\noindent%
}{%
  \par%
}

\newenvironment*{lem}[1]{%
  \par\noindent\textbf{Lemma \thesubsection.\stepcounter{index}\theindex\ {#1}:}%
  \par\noindent%
}{%
  \par%
}

\newenvironment*{ax}[1]{%
  \par\noindent\textbf{Axiom \thesubsection.\stepcounter{index}\theindex\ {#1}:}%
  \par\noindent%
}{%
  \par%
}

\newenvironment*{prop}[1]{%
  \par\noindent\textbf{Proposition \thesubsection.\stepcounter{index}\theindex\ {#1}:}%
  \par\noindent%
}{%
  \par%
}

\newenvironment*{conj}[1]{%
  \par\noindent\textbf{Conjecture \thesubsection.\stepcounter{index}\theindex\ {#1}:}%
  \par\noindent%
}{%
  \par%
}

\newenvironment*{nota}{%
  \par\noindent\textbf{Notation \thesubsection.\stepcounter{index}\theindex:}%
  \par\noindent%
}{%
  \par%
}

\newenvironment*{clm}{%
  \par\noindent\textbf{Claim \thesubsection.\stepcounter{index}\theindex:}%
  \par\noindent%
}{%
  \par%
}


\newcounter{nprf}[subsection]
\setcounter{nprf}{0}
\newenvironment*{prf}{\noindent\textbf{\textit{Proof \stepcounter{nprf}\thenprf.}}}{\hfill$\blacksquare$\\}
\newenvironment*{dis}{\indent\textbf{\textit{Disproof \stepcounter{nprf}\thenprf.}}}{\hfill$\blacksquare$\\}
\newenvironment*{sol}{\indent\textbf{\textit{Solution \stepcounter{nprf}\thenprf.}}\\}{\hfill{$\square$}\\}

\newenvironment*{prf*}{\noindent\textit{Proof.}\ }{$\qquad\square$\\}
\newenvironment*{dis*}{\indent\textit{Disproof.}\ }{$\qquad\square$\\}
\newenvironment*{sol*}{\indent\textit{Solution.}\ }{$\qquad\square$\\}

\newtheorem{hint}{Hint}[section]
\newtheorem{rmk}{Remark}[section]
\newtheorem{ext}{Extension}[section]

\newtheorem*{df*}{Definition}
\newtheorem*{thm*}{Theorem}
\newtheorem*{clm*}{Claim}
\newtheorem*{cor*}{Corollary}
\newtheorem*{lem*}{Lemma}
\newtheorem*{ax*}{Axiom}
\newtheorem*{prop*}{Proposition}
\newtheorem*{conj*}{Conjecture}
\newtheorem*{nota*}{Notation}

\linespread{1.25}

\newcommand{\inprod}[2]{\left\langle #1, #2 \right\rangle}

\def\Z{{\mathbb{Z}}}
\def\H{{\mathcal{H}}}
\def\M{{\mathcal{M}}}
\def\R{{\mathbb{R}}}
\def\C{{\mathbb{C}}}
\def\Q{{\mathbb{Q}}}
\def\d{{\mathrm{d}}}
\def\i{{\mathrm{i}}}
\def\ep{{\varepsilon}}
\def\N{\mathbb{N}}
\def\1{\mathds{1}}
\def\bigO{\mathcal{O}}
\def\sp{\operatorname{span}}
\def\epsilon{\varepsilon}
\def\emptyset{\varnothing}
\def\phi{\varphi}
\def\dsst{\displaystyle}
\def\st{\ s.t.\ }
\def\wrt{\ w.r.t.\ }
\def\bar{\overline}
\def\tilde{\widetilde}
\def\E{\vb{E}}
\def\B{\vb{B}}
\def\L{\vb{L}}
\def\I{\vb{I}}
\def\Var{\vb{Var}}
\def\V{\vb{Var}}
\def\Cov{\vb{Cov}}
\def\MSE{\vb{MSE}}
\def\P{\vb{P}}
\def\M{\vb{M}}
\def\iid{i.i.d.}
\def\argmax{\arg\max}
\def\argmin{\arg\min}
\def\l{\ell}
\def\hat{\widehat}
\def\independ{\perp\!\!\!\perp}
\def\depend{\leftrightsquigarrow}
\def\residual{\varepsilon}
\def\sd{\mathrm{sd}}
\def\LI{\mathrm{L.I.}}
\def\range{\operatorname{range}}
\def\Null{\operatorname{null}}
\def\nullity{\operatorname{nullity}}
\def\A{A^{-1}}
\def\alg{\operatorname{alg}}
\def\fl{\operatorname{fl}}
\def\algmult{\operatorname{alg. mult.}}
\def\geomult{\operatorname{geo. mult.}}
\def\diag{\operatorname{diag}}
\def\gap{\operatorname{gap}}
\def\pqde{\quad\square}
\def\lub{\operatorname{lub}}
\def\Int{\operatorname{int}}
\def\ac{\operatorname{ac}}
\def\cl{\operatorname{cl}}
\def\bd{\operatorname{bd}}
\DeclareMathOperator*{\plim}{plim}
\def\upint{\mathchoice%
    {\mkern13mu\overline{\vphantom{\intop}\mkern7mu}\mkern-20mu}%
    {\mkern7mu\overline{\vphantom{\intop}\mkern7mu}\mkern-14mu}%
    {\mkern7mu\overline{\vphantom{\intop}\mkern7mu}\mkern-14mu}%
    {\mkern7mu\overline{\vphantom{\intop}\mkern7mu}\mkern-14mu}%
  \int}
\def\lowint{\mkern3mu\underline{\vphantom{\intop}\mkern7mu}\mkern-10mu\int}




\title{\textbf{% ECON 620\\
               Probability and Statistical Inference}}
\author{Tianqi Zhang\\
Emory University}
\date{Apr 17th 2025}

\begin{document}
\maketitle
\setcounter{tocdepth}{1} % Only show sections in the table of contents


\begin{thm}{Binomial converges in distribution to Poisson}
Let $\qty{X_n}_{n\in \N^+}$ be a sequence of random variables defined on a probability space $(\Omega, \mathcal{B}, P)$ and suppose that
$$X_n\sim \mathrm{Binom}\qty(n, \frac{\lambda}{n})$$
Then $X_n\xrightarrow{d} X$ such that $X\sim \mathrm{Poisson}(\lambda)$. 
\end{thm}

\begin{prf*}
To show the convergence in distribution, we need to show that the probability densities converge to a limit density, that is, 
$$\lim_{n\to\infty} P(X_n=k) = P(X=k) = e^{-\lambda}\frac{\lambda^{k}}{k!}$$
Fix $k\in \N^+$, given each $X_n\sim\mathrm{Binom}\qty(n, \frac{\lambda}{n})$, we have
\begin{align*}
	\lim_{n\to\infty} P(X_n = k) &= \lim_{n\to\infty}\binom{n}{k} \qty(\frac{\lambda}{n})^{k}\qty(1-\frac{\lambda}{n})^{n-k}\\
	&= \lim_{n\to\infty}\frac{n!}{k!(n-k)!}\qty(\frac{\lambda}{n})^{k}\qty(1-\frac{\lambda}{n})^{n}\qty(1-\frac{\lambda}{n})^{-k} \quad \text{definition of binomial coefficient}\\
	&= \underbrace{\frac{\lambda^k}{k!}}_{\textcolor{blue}{\text{Poisson-ish}}}\lim_{n\to\infty}\frac{n!}{(n-k)!}\qty(1-\frac{\lambda}{n})^{n}\frac{1}{n^k\qty(1-\frac{\lambda}{n})^k} \quad \text{grouping terms and factoring}\\
	&= \frac{\lambda^k}{k!}\lim_{n\to\infty}\qty{n(n-1)\cdots(n-k+1)}\qty(1-\frac{\lambda}{n})^{n}\frac{1}{(n-\lambda)^k}\\
	&= \frac{\lambda^k}{k!}\lim_{n\to\infty}\underbrace{\frac{\qty{n(n-1)\cdots(n-k+1)}}{(n-\lambda)^k}}_{(1)}\underbrace{\qty(1-\frac{\lambda}{n})^{n}}_{\rightarrow e^{-\lambda} \text{ by def}}
\end{align*}
Notice that term $(1)$ expands to: 
\begin{align*}
	\lim_{n\to\infty} \frac{n(n-1)\cdots(n-k+1)}{(n-\lambda)^k} &= \lim_{n\to\infty} \frac{n^k}{n^k \qty(1 - \frac{\lambda}{n})^k} \prod_{j=0}^{k-1} \qty(1 - \frac{j}{n})\\
	&= \lim_{n\to\infty} \qty(1-\frac{\lambda}{n})^{-k}\cdot \prod_{j=0}^{k-1} \qty(1 - \frac{j}{n})\\
	&= 1
\end{align*}
The first term and each term in the product go to 1. Therefore, $\lim_{n\to\infty}(1) = 1$.\\

\noindent We have shown that $\exists X$ s.t. $\lim_{n\to\infty} X_n = X$ and $X\sim \mathrm{Poisson}(\lambda)$. 
\end{prf*}


A more standard approach in proving weak convergence (convergence in distribution) is to look at the pointwise convergence in the characteristic functions. For \(X_n \sim \mathrm{Binomial}\qty(n, \frac{\lambda}{n})\),
\begin{clm*}
$$\varphi_{X_n}(t) = \mathbb{E}\bigl[e^{\,i\,t\,X_n}\bigr] = \qty(1 - \frac{\lambda}{n} + \frac{\lambda}{n}\,e^{\,i\,t})^{n}$$
\end{clm*}
%
%\noindent To show this: 
%\begin{align*}
%	\varphi_{X_n}(t) &= \mathbb{E}\bigl[e^{\,i\,t\,X_n}\bigr]\\
%	&= \sum_{k=0}^n e^{itk} \cdot P(X_n = k)\\
%	&= \sum_{k=0}^n \binom{n}{k} \qty(e^{it}\frac{\lambda}{n})^k \qty(1-\frac{\lambda}{n})^{n-k}
%\end{align*}
%According to the binomial theorem, for $x, y\in \R, (x+y)^n = \sum_{k=0}^n \binom{n}{k} x^ky^{n-k}$. In this case, let $x \equiv e^{\,i\,t}\frac{\lambda}{n}$ and $y = \qty(1-\frac{\lambda}{n})$, we have the above expression simplified into 
%$$\varphi_{X_n}(t) = \qty( e^{\,i\,t}\frac{\lambda}{n} + 1 - \frac{\lambda}{n})^n$$
\noindent Then the convergence becomes more apparent as
\begin{align*}
	\lim_{n\to\infty} \varphi_{X_n}(t) &= \lim_{n\to\infty} \qty( 1 - \frac{\lambda}{n} + \frac{\lambda}{n}\,e^{\,i\,t})^{n}\\
	&= \lim_{n\to\infty} \left(1 + \frac{\lambda\qty(e^{\,i\,t}-1)}{n}\right)^{n}\\
	&= \exp{\lambda\qty(e^{\,i\,t}-1)}
\end{align*}
The result is precisely the characteristic function of a Poisson-distributed random variable. \\


\end{document}