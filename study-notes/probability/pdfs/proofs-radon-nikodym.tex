\documentclass[12pt, letterpaper]{article}

\usepackage[utf8]{inputenc}
\usepackage[framemethod=TikZ]{mdframed}
\usepackage[hidelinks]{hyperref}
\usepackage{mathtools, amssymb, amsmath, cleveref, fancyhdr, geometry, tcolorbox, graphicx, float, subfigure, arydshln, url, setspace, framed, pifont, physics, ntheorem, cancel, mathrsfs}


%%% for coding %%%
\usepackage{listings}
\usepackage[ruled, vlined, linesnumbered]{algorithm2e}
\SetKwComment{Comment}{/* }{ */}
\newcommand\mycommfont[1]{\small\ttfamily\textcolor{mygreen}{#1}}
\SetCommentSty{mycommfont}

\geometry{letterpaper, left=2cm, right=2cm, bottom=2cm, top=2cm}

\pagestyle{fancy}
\fancyhead{}
\fancyhead[L]{\leftmark}
\fancyhead[R]{\rightmark}
\fancyfoot{}
\fancyfoot[C]{\thepage}
%\rfoot{\footnotesize  Tianqi Zhang}


%\renewcommand{\headrulewidth}{0pt}
\renewcommand{\footrulewidth}{0pt}

\hypersetup{
	colorlinks = true,
	bookmarks = true,
	bookmarksnumbered = true,
	pdfborder = 001,
	linkcolor = blue
}

\definecolor{emoryblue}{RGB}{1, 33, 105} 
\definecolor{lightblue}{RGB}{0, 125, 186}
\definecolor{mediumblue}{RGB}{ 0, 51, 160}
\definecolor{darkblue}{RGB}{12, 35, 64}
\definecolor{red}{RGB}{185, 58, 38}
\definecolor{green}{RGB}{72, 127, 132}
\definecolor{gray1}{RGB}{217, 217, 214}
\definecolor{gray5}{RGB}{177, 179, 179}
\definecolor{gray3}{RGB}{208, 208, 206}

\definecolor{grey}{rgb}{0.49,0.38,0.29}
\definecolor{mygreen}{rgb}{0,0.6,0}
\definecolor{grey}{rgb}{0.49,0.38,0.29}
\definecolor{mygreen}{rgb}{0,0.6,0}


%%% for coding %%%
\lstset{basicstyle = \ttfamily\small,commentstyle = \color{mygreen}\textit, deletekeywords = {...}, escapeinside = {\%*}{*)}, frame = single, framesep = 0.5em, keywordstyle = \bfseries\color{blue}, morekeywords = {*}, emph = {self}, emphstyle=\bfseries\color{red}, numbers = left, numbersep = 1.5em, numberstyle = \ttfamily\small\color{grey},  rulecolor = \color{black}, showstringspaces = false, stringstyle = \ttfamily\color{purple}, tabsize = 4, columns = flexible}


\newcounter{index}[subsection]
\setcounter{index}{0}
\newenvironment*{df}[1]{\noindent\textbf{Definition \thesubsection.\stepcounter{index}\theindex\ (#1).}}{\\}

%\newenvironment*{eg}[1]{\begin{framed}\\\noindent\textbf{Example \thesubsection.\stepcounter{index}\theindex\ #1}\\ }{\\\end{framed}}

\newenvironment*{eg}[1]{
    \refstepcounter{index} % Increment the example counter
    \begin{framed}
    \noindent\textbf{Example \thesubsection.\theindex\ #1}
}{
    \end{framed}
}
%\newenvironment*{thm}[1]{\begin{tcolorbox}{\textbf{Theorem \thesubsection.\stepcounter{index}\theindex\ {#1}}}\\}{\\\end{tcolorbox}}
%\newenvironment*{cor}[1]{\noindent\textbf{Corollary \thesubsection.\stepcounter{index}\theindex\ #1:}}{\\}
%\newenvironment*{lem}[1]{\noindent\textbf{Lemma \thesubsection.\stepcounter{index}\theindex\ #1:}}{\\}
%\newenvironment*{ax}[1]{\noindent\textbf{Axiom \thesubsection.\stepcounter{index}\theindex\ #1:}}{\\}
%\newenvironment*{prop}[1]{\noindent\textbf{Proposition \thesubsection.\stepcounter{index}\theindex\ #1:}}{\\}
%\newenvironment*{conj}[1]{\noindent\textbf{Conjecture \thesubsection.\stepcounter{index}\theindex\ #1:}}{\\}
%\newenvironment*{nota}{\noindent\textbf{Notation \thesubsection.\stepcounter{index}\theindex.}}{\\}
%\newenvironment*{clm}{\noindent\textbf{Claim \thesubsection.\stepcounter{index}\theindex}}{\\}

% Ensure proper grouping and formatting for compatibility with lists
\newenvironment*{thm}[1]{%
  \begin{tcolorbox}%
  \textbf{Theorem \thesubsection.\stepcounter{index}\theindex\ {#1}}%
  \par\noindent%
}{%
  \end{tcolorbox}%
}

\newenvironment*{cor}[1]{%
  \par\noindent\textbf{Corollary \thesubsection.\stepcounter{index}\theindex\ {#1}:}%
  \par\noindent%
}{%
  \par%
}

\newenvironment*{lem}[1]{%
  \par\noindent\textbf{Lemma \thesubsection.\stepcounter{index}\theindex\ {#1}:}%
  \par\noindent%
}{%
  \par%
}

\newenvironment*{ax}[1]{%
  \par\noindent\textbf{Axiom \thesubsection.\stepcounter{index}\theindex\ {#1}:}%
  \par\noindent%
}{%
  \par%
}

\newenvironment*{prop}[1]{%
  \par\noindent\textbf{Proposition \thesubsection.\stepcounter{index}\theindex\ {#1}:}%
  \par\noindent%
}{%
  \par%
}

\newenvironment*{conj}[1]{%
  \par\noindent\textbf{Conjecture \thesubsection.\stepcounter{index}\theindex\ {#1}:}%
  \par\noindent%
}{%
  \par%
}

\newenvironment*{nota}{%
  \par\noindent\textbf{Notation \thesubsection.\stepcounter{index}\theindex:}%
  \par\noindent%
}{%
  \par%
}

\newenvironment*{clm}{%
  \par\noindent\textbf{Claim \thesubsection.\stepcounter{index}\theindex:}%
  \par\noindent%
}{%
  \par%
}


\newcounter{nprf}[subsection]
\setcounter{nprf}{0}
\newenvironment*{prf}{\noindent\textbf{\textit{Proof \stepcounter{nprf}\thenprf.}}}{\hfill$\blacksquare$\\}
\newenvironment*{dis}{\indent\textbf{\textit{Disproof \stepcounter{nprf}\thenprf.}}}{\hfill$\blacksquare$\\}
\newenvironment*{sol}{\indent\textbf{\textit{Solution \stepcounter{nprf}\thenprf.}}\\}{\hfill{$\square$}\\}

\newenvironment*{prf*}{\noindent\textit{Proof.}\ }{$\qquad\square$\\}
\newenvironment*{dis*}{\indent\textit{Disproof.}\ }{$\qquad\square$\\}
\newenvironment*{sol*}{\indent\textit{Solution.}\ }{$\qquad\square$\\}

\newtheorem{hint}{Hint}[section]
\newtheorem{rmk}{Remark}[section]
\newtheorem{ext}{Extension}[section]

\newtheorem*{df*}{Definition}
\newtheorem*{thm*}{Theorem}
\newtheorem*{clm*}{Claim}
\newtheorem*{cor*}{Corollary}
\newtheorem*{lem*}{Lemma}
\newtheorem*{ax*}{Axiom}
\newtheorem*{prop*}{Proposition}
\newtheorem*{conj*}{Conjecture}
\newtheorem*{nota*}{Notation}

\linespread{1.25}

\newcommand{\inprod}[2]{\left\langle #1, #2 \right\rangle}

\def\Z{{\mathbb{Z}}}
\def\H{{\mathcal{H}}}
\def\M{{\mathcal{M}}}
\def\R{{\mathbb{R}}}
\def\C{{\mathbb{C}}}
\def\Q{{\mathbb{Q}}}
\def\d{{\mathrm{d}}}
\def\i{{\mathrm{i}}}
\def\ep{{\varepsilon}}
\def\N{\mathbb{N}}
\def\1{\mathds{1}}
\def\bigO{\mathcal{O}}
\def\sp{\operatorname{span}}
\def\epsilon{\varepsilon}
\def\emptyset{\varnothing}
\def\phi{\varphi}
\def\dsst{\displaystyle}
\def\st{\ s.t.\ }
\def\wrt{\ w.r.t.\ }
\def\bar{\overline}
\def\tilde{\widetilde}
\def\E{\vb{E}}
\def\B{\vb{B}}
\def\L{\vb{L}}
\def\I{\vb{I}}
\def\Var{\vb{Var}}
\def\V{\vb{Var}}
\def\Cov{\vb{Cov}}
\def\MSE{\vb{MSE}}
\def\P{\vb{P}}
\def\M{\vb{M}}
\def\iid{i.i.d.}
\def\argmax{\arg\max}
\def\argmin{\arg\min}
\def\l{\ell}
\def\hat{\widehat}
\def\independ{\perp\!\!\!\perp}
\def\depend{\leftrightsquigarrow}
\def\residual{\varepsilon}
\def\sd{\mathrm{sd}}
\def\LI{\mathrm{L.I.}}
\def\range{\operatorname{range}}
\def\Null{\operatorname{null}}
\def\nullity{\operatorname{nullity}}
\def\A{A^{-1}}
\def\alg{\operatorname{alg}}
\def\fl{\operatorname{fl}}
\def\algmult{\operatorname{alg. mult.}}
\def\geomult{\operatorname{geo. mult.}}
\def\diag{\operatorname{diag}}
\def\gap{\operatorname{gap}}
\def\pqde{\quad\square}
\def\lub{\operatorname{lub}}
\def\Int{\operatorname{int}}
\def\ac{\operatorname{ac}}
\def\cl{\operatorname{cl}}
\def\bd{\operatorname{bd}}
\DeclareMathOperator*{\plim}{plim}
\def\upint{\mathchoice%
    {\mkern13mu\overline{\vphantom{\intop}\mkern7mu}\mkern-20mu}%
    {\mkern7mu\overline{\vphantom{\intop}\mkern7mu}\mkern-14mu}%
    {\mkern7mu\overline{\vphantom{\intop}\mkern7mu}\mkern-14mu}%
    {\mkern7mu\overline{\vphantom{\intop}\mkern7mu}\mkern-14mu}%
  \int}
\def\lowint{\mkern3mu\underline{\vphantom{\intop}\mkern7mu}\mkern-10mu\int}




\title{\textbf{% ECON 620\\
               Probability and Statistical Inference}}
\author{Tianqi Zhang\\
Emory University}
\date{Apr 17th 2025}

\begin{document}
\maketitle
\setcounter{tocdepth}{1} % Only show sections in the table of contents


\begin{thm}{Radon-Nikodym Theorem}
Let $(X, \mathcal{M})$ be a measurable space, and let $\nu$ and $\mu$ be $\sigma$-finite measures on $(X, \mathcal{M})$. Then there exists a unique decomposition
$$\nu = \nu_a + \nu_s$$
such that:
\begin{enumerate}
	\item $\nu_a \ll \mu$ (absolutely continuous part),
	\item $\nu_s \perp \mu$ (singular part).
\end{enumerate}
Moreover, there exists a unique (up to $\mu$-a.e. equivalence) measurable function $f \geq 0$ such that
$d\nu_a = f \, d\mu$.
We denote this density function as the Radon-Nikodym derivative:
$$f = \frac{d\nu}{d\mu}$$
\end{thm}

To prove this, we firstly need Riesz representation theorem for Hilbert Space:\\

\begin{thm}{Riesz Representation Theorem (Hilbert Space)}
Given $\mathcal{H}$ and a continuous linear functional
$$\ell: \mathcal H \to \mathbb C$$
Then $\exists!$ a vector $w\in \mathcal{H}$ such that $\forall v\in \mathcal H$, 
$$\ell(v) = \bra{v}\ket{w}$$
\end{thm}
 
\begin{prf}
We start by a proposition as follows: 
\begin{prop}{}
For a closed subspace $S\subset \H$, we have a decomposition
$$\H = S \oplus S^\perp$$	
Such that $S^\perp \equiv \qty{w\in \H: \bra{v}\ket{w} = 0, \ v\in S}$. Given $u\in \H$, set
$$d(u, S) \equiv \inf \qty{\norm{u-v}: v\in S}$$
Then there exists a sequence $\qty{v_n}$ such that $\norm{u-v_n} \to d$. \\

By the law of parallelogram, for any $v_n, v_m$ in the sequence, 
\begin{align*}
	\norm{v_n-v_m}^2 &= 2\norm{u-v_n}^2 + 2\norm{u-v_m}^2 -  4\norm{u-\frac{v_n+v_m}{2}}^2\\
	\lim_{n, m\to \infty} \norm{v_n-v_m}^2 &\leq 2d^2 + 2d^2 - 4d^2 \leq 0
\end{align*}
Therefore $\qty{v_n}$ is Cauchy. Since $S$ is closed, the limit exists and is in $S$, we denote it by $v$.\\

%Now let $w = u-v$, we have $\bra{v}\ket{w} = 0$. We have proved the proposition. 



Define $w \equiv u - v$. We claim $w \in S^\perp$. 
Indeed, for any $s \in S$, consider 
\[
   \|u - s\|^2 
   \;=\; \|(v + w) - s\|^2
   \;=\; \|(v - s) + w\|^2.
\]

Since $v$ is the chosen minimizer in $S$, the first-order condition for minimality gives 
\(\langle w, s-v\rangle = 0\). 
In particular, taking $s = v$ yields $\langle w, v-v\rangle=0$ which is trivial, 
but more importantly, by choosing $s$ to approach $v$ suitably, 
we obtain $\langle w, s\rangle = 0$ for any $s\in S$. Thus $w$ is orthogonal to every vector in $S$, i.e.\ $w \in S^\perp$.
\end{prop}\

\textbf{Back to Riesz:} We are given a continuous and linear functional $\ell: \H \to \C$. Take any $S\subset \H$ as the kernel of $\ell$, i.e
$$S \equiv \ker(\ell) = \qty{v\in \H: \ell(v)=0}$$ 
Then $\ell$ is linear implies that $S$ is a subspace (by Kernel), and $\ell$ is continuous implies that $S$ is closed (by pre-image of the kernel). Then we have a decomposition by the proposition above: 
$$\H = S \oplus S^\perp$$
Note that it would be trivial if either $S$ or $S^\perp$ is the empty set. Then we can choose $w = 0$ and the theorem is proven. \\

Therefore, suppose that neither is the empty set. We fix any $w\in S^\perp$ with $\norm{w_1} =1$. We take $v\in \H$ and observe $v\ell(w_1) - w_1\ell(v)\in S$ since $S$ is closed. Then by orthogonality, 
$$\bra{v\ell(w_1) - w_1\ell(v)}\ket{w} = 0$$
We extend the expression into: 
$$\bra{v}\ket{\overline{\ell(w_1)}w_1} - \ell(v)\norm{w_1}^2 = 0$$
And finally: 
$$\ell(v) = \bra{v}\ket{\overline{\ell(w_1)}w_1}$$
We define $w\in S^\perp$ to be $w \equiv \overline{\ell(w_1)}w_1$. We have shown the existence of $w\in S^\perp$ in this context. \\

As for uniqueness, assume $w'\in S^\perp$ that also suffices the above assumptions, then for all $v\in \H$. 
\begin{align*}
	\ell(v) - \ell(v) &= \bra{v}\ket{w}-\bra{v}\ket{w'}\\
					0 &= \bra{v}\ket{w-w'}
\end{align*}
We choose $v = w-w'$, then $\norm{w-w'}^2 =0$. By positive definite of the norm, we have $w = w'$. We have shown the uniqueness of such $w$.\\


\textbf{Back to R-N:} Let $(X, \mathcal{M})$ be a measure space with $\sigma$-finite measure $\mu, \nu$. Let $\rho = \mu + \nu$. Since both are sigma finite, $\rho$ is also a properly defined measure on $X$. We define $\ell: \mathcal{L}^2(X, d\rho)\to \C$ by 
$$\ell(\psi) \equiv \int_X \psi \, d\nu$$
Then since $\nu \leq \rho$,
$$\abs{\ell(\psi)} = \bra{1}\ket{\psi}_{\mathcal L^2(X, d\rho)} \leq \underbrace{\norm{1}}_{<\infty}\norm{\psi}_{\mathcal L^2(X, d\rho)} \leq C \norm{\psi}_{\mathcal L^2(X, d\rho)}$$
Then $\ell$ is continuous. \\

By Riesz, $\exists g\in \mathcal L^2(X, d\rho)$ such that $\ell(\psi) = \bra{\psi}\ket{g} = \int_X \psi \overline g\, d\rho$. In particular for any $E\subset X$, we can express its measure 
$$\nu(E) = \ell (\chi_E) = \int_E \overline g\, d\rho$$ 
Then $g$ is a real and non-negative function a.e. on $\rho$. \\

Also, $\nu(E) \leq \rho(E)$, we have
\begin{align*}
	\nu (E) = \int_E d\nu &\leq \int_E d\rho \quad \forall E\in \mathcal M\\
	\int_E g\, d\rho &\leq \int_E d\rho 
\end{align*}
We have $g\leq 1$ a.e. on $\rho$. \\

Now given $\psi\in \mathcal L^2(X, d\rho)$. 
\begin{align}
	\ell(\psi) &= \int_X \psi gd\rho \notag \\
    \int_X \psi \, d\nu &= \int_X \psi g\, d\mu + \int_X \psi g\, d\nu \notag\\
	\int_X \psi (1-g) \, d\nu &= \int_X \psi g\, d\mu \label{star}
\end{align}
As $g: X\to [0, 1]$, we can define the set $A\equiv \qty{g<1}$ and $B\equiv \qty{g=1}$. We have accordingly the two measures $\nu_a(E) \equiv \nu(A\cap E)$ and $\nu_s(E) \equiv \nu(B\cap E)$. By \eqref{star} we have
\begin{align*}
	\mu(B) &= \int_B\, d\mu\\
		   &= \int_X\chi_B g\, d\mu \quad \text{as }g = 1\text{ on }B\\
		   &= \int_X \chi_B (1-g) \, d\nu = 0 
\end{align*}
Therefore, the measure $\mu$ cannot see $B$. We have $\nu_s \perp \mu$ proving the first half of the theorem. \\

Revisit \eqref{star}, let $\psi \equiv \chi_E(1+g+g^2+\dots + g^n)$ for any $E\in \mathcal M$. \eqref{star} equals to the following:  
\begin{align*}
	\int_E\qty(1-g^{n+1})\, d\nu &= \int_E g\qty(1+\dots + g^n)\, d\mu\\
	\int_{E\cap A} 1-g^{n+1}\, d\nu &= \int_E  g\qty(1+\dots + g^n)\, d\mu
\end{align*}
If we take $n\to \infty$, \\

The left hand side will be dominated by $\nu_a(E)$ since $g<1$ on $A$. Then by DCT, we have the left as $\int_E \, d\nu_a$. 

The right hand side will converge to $\int_E \frac{g}{1-g}\, d\mu$ as a geometric series. We have the following expression: 
$$\nu_a(E) = \int_E \frac{g}{1-g} \, d\mu$$
We define $f\equiv \frac{g}{1-g}$. Therefore, we have the Radon-Nikodym derivative. 

\end{prf}


%\begin{rmk}
% 
%\end{rmk}


 
 
 


\end{document}