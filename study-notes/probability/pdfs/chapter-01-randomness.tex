\usepackage[utf8]{inputenc}
\usepackage[framemethod=TikZ]{mdframed}
\usepackage[hidelinks]{hyperref}
\usepackage{mathtools, amssymb, amsmath, cleveref, fancyhdr, geometry, tcolorbox, graphicx, float, subfigure, arydshln, url, setspace, framed, pifont, physics, ntheorem, cancel}
%%% for coding %%%
\usepackage{listings}
\usepackage[ruled, vlined, linesnumbered]{algorithm2e}
\SetKwComment{Comment}{/* }{ */}
\newcommand\mycommfont[1]{\small\ttfamily\textcolor{mygreen}{#1}}
\SetCommentSty{mycommfont}

\geometry{letterpaper, left=2cm, right=2cm, bottom=2cm, top=2cm}

\pagestyle{fancy}
\fancyhead{}
\fancyhead[L]{\leftmark}
\fancyhead[R]{\rightmark}
\fancyfoot{}
\fancyfoot[C]{\thepage}
%\renewcommand{\headrulewidth}{0pt}
\renewcommand{\footrulewidth}{0pt}

\hypersetup{
	colorlinks = true,
	bookmarks = true,
	bookmarksnumbered = true,
	pdfborder = 001,
	linkcolor = blue
}

\definecolor{emoryblue}{RGB}{1, 33, 105} 
\definecolor{lightblue}{RGB}{0, 125, 186}
\definecolor{mediumblue}{RGB}{ 0, 51, 160}
\definecolor{darkblue}{RGB}{12, 35, 64}
\definecolor{red}{RGB}{185, 58, 38}
\definecolor{green}{RGB}{72, 127, 132}
\definecolor{gray1}{RGB}{217, 217, 214}
\definecolor{gray5}{RGB}{177, 179, 179}
\definecolor{gray3}{RGB}{208, 208, 206}

\definecolor{grey}{rgb}{0.49,0.38,0.29}
\definecolor{mygreen}{rgb}{0,0.6,0}

%%% for coding %%%
\lstset{basicstyle = \ttfamily\small,commentstyle = \color{mygreen}\textit, deletekeywords = {...}, escapeinside = {\%*}{*)}, frame = single, framesep = 0.5em, keywordstyle = \bfseries\color{blue}, morekeywords = {*}, emph = {self}, emphstyle=\bfseries\color{red}, numbers = left, numbersep = 1.5em, numberstyle = \ttfamily\small\color{grey},  rulecolor = \color{black}, showstringspaces = false, stringstyle = \ttfamily\color{purple}, tabsize = 4, columns = flexible}

\newcounter{index}[section]
\renewcommand{\theindex}{\thesection.\arabic{index}}

\newenvironment*{df}[1]{%
  \stepcounter{index}%
  \noindent\textbf{Definition \theindex\ (#1).}%
}{\\}

\newenvironment*{eg}[1]{%
  \stepcounter{index}%
  \begin{framed}\noindent\textbf{Example \theindex\ #1}\\ 
}{%
  \end{framed}
}

\newenvironment*{thm}[1]{%
  \stepcounter{index}%
  \begin{tcolorbox}[colback=gray!5, colframe=gray!40!black,
    title=Theorem \theindex\ (#1)]%
}{%
  \end{tcolorbox}
}

\newenvironment*{cor}[1]{\stepcounter{index}\noindent\textbf{Corollary \theindex\ (#1):}}{\\}
\newenvironment*{lem}[1]{\stepcounter{index}\noindent\textbf{Lemma \theindex\ (#1):}}{\\}
\newenvironment*{ax}[1]{\stepcounter{index}\noindent\textbf{Axiom \theindex\ (#1):}}{\\}
\newenvironment*{prop}[1]{\stepcounter{index}\noindent\textbf{Proposition \theindex\ (#1):}}{\\}
\newenvironment*{conj}[1]{\stepcounter{index}\noindent\textbf{Conjecture \theindex\ (#1):}}{\\}
\newenvironment*{nota}{\stepcounter{index}\noindent\textbf{Notation \theindex.}}{\\}
\newenvironment*{clm}{\stepcounter{index}\noindent\textbf{Claim \theindex}}{\\}

\newcounter{nprf}[section]
\setcounter{nprf}{0}
\newenvironment*{prf}{\noindent\textbf{\textit{Proof \stepcounter{nprf}\thenprf.}}}{\hfill$\blacksquare$\\}
\newenvironment*{dis}{\indent\textbf{\textit{Disproof \stepcounter{nprf}\thenprf.}}}{\hfill$\blacksquare$\\}
\newenvironment*{sol}{\indent\textbf{\textit{Solution \stepcounter{nprf}\thenprf.}}\\}{\hfill{$\square$}\\}

\newenvironment*{prf*}{\noindent\textit{Proof.}\ }{$\qquad\square$\\}
\newenvironment*{dis*}{\indent\textit{Disproof.}\ }{$\qquad\square$\\}
\newenvironment*{sol*}{\indent\textit{Solution.}\ }{$\qquad\square$\\}

\newtheorem{hint}{Hint}[section]
\newtheorem{rmk}{Remark}[section]
\newtheorem{ext}{Extension}[section]

\newtheorem*{df*}{Definition}
\newtheorem*{thm*}{Theorem}
\newtheorem*{clm*}{Claim}
\newtheorem*{cor*}{Corollary}
\newtheorem*{lem*}{Lemma}
\newtheorem*{ax*}{Axiom}
\newtheorem*{prop*}{Proposition}
\newtheorem*{conj*}{Conjecture}
\newtheorem*{nota*}{Notation}

\linespread{1.25}

\newcommand{\inprod}[2]{\left\langle #1, #2 \right\rangle}

\def\Z{{\mathbb{Z}}}
\def\R{{\mathbb{R}}}
\def\C{{\mathbb{C}}}
\def\Q{{\mathbb{Q}}}
\def\d{{\mathrm{d}}}
\def\i{{\mathrm{i}}}
\def\ep{{\varepsilon}}
\def\N{\mathbb{N}}
\def\1{\mathds{1}}
\def\bigO{\mathcal{O}}
\def\sp{\operatorname{span}}
\def\epsilon{\varepsilon}
\def\emptyset{\varnothing}
\def\phi{\varphi}
\def\dsst{\displaystyle}
\def\st{\ s.t.\ }
\def\wrt{\ w.r.t.\ }
\def\bar{\overline}
\def\tilde{\widetilde}
\def\E{\mathbb{E}}
\def\B{\vb{B}}
\def\L{\vb{L}}
\def\I{\vb{I}}
\def\Var{\vb{Var}}
\def\V{\vb{Var}}
\def\Cov{\vb{Cov}}
\def\MSE{\vb{MSE}}
\def\P{\vb{P}}
\def\M{\vb{M}}
\def\iid{i.i.d.}
\def\argmax{\arg\max}
\def\argmin{\arg\min}
\def\l{\ell}
\def\hat{\widehat}
\def\independ{\perp\!\!\!\perp}
\def\depend{\leftrightsquigarrow}
\def\residual{\varepsilon}
\def\sd{\mathrm{sd}}
\def\LI{\mathrm{L.I.}}
\def\range{\operatorname{range}}
\def\Null{\operatorname{null}}
\def\nullity{\operatorname{nullity}}
\def\A{A^{-1}}
\def\alg{\operatorname{alg}}
\def\fl{\operatorname{fl}}
\def\algmult{\operatorname{alg. mult.}}
\def\geomult{\operatorname{geo. mult.}}
\def\diag{\operatorname{diag}}
\def\gap{\operatorname{gap}}
\def\pqde{\quad\square}
\def\lub{\operatorname{lub}}
\def\Int{\operatorname{int}}
\def\ac{\operatorname{ac}}
\def\cl{\operatorname{cl}}
\def\bd{\operatorname{bd}}
\def\upint{\mathchoice%
    {\mkern13mu\overline{\vphantom{\intop}\mkern7mu}\mkern-20mu}%
    {\mkern7mu\overline{\vphantom{\intop}\mkern7mu}\mkern-14mu}%
    {\mkern7mu\overline{\vphantom{\intop}\mkern7mu}\mkern-14mu}%
    {\mkern7mu\overline{\vphantom{\intop}\mkern7mu}\mkern-14mu}%
  \int}
\def\lowint{\mkern3mu\underline{\vphantom{\intop}\mkern7mu}\mkern-10mu\int}


\title{\textbf{% ECON 620\\
               Probability and Statistical Inference}}
\author{Tianqi Zhang\\
Emory University}
\date{Apr 17th 2025}


\begin{document}
\maketitle
\setcounter{tocdepth}{1} % Only show sections in the table of contents
%\tableofcontents

\section{Randomness}

\subsection{Randomness: A Model of Empirical Observations}
\begin{df}{Latent Space $\Omega$}\\
	We denote the latent space $\Omega$ to be the set of all possible outcomes. \\
	\textbf{Random:} A model of an empirically observed property of the world.
\end{df}

\begin{df}{Random Variable}
The random variable $X$ maps each $\omega \in \Omega$ to $\mathbb{R}^d$
\end{df}

\begin{eg}{Coin toss}\\
The random variable $X: \qty{\text{Head, Tail}} \rightarrow \R$ is defined to be: 
$$\quad X(\text{Head}) = 1, \quad X(\text{Tail}) = 0$$
\end{eg}

\noindent If we repeat the same experiment under the same conditions, an event \(A \subseteq \Omega\) will occur in some experiments but not in others.

\subsection{Sampling Frequency}
If we conduct \(n\) experiments, event \(A\) occurs exactly \(n_A\) times. Then the sampling frequency of \(A\) is:
\[
f_n(A) = \frac{n_A}{n}.
\]

\begin{eg}{Coin toss}\\
\(\{\text{3 heads, 97 tails}\}\) implies in 100 random experiments with a fair coin: 
\[
    f_{100}(\text{3 heads, 97 tails}) = \frac{3}{100}.
\]
\end{eg}


\subsection{Two Philosophies of Randomness}
\noindent What happens as \(n \to \infty\)?
\subsubsection{Frequentist Inference}
\begin{itemize}
    \item Stability at large scale: volatility of fluctuations of \(f_n(A)\) tends to decrease, where \(n \to \infty\).
\end{itemize}

\begin{thm}{Frequentists' idea}
	Existence of population probability \(P(A) \in [0, 1]\), that is not random, where 
	\[P(A) = \lim_{n \to \infty} f_n(A)\].
\end{thm}
\noindent However, this limit cannot be deterministic. For example, bad events like:
\[
B_n = \{|f_n(A) - P(A)| \geq \epsilon\}
\]
may still occur for large \(n\), which paves the way for the laws of large numbers (LLN) to control \(B_n\).


\begin{rmk}
For all \(\epsilon > 0\), \(\Pr(B_n \text{ happens}) \to 0\) as \(n \to \infty\), where:
\[
B_{n, \ep} = \{|f_n(A) - P(A)| \geq \epsilon\}.
\]
\end{rmk}

\subsubsection{Probability Theory vs. Statistical Inference}

\begin{rmk}{Probability Theory:} 
	For given \(P(A)\), compute the probability that a future series of \(n\) events lies in an interval:  
    \[
    f_n(A) \in [P(A) - \epsilon, P(A) + \epsilon].
    \]
\end{rmk}
\begin{rmk}{Statistical Inference:}  
For given statistical evidence, 
\begin{enumerate}
	\item A point estimate for $P(A)$: 
    	\[
   		\lim_{n\to\infty} f_n(A) = P(A)
   		\]
	\item A \textbf{confidence interval estimate} for \(P(A)\):  
    \[
   	CI_n = [f_n(A) - Z_n, f_n(A) + Z_n] 
    \]
    Such that $\lim_{n\to \infty}Pr\qty{P(A)\subset CI_n} \geq 1-\alpha$ (commonly $\alpha = 0.05$)
   
\end{enumerate}
	
\end{rmk}
So the two remarks are the inverse problems of each other. \\

\begin{df}{Estimator}
An \textbf{estimator} is a computational rule with given data (a function, or later defined as a random variable of data). 

	Independent experiments help reduce fluctuations by leveraging concentration of measure results.
\end{df}


\subsubsection{Bayesian Estimation}

\begin{thm}{Bayesian Idea}
The unknown \(p = P(A)\) is itself random, and Statistical data reduces uncertainty of the parameter (information update).
\begin{itemize}
\item \textbf{Prior:} What we believe about \(P(A)\) before gathering data.
\item \textbf{Posterior:} Updated beliefs after gathering data.
\end{itemize}
\end{thm}

The Bernstein-von Mises Theorem might offer some insights reconciling the two schools of thoughts when $n$ is sufficiently large. 

\begin{thm}{Bernstein-von Mises Theorem}
The posterior distribution is independent of the prior
distribution (under some conditions) once the amount of information supplied by a sample of data
is large enough
\end{thm}
\end{document}