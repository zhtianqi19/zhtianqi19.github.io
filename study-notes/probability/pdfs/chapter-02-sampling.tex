\documentclass[12pt, letterpaper]{article}

\usepackage[utf8]{inputenc}
\usepackage[framemethod=TikZ]{mdframed}
\usepackage[hidelinks]{hyperref}
\usepackage{mathtools, amssymb, amsmath, cleveref, fancyhdr, geometry, tcolorbox, graphicx, float, subfigure, arydshln, url, setspace, framed, pifont, physics, ntheorem, cancel, mathrsfs}


%%% for coding %%%
\usepackage{listings}
\usepackage[ruled, vlined, linesnumbered]{algorithm2e}
\SetKwComment{Comment}{/* }{ */}
\newcommand\mycommfont[1]{\small\ttfamily\textcolor{mygreen}{#1}}
\SetCommentSty{mycommfont}

\geometry{letterpaper, left=2cm, right=2cm, bottom=2cm, top=2cm}

\pagestyle{fancy}
\fancyhead{}
\fancyhead[L]{\leftmark}
\fancyhead[R]{\rightmark}
\fancyfoot{}
\fancyfoot[C]{\thepage}
%\rfoot{\footnotesize  Tianqi Zhang}


%\renewcommand{\headrulewidth}{0pt}
\renewcommand{\footrulewidth}{0pt}

\hypersetup{
	colorlinks = true,
	bookmarks = true,
	bookmarksnumbered = true,
	pdfborder = 001,
	linkcolor = blue
}

\definecolor{emoryblue}{RGB}{1, 33, 105} 
\definecolor{lightblue}{RGB}{0, 125, 186}
\definecolor{mediumblue}{RGB}{ 0, 51, 160}
\definecolor{darkblue}{RGB}{12, 35, 64}
\definecolor{red}{RGB}{185, 58, 38}
\definecolor{green}{RGB}{72, 127, 132}
\definecolor{gray1}{RGB}{217, 217, 214}
\definecolor{gray5}{RGB}{177, 179, 179}
\definecolor{gray3}{RGB}{208, 208, 206}

\definecolor{grey}{rgb}{0.49,0.38,0.29}
\definecolor{mygreen}{rgb}{0,0.6,0}
\definecolor{grey}{rgb}{0.49,0.38,0.29}
\definecolor{mygreen}{rgb}{0,0.6,0}


%%% for coding %%%
\lstset{basicstyle = \ttfamily\small,commentstyle = \color{mygreen}\textit, deletekeywords = {...}, escapeinside = {\%*}{*)}, frame = single, framesep = 0.5em, keywordstyle = \bfseries\color{blue}, morekeywords = {*}, emph = {self}, emphstyle=\bfseries\color{red}, numbers = left, numbersep = 1.5em, numberstyle = \ttfamily\small\color{grey},  rulecolor = \color{black}, showstringspaces = false, stringstyle = \ttfamily\color{purple}, tabsize = 4, columns = flexible}


\newcounter{index}[subsection]
\setcounter{index}{0}
\newenvironment*{df}[1]{\noindent\textbf{Definition \thesubsection.\stepcounter{index}\theindex\ (#1).}}{\\}

%\newenvironment*{eg}[1]{\begin{framed}\\\noindent\textbf{Example \thesubsection.\stepcounter{index}\theindex\ #1}\\ }{\\\end{framed}}

\newenvironment*{eg}[1]{
    \refstepcounter{index} % Increment the example counter
    \begin{framed}
    \noindent\textbf{Example \thesubsection.\theindex\ #1}
}{
    \end{framed}
}
%\newenvironment*{thm}[1]{\begin{tcolorbox}{\textbf{Theorem \thesubsection.\stepcounter{index}\theindex\ {#1}}}\\}{\\\end{tcolorbox}}
%\newenvironment*{cor}[1]{\noindent\textbf{Corollary \thesubsection.\stepcounter{index}\theindex\ #1:}}{\\}
%\newenvironment*{lem}[1]{\noindent\textbf{Lemma \thesubsection.\stepcounter{index}\theindex\ #1:}}{\\}
%\newenvironment*{ax}[1]{\noindent\textbf{Axiom \thesubsection.\stepcounter{index}\theindex\ #1:}}{\\}
%\newenvironment*{prop}[1]{\noindent\textbf{Proposition \thesubsection.\stepcounter{index}\theindex\ #1:}}{\\}
%\newenvironment*{conj}[1]{\noindent\textbf{Conjecture \thesubsection.\stepcounter{index}\theindex\ #1:}}{\\}
%\newenvironment*{nota}{\noindent\textbf{Notation \thesubsection.\stepcounter{index}\theindex.}}{\\}
%\newenvironment*{clm}{\noindent\textbf{Claim \thesubsection.\stepcounter{index}\theindex}}{\\}

% Ensure proper grouping and formatting for compatibility with lists
\newenvironment*{thm}[1]{%
  \begin{tcolorbox}%
  \textbf{Theorem \thesubsection.\stepcounter{index}\theindex\ {#1}}%
  \par\noindent%
}{%
  \end{tcolorbox}%
}

\newenvironment*{cor}[1]{%
  \par\noindent\textbf{Corollary \thesubsection.\stepcounter{index}\theindex\ {#1}:}%
  \par\noindent%
}{%
  \par%
}

\newenvironment*{lem}[1]{%
  \par\noindent\textbf{Lemma \thesubsection.\stepcounter{index}\theindex\ {#1}:}%
  \par\noindent%
}{%
  \par%
}

\newenvironment*{ax}[1]{%
  \par\noindent\textbf{Axiom \thesubsection.\stepcounter{index}\theindex\ {#1}:}%
  \par\noindent%
}{%
  \par%
}

\newenvironment*{prop}[1]{%
  \par\noindent\textbf{Proposition \thesubsection.\stepcounter{index}\theindex\ {#1}:}%
  \par\noindent%
}{%
  \par%
}

\newenvironment*{conj}[1]{%
  \par\noindent\textbf{Conjecture \thesubsection.\stepcounter{index}\theindex\ {#1}:}%
  \par\noindent%
}{%
  \par%
}

\newenvironment*{nota}{%
  \par\noindent\textbf{Notation \thesubsection.\stepcounter{index}\theindex:}%
  \par\noindent%
}{%
  \par%
}

\newenvironment*{clm}{%
  \par\noindent\textbf{Claim \thesubsection.\stepcounter{index}\theindex:}%
  \par\noindent%
}{%
  \par%
}


\newcounter{nprf}[subsection]
\setcounter{nprf}{0}
\newenvironment*{prf}{\noindent\textbf{\textit{Proof \stepcounter{nprf}\thenprf.}}}{\hfill$\blacksquare$\\}
\newenvironment*{dis}{\indent\textbf{\textit{Disproof \stepcounter{nprf}\thenprf.}}}{\hfill$\blacksquare$\\}
\newenvironment*{sol}{\indent\textbf{\textit{Solution \stepcounter{nprf}\thenprf.}}\\}{\hfill{$\square$}\\}

\newenvironment*{prf*}{\noindent\textit{Proof.}\ }{$\qquad\square$\\}
\newenvironment*{dis*}{\indent\textit{Disproof.}\ }{$\qquad\square$\\}
\newenvironment*{sol*}{\indent\textit{Solution.}\ }{$\qquad\square$\\}

\newtheorem{hint}{Hint}[section]
\newtheorem{rmk}{Remark}[section]
\newtheorem{ext}{Extension}[section]

\newtheorem*{df*}{Definition}
\newtheorem*{thm*}{Theorem}
\newtheorem*{clm*}{Claim}
\newtheorem*{cor*}{Corollary}
\newtheorem*{lem*}{Lemma}
\newtheorem*{ax*}{Axiom}
\newtheorem*{prop*}{Proposition}
\newtheorem*{conj*}{Conjecture}
\newtheorem*{nota*}{Notation}

\linespread{1.25}

\newcommand{\inprod}[2]{\left\langle #1, #2 \right\rangle}

\def\Z{{\mathbb{Z}}}
\def\H{{\mathcal{H}}}
\def\M{{\mathcal{M}}}
\def\R{{\mathbb{R}}}
\def\C{{\mathbb{C}}}
\def\Q{{\mathbb{Q}}}
\def\d{{\mathrm{d}}}
\def\i{{\mathrm{i}}}
\def\ep{{\varepsilon}}
\def\N{\mathbb{N}}
\def\1{\mathds{1}}
\def\bigO{\mathcal{O}}
\def\sp{\operatorname{span}}
\def\epsilon{\varepsilon}
\def\emptyset{\varnothing}
\def\phi{\varphi}
\def\dsst{\displaystyle}
\def\st{\ s.t.\ }
\def\wrt{\ w.r.t.\ }
\def\bar{\overline}
\def\tilde{\widetilde}
\def\E{\vb{E}}
\def\B{\vb{B}}
\def\L{\vb{L}}
\def\I{\vb{I}}
\def\Var{\vb{Var}}
\def\V{\vb{Var}}
\def\Cov{\vb{Cov}}
\def\MSE{\vb{MSE}}
\def\P{\vb{P}}
\def\M{\vb{M}}
\def\iid{i.i.d.}
\def\argmax{\arg\max}
\def\argmin{\arg\min}
\def\l{\ell}
\def\hat{\widehat}
\def\independ{\perp\!\!\!\perp}
\def\depend{\leftrightsquigarrow}
\def\residual{\varepsilon}
\def\sd{\mathrm{sd}}
\def\LI{\mathrm{L.I.}}
\def\range{\operatorname{range}}
\def\Null{\operatorname{null}}
\def\nullity{\operatorname{nullity}}
\def\A{A^{-1}}
\def\alg{\operatorname{alg}}
\def\fl{\operatorname{fl}}
\def\algmult{\operatorname{alg. mult.}}
\def\geomult{\operatorname{geo. mult.}}
\def\diag{\operatorname{diag}}
\def\gap{\operatorname{gap}}
\def\pqde{\quad\square}
\def\lub{\operatorname{lub}}
\def\Int{\operatorname{int}}
\def\ac{\operatorname{ac}}
\def\cl{\operatorname{cl}}
\def\bd{\operatorname{bd}}
\DeclareMathOperator*{\plim}{plim}
\def\upint{\mathchoice%
    {\mkern13mu\overline{\vphantom{\intop}\mkern7mu}\mkern-20mu}%
    {\mkern7mu\overline{\vphantom{\intop}\mkern7mu}\mkern-14mu}%
    {\mkern7mu\overline{\vphantom{\intop}\mkern7mu}\mkern-14mu}%
    {\mkern7mu\overline{\vphantom{\intop}\mkern7mu}\mkern-14mu}%
  \int}
\def\lowint{\mkern3mu\underline{\vphantom{\intop}\mkern7mu}\mkern-10mu\int}




\title{\textbf{% ECON 620\\
               Probability and Statistical Inference}}
\author{Tianqi Zhang\\
Emory University}
\date{Apr 17th 2025}

\begin{document}

\newpage
\maketitle
\setcounter{tocdepth}{1} % Only show sections in the table of contents

\setcounter{section}{1}
\section{Sampling}
\subsection{Sample Spaces and Probability Models}

To describe randomness formally, we need a language that specifies all the possible outcomes and assigns probabilities to them. This is the role of a probability model. \\

\begin{df}{Sample Space}
A \textbf{sample space} $\Omega$ is the set of all possible outcomes of a random experiment. A probability model also requires a rule that assigns probabilities to events, which are subsets of $\Omega$.
\end{df}

Let’s look at a few examples to make this concrete.

\begin{eg}{Coin Tossing}
In a single coin toss, the outcome is either heads (H) or tails (T). We write:
\[
\Omega = \{\text{H}, \text{T}\}.
\]
If the coin is fair, it makes sense to assign:
\[
P(\text{H}) = P(\text{T}) = \frac{1}{2}.
\]
We can extend this assignment to events involving multiple outcomes. For example, $$P(\{\text{H or T}\}) = 1$$
This kind of modeling is supported by empirical data. By the \textbf{Law of Large Numbers (LLN)}, if we toss the coin many times and track how often heads appears, we should see the relative frequency stabilize around 0.5:
\[
P(A) \approx \lim_{n \to \infty} f_n(A).
\]
We’ll revisit this idea — and formally define the limit $\plim$ — later.
\end{eg}


\subsection{Uniform Probability Models and Discrete Distributions}

The coin toss is an example of a uniform probability model: each outcome is equally likely. Let’s generalize this to any finite or countable space.

\begin{eg}{Uniform Distribution}
Suppose the sample space is:
\[
\Omega = \{\omega_1, \omega_2, \dots, \omega_N\}.
\]
Then a uniform probability model assigns equal weight to each outcome:
\[
P(\omega_i) = \frac{1}{N}, \quad \text{for all } i = 1, \dots, N.
\]
More generally, for any event \( A \subseteq \Omega \), we define:
\[
P(A) = \frac{|A|}{|\Omega|}.
\]

\noindent This approach works well for simple random experiments like rolling a die. For example:
\[
\Omega = \{1, 2, 3, 4, 5, 6\}, \quad P(i) = \frac{1}{6}.
\]
Let $A$ be the event that the outcome is even. Then:
\[
A = \{2, 4, 6\}, \quad P(A) = \frac{3}{6} = \frac{1}{2}.
\]
\end{eg}

\begin{rmk}{On Notation}
Even though we write $P(\omega_i)$ for convenience, we are technically assigning probability to the set $\{\omega_i\}$, not the element itself. That is,
\[
P(\omega_i) \equiv P(\{\omega_i\}),
\]
since events are subsets of $\Omega$.
\end{rmk}



\subsection{Why the Sample Space Matters}

In some problems, how we define $\Omega$ completely changes the model.

\begin{eg}{The Twin Paradox}
Suppose a family has two children, and we’re told that one of them is a girl. What is the probability that both are girls?

Let’s break it into two interpretations:

\textbf{Case 1:} We are told the gender of the first child. The second child is still random:
\[
\Omega = \{\text{G}, \text{B}\}, \quad P(\text{G}) = \frac{1}{2}.
\]

\textbf{Case 2:} We are told at least one child is a girl, but not which one. Then the possible combinations are:
\[
\Omega = \{\text{GG}, \text{GB}, \text{BG}\}.
\]
Each outcome is equally likely, so:
\[
P(\text{GG}) = \frac{1}{3}.
\]

This example illustrates how the structure of $\Omega$ — and what we know — affects the probability model.
\end{eg}


\subsection{Bernoulli Trials and the Binomial Distribution}

Let’s now consider what happens when we repeat a simple random experiment multiple times — like flipping a coin, answering a yes/no question, or testing whether a lightbulb works.\\

\begin{df}{Bernoulli Trial}
A \textbf{Bernoulli trial} is a random experiment with only two possible outcomes, usually called “success” (1) and “failure” (0). For example, in a coin flip:

\[
\Omega = \{0, 1\}, \quad P(1) = p, \quad P(0) = 1 - p.
\]
Here, $p$ is the probability of success.
\end{df}

Now suppose we repeat the experiment $n$ times independently. The full sample space is then:
\[
\Omega^n = \{0, 1\}^n,
\]
which consists of all sequences of 0s and 1s of length $n$. For example, a possible outcome could be $(1, 0, 1, 1, 0)$ — meaning 3 successes and 2 failures.

We define a function that counts the number of successes: Let $S_n: \Omega^n \to \{0, 1, \dots, n\}$ be the function
\[
S_n(\omega) = \sum_{i=1}^n \omega_i.
\]
This counts how many of the $n$ trials resulted in success.\\


What is the probability that we get exactly $x$ successes in $n$ trials?

Each specific sequence with $x$ successes and $(n - x)$ failures has probability:
\[
p^x (1 - p)^{n - x}.
\]

And how many such sequences are there? That’s given by a binomial coefficient:

\begin{df}{Binomial Coefficient}
\[
\binom{n}{x} = \frac{n!}{x!(n - x)!}
\]
is the number of different sequences (orderings) that contain exactly $x$ successes in $n$ trials.
\end{df}

Putting this together, we get the \textbf{binomial distribution}:

\begin{df}{Binomial Distribution}
The probability of observing exactly $x$ successes in $n$ independent Bernoulli trials, each with success probability $p$, is:
\[
P(X = x) = \binom{n}{x} p^x (1 - p)^{n - x}, \quad x = 0, 1, \dots, n.
\]
This is called the \textbf{binomial distribution} with parameters $n$ and $p$, written:
\[
X \sim Binom(n, p).
\]
\end{df}

\begin{rmk}
The binomial formula accounts for both the probability of each individual sequence and the number of such sequences. The sum of all probabilities over $x = 0$ to $n$ equals 1.
\end{rmk}



\end{document}