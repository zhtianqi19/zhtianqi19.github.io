\usepackage[utf8]{inputenc}
\usepackage[framemethod=TikZ]{mdframed}
\usepackage[hidelinks]{hyperref}
\usepackage{mathtools, amssymb, amsmath, cleveref, fancyhdr, geometry, tcolorbox, graphicx, float, subfigure, arydshln, url, setspace, framed, pifont, physics, ntheorem, cancel}
%%% for coding %%%
\usepackage{listings}
\usepackage[ruled, vlined, linesnumbered]{algorithm2e}
\SetKwComment{Comment}{/* }{ */}
\newcommand\mycommfont[1]{\small\ttfamily\textcolor{mygreen}{#1}}
\SetCommentSty{mycommfont}

\geometry{letterpaper, left=2cm, right=2cm, bottom=2cm, top=2cm}

\pagestyle{fancy}
\fancyhead{}
\fancyhead[L]{\leftmark}
\fancyhead[R]{\rightmark}
\fancyfoot{}
\fancyfoot[C]{\thepage}
%\renewcommand{\headrulewidth}{0pt}
\renewcommand{\footrulewidth}{0pt}

\hypersetup{
	colorlinks = true,
	bookmarks = true,
	bookmarksnumbered = true,
	pdfborder = 001,
	linkcolor = blue
}

\definecolor{emoryblue}{RGB}{1, 33, 105} 
\definecolor{lightblue}{RGB}{0, 125, 186}
\definecolor{mediumblue}{RGB}{ 0, 51, 160}
\definecolor{darkblue}{RGB}{12, 35, 64}
\definecolor{red}{RGB}{185, 58, 38}
\definecolor{green}{RGB}{72, 127, 132}
\definecolor{gray1}{RGB}{217, 217, 214}
\definecolor{gray5}{RGB}{177, 179, 179}
\definecolor{gray3}{RGB}{208, 208, 206}

\definecolor{grey}{rgb}{0.49,0.38,0.29}
\definecolor{mygreen}{rgb}{0,0.6,0}

%%% for coding %%%
\lstset{basicstyle = \ttfamily\small,commentstyle = \color{mygreen}\textit, deletekeywords = {...}, escapeinside = {\%*}{*)}, frame = single, framesep = 0.5em, keywordstyle = \bfseries\color{blue}, morekeywords = {*}, emph = {self}, emphstyle=\bfseries\color{red}, numbers = left, numbersep = 1.5em, numberstyle = \ttfamily\small\color{grey},  rulecolor = \color{black}, showstringspaces = false, stringstyle = \ttfamily\color{purple}, tabsize = 4, columns = flexible}

\newcounter{index}[section]
\renewcommand{\theindex}{\thesection.\arabic{index}}

\newenvironment*{df}[1]{%
  \stepcounter{index}%
  \noindent\textbf{Definition \theindex\ (#1).}%
}{\\}

\newenvironment*{eg}[1]{%
  \stepcounter{index}%
  \begin{framed}\noindent\textbf{Example \theindex\ #1}\\ 
}{%
  \end{framed}
}

\newenvironment*{thm}[1]{%
  \stepcounter{index}%
  \begin{tcolorbox}[colback=gray!5, colframe=gray!40!black,
    title=Theorem \theindex\ (#1)]%
}{%
  \end{tcolorbox}
}

\newenvironment*{cor}[1]{\stepcounter{index}\noindent\textbf{Corollary \theindex\ (#1):}}{\\}
\newenvironment*{lem}[1]{\stepcounter{index}\noindent\textbf{Lemma \theindex\ (#1):}}{\\}
\newenvironment*{ax}[1]{\stepcounter{index}\noindent\textbf{Axiom \theindex\ (#1):}}{\\}
\newenvironment*{prop}[1]{\stepcounter{index}\noindent\textbf{Proposition \theindex\ (#1):}}{\\}
\newenvironment*{conj}[1]{\stepcounter{index}\noindent\textbf{Conjecture \theindex\ (#1):}}{\\}
\newenvironment*{nota}{\stepcounter{index}\noindent\textbf{Notation \theindex.}}{\\}
\newenvironment*{clm}{\stepcounter{index}\noindent\textbf{Claim \theindex}}{\\}

\newcounter{nprf}[section]
\setcounter{nprf}{0}
\newenvironment*{prf}{\noindent\textbf{\textit{Proof \stepcounter{nprf}\thenprf.}}}{\hfill$\blacksquare$\\}
\newenvironment*{dis}{\indent\textbf{\textit{Disproof \stepcounter{nprf}\thenprf.}}}{\hfill$\blacksquare$\\}
\newenvironment*{sol}{\indent\textbf{\textit{Solution \stepcounter{nprf}\thenprf.}}\\}{\hfill{$\square$}\\}

\newenvironment*{prf*}{\noindent\textit{Proof.}\ }{$\qquad\square$\\}
\newenvironment*{dis*}{\indent\textit{Disproof.}\ }{$\qquad\square$\\}
\newenvironment*{sol*}{\indent\textit{Solution.}\ }{$\qquad\square$\\}

\newtheorem{hint}{Hint}[section]
\newtheorem{rmk}{Remark}[section]
\newtheorem{ext}{Extension}[section]

\newtheorem*{df*}{Definition}
\newtheorem*{thm*}{Theorem}
\newtheorem*{clm*}{Claim}
\newtheorem*{cor*}{Corollary}
\newtheorem*{lem*}{Lemma}
\newtheorem*{ax*}{Axiom}
\newtheorem*{prop*}{Proposition}
\newtheorem*{conj*}{Conjecture}
\newtheorem*{nota*}{Notation}

\linespread{1.25}

\newcommand{\inprod}[2]{\left\langle #1, #2 \right\rangle}

\def\Z{{\mathbb{Z}}}
\def\R{{\mathbb{R}}}
\def\C{{\mathbb{C}}}
\def\Q{{\mathbb{Q}}}
\def\d{{\mathrm{d}}}
\def\i{{\mathrm{i}}}
\def\ep{{\varepsilon}}
\def\N{\mathbb{N}}
\def\1{\mathds{1}}
\def\bigO{\mathcal{O}}
\def\sp{\operatorname{span}}
\def\epsilon{\varepsilon}
\def\emptyset{\varnothing}
\def\phi{\varphi}
\def\dsst{\displaystyle}
\def\st{\ s.t.\ }
\def\wrt{\ w.r.t.\ }
\def\bar{\overline}
\def\tilde{\widetilde}
\def\E{\mathbb{E}}
\def\B{\vb{B}}
\def\L{\vb{L}}
\def\I{\vb{I}}
\def\Var{\vb{Var}}
\def\V{\vb{Var}}
\def\Cov{\vb{Cov}}
\def\MSE{\vb{MSE}}
\def\P{\vb{P}}
\def\M{\vb{M}}
\def\iid{i.i.d.}
\def\argmax{\arg\max}
\def\argmin{\arg\min}
\def\l{\ell}
\def\hat{\widehat}
\def\independ{\perp\!\!\!\perp}
\def\depend{\leftrightsquigarrow}
\def\residual{\varepsilon}
\def\sd{\mathrm{sd}}
\def\LI{\mathrm{L.I.}}
\def\range{\operatorname{range}}
\def\Null{\operatorname{null}}
\def\nullity{\operatorname{nullity}}
\def\A{A^{-1}}
\def\alg{\operatorname{alg}}
\def\fl{\operatorname{fl}}
\def\algmult{\operatorname{alg. mult.}}
\def\geomult{\operatorname{geo. mult.}}
\def\diag{\operatorname{diag}}
\def\gap{\operatorname{gap}}
\def\pqde{\quad\square}
\def\lub{\operatorname{lub}}
\def\Int{\operatorname{int}}
\def\ac{\operatorname{ac}}
\def\cl{\operatorname{cl}}
\def\bd{\operatorname{bd}}
\def\upint{\mathchoice%
    {\mkern13mu\overline{\vphantom{\intop}\mkern7mu}\mkern-20mu}%
    {\mkern7mu\overline{\vphantom{\intop}\mkern7mu}\mkern-14mu}%
    {\mkern7mu\overline{\vphantom{\intop}\mkern7mu}\mkern-14mu}%
    {\mkern7mu\overline{\vphantom{\intop}\mkern7mu}\mkern-14mu}%
  \int}
\def\lowint{\mkern3mu\underline{\vphantom{\intop}\mkern7mu}\mkern-10mu\int}


\title{\textbf{% ECON 620\\
               Probability and Statistical Inference}}
\author{Tianqi Zhang\\
Emory University}
\date{Apr 17th 2025}

\begin{document}

\newpage
\maketitle
\setcounter{tocdepth}{1} % Only show sections in the table of contents

\setcounter{section}{1}
\section{Sampling}
\subsection{Sample Space and Probability Models}
\begin{df}{Sample Space}\\
The specification of a probability model requires:
\begin{enumerate}
	\item A sample space, \(\Omega\): The set of all possible outcomes in the problem.
	\item A probability assignment for these outcomes (subsets of $\Omega$).
\end{enumerate}
\
\end{df}
Consider the following examples to help understanding the construction of probability models: 
\begin{eg}{Coin Tossing}
\[
\Omega = \{\text{H}, \text{T}\}.
\]
\begin{itemize}
    \item If assuming the coin is balanced, we state \(P(\text{H}) = P(\text{T}) = \frac{1}{2}\).
    \item This assumption can be extended to all subsets of \(\Omega\).
    \item Assumption can be justified empirically using the \textbf{Law of Large Numbers (LLN)}:  
    \[
    P(A) = \plim_{n \to \infty} f_n(A).
    \]
    For a proper definition of "$\plim$"
\end{itemize}
\end{eg}

To generalize the above results: \\

\begin{df}{Uniform probability measure/distribution}
\begin{itemize}
    \item \(\Omega = \qty{\omega_1, \dots, \omega_N}\) or countable
    \item \textbf{Uniform probability measure (distribution)}:  
    For any \(A \subseteq \Omega\),  
    \[
    P(A) = \frac{|A|}{|\Omega|}.
    \]
    and define 
    \[P(\omega_i) = \frac{1}{N}, \forall i \in \{1, 2, \dots, N\}\] 
    \item \textbf{Warning on abuse of notation:} Note that this is not a proper probability measure, as it assigns probabilities to individual elements, not subsets.
    $$P(\omega_i) \equiv P(\qty{\omega_i})$$
    even though $\omega_i\in \Omega$, $\qty{\omega_i}\subseteq \Omega$, and $\qty{\omega_i} \in \mathscr{P}(\Omega)$ as the power set.
    	
\end{itemize}
Therefore, we have the above function $P$ defined to be $ \mathscr{P}(\Omega) \rightarrow [0, 1]$ 
\end{df}

\begin{eg}{Fair Die}
\[
\Omega = \{1, 2, 3, 4, 5, 6\}.
\]
\begin{itemize}
    \item Let \(A =\) "even numbers" \( = \{2, 4, 6\}\).  
    Then:
    \[
    P(A) = \frac{|A|}{|\Omega|} = \frac{3}{6} = \frac{1}{2}.
    \]
\end{itemize}
\end{eg}


\begin{rmk}{Conditioning and Specifying $\Omega$ is crucial}

\end{rmk}
\begin{eg}{Twin paradox example}
	\begin{itemize}
    \item \textbf{Assumptions:}
    \begin{itemize}
        \item (i) Gender of newborn: \(P(\text{Girl}) = P(\text{Boy}) = \frac{1}{2}\).
        \item (ii) Gender of one child is independent of the gender of the other child.
    \end{itemize}
    \item Known: A family has two children, one of whom is a girl.  
    \item Question: What is \(P(\text{both children are girls})\)?
    \begin{itemize}
        \item Case (i): If we know the gender of the first child, then $\Omega$ considers only the second child
        \[
        \Omega_i = \{\text{G}, \text{B}\}.
        \]
        Then:
        \[
        P(\text{G}) = \frac{1}{2}.
        \]
        \item Case (ii): If we know at least one child is a girl:  
        \[
        \Omega_{ii} = \{\text{GG}, \text{GB}, \text{BG}\}.
        \]
        Then:
        \[
        P(\text{GG}) = \frac{1}{3}.
        \]
    \end{itemize}
\end{itemize}
\end{eg}

\subsection{Bernoulli Trials and Binomial Distribution}

\begin{eg}{Bernoulli Distribution}
\begin{itemize}
\item Suppose that we have a sequence of \(n\) independent experiments
$$P(\text{Success}) = p,\ P(\text{Failure}) = 1 - p$$
\item We have the sample space: 
	\[
    \Omega = \{0, 1\}^n \quad \text{(Cartesian product of the simple $\qty{0, 1}$)}.
    \]
\item A typical element (draw) \(\omega \in \Omega\) is \(\omega = (\omega_1, \omega_2, \dots, \omega_n)\) as a sequence.
\item Define $S_n: \Omega \rightarrow \R$, $S_n(\omega) = \sum_{i=1}^n \omega_i$ to be the count of success. 
\end{itemize}
\end{eg}

\begin{clm}
Then the \textbf{independent experiment} with its probability measure: 
$$P(\omega) = p^{S_n(\omega)}(1-p)^{n-S_n(\omega)}$$
\end{clm}

\begin{rmk}
\(S_n\) is a \textbf{random variable}, a function from $\Omega$ endowed with the probability measure $P$ and maps outcomes to \(\mathcal{S} \equiv \{0, 1, \dots, n\}\).	
\end{rmk}

\noindent Since the original Bernoulli trial is simple enough to be binary: $\Omega = \qty{0, 1}$, the count of success can be reduced into the simple sum in the sequence. Therefore, we can then define a new probability measure (distribution) $P^{S_n}$ on $S$ such that: 
\begin{thm}{Push-Forward Measure}
We can then define a new probability measure (distribution) $P^{S_n}$ on $S$ such that
$$\forall x\in S, P(x) \equiv \sum_{\omega\in S_n^{-1}(x)} P(\omega) = P\qty[S_n^{-1}(x)] \equiv P^{S_n}(x)$$
Where 
$$S_n^{-1}(A) \equiv \qty{\omega\in \Omega, S_n(\omega) \in A}\quad \text{is the pre-image}$$
\end{thm}
 
 \noindent For computation purposes: 
 $$\forall x\in \mathcal{S}, \ P^{S_n}(x) = \abs{S_n^{-1}(x)} p^x(1-p)^{n-x}$$
 Intuitively speaking, the pre-image $S_n^{-1}(x)$ indicates all possible sequences $\omega$ such that they contain exactly $x$ counts of success. We give the calculation as follows

\begin{df}{Binomial number}
$$\abs{S_n^{-1}(x)} = \binom{n}{x} = \frac{n!}{x!(n-x)!} = \binom{n}{n-x}$$
To be the number of ways to pick $x$ elements from $n$ elements without order. 	
\end{df}


\begin{df}{Binomial Distribution}\\
Binomial Distribution: $Binom(n, p)$ is the push-forward probability measure of the Bernoulli trial with $n$ experiments and success probability $p$ via the random variable $S_n$ which counts success. 
$$\forall x \in \mathcal{S}, P^{S_n}(x) =  \binom{n}{x} p^x(1-p)^{n-x}$$
and
  \[\sum_{n=0}^n \binom{n}{x} p^x (1-p)^{n-x} = 1\]
\end{df}

\noindent We say the \textbf{random variable $S_n$ follows the binomial distribution}, $S_n\sim Binom(n, p)$.  

\end{document}