\documentclass[12pt, letterpaper]{article}

\usepackage[utf8]{inputenc}
\usepackage[framemethod=TikZ]{mdframed}
\usepackage[hidelinks]{hyperref}
\usepackage{mathtools, amssymb, amsmath, cleveref, fancyhdr, geometry, tcolorbox, graphicx, float, subfigure, arydshln, url, setspace, framed, pifont, physics, ntheorem, cancel, mathrsfs}


%%% for coding %%%
\usepackage{listings}
\usepackage[ruled, vlined, linesnumbered]{algorithm2e}
\SetKwComment{Comment}{/* }{ */}
\newcommand\mycommfont[1]{\small\ttfamily\textcolor{mygreen}{#1}}
\SetCommentSty{mycommfont}

\geometry{letterpaper, left=2cm, right=2cm, bottom=2cm, top=2cm}

\pagestyle{fancy}
\fancyhead{}
\fancyhead[L]{\leftmark}
\fancyhead[R]{\rightmark}
\fancyfoot{}
\fancyfoot[C]{\thepage}
%\rfoot{\footnotesize  Tianqi Zhang}


%\renewcommand{\headrulewidth}{0pt}
\renewcommand{\footrulewidth}{0pt}

\hypersetup{
	colorlinks = true,
	bookmarks = true,
	bookmarksnumbered = true,
	pdfborder = 001,
	linkcolor = blue
}

\definecolor{emoryblue}{RGB}{1, 33, 105} 
\definecolor{lightblue}{RGB}{0, 125, 186}
\definecolor{mediumblue}{RGB}{ 0, 51, 160}
\definecolor{darkblue}{RGB}{12, 35, 64}
\definecolor{red}{RGB}{185, 58, 38}
\definecolor{green}{RGB}{72, 127, 132}
\definecolor{gray1}{RGB}{217, 217, 214}
\definecolor{gray5}{RGB}{177, 179, 179}
\definecolor{gray3}{RGB}{208, 208, 206}

\definecolor{grey}{rgb}{0.49,0.38,0.29}
\definecolor{mygreen}{rgb}{0,0.6,0}
\definecolor{grey}{rgb}{0.49,0.38,0.29}
\definecolor{mygreen}{rgb}{0,0.6,0}


%%% for coding %%%
\lstset{basicstyle = \ttfamily\small,commentstyle = \color{mygreen}\textit, deletekeywords = {...}, escapeinside = {\%*}{*)}, frame = single, framesep = 0.5em, keywordstyle = \bfseries\color{blue}, morekeywords = {*}, emph = {self}, emphstyle=\bfseries\color{red}, numbers = left, numbersep = 1.5em, numberstyle = \ttfamily\small\color{grey},  rulecolor = \color{black}, showstringspaces = false, stringstyle = \ttfamily\color{purple}, tabsize = 4, columns = flexible}


\newcounter{index}[subsection]
\setcounter{index}{0}
\newenvironment*{df}[1]{\noindent\textbf{Definition \thesubsection.\stepcounter{index}\theindex\ (#1).}}{\\}

%\newenvironment*{eg}[1]{\begin{framed}\\\noindent\textbf{Example \thesubsection.\stepcounter{index}\theindex\ #1}\\ }{\\\end{framed}}

\newenvironment*{eg}[1]{
    \refstepcounter{index} % Increment the example counter
    \begin{framed}
    \noindent\textbf{Example \thesubsection.\theindex\ #1}
}{
    \end{framed}
}
%\newenvironment*{thm}[1]{\begin{tcolorbox}{\textbf{Theorem \thesubsection.\stepcounter{index}\theindex\ {#1}}}\\}{\\\end{tcolorbox}}
%\newenvironment*{cor}[1]{\noindent\textbf{Corollary \thesubsection.\stepcounter{index}\theindex\ #1:}}{\\}
%\newenvironment*{lem}[1]{\noindent\textbf{Lemma \thesubsection.\stepcounter{index}\theindex\ #1:}}{\\}
%\newenvironment*{ax}[1]{\noindent\textbf{Axiom \thesubsection.\stepcounter{index}\theindex\ #1:}}{\\}
%\newenvironment*{prop}[1]{\noindent\textbf{Proposition \thesubsection.\stepcounter{index}\theindex\ #1:}}{\\}
%\newenvironment*{conj}[1]{\noindent\textbf{Conjecture \thesubsection.\stepcounter{index}\theindex\ #1:}}{\\}
%\newenvironment*{nota}{\noindent\textbf{Notation \thesubsection.\stepcounter{index}\theindex.}}{\\}
%\newenvironment*{clm}{\noindent\textbf{Claim \thesubsection.\stepcounter{index}\theindex}}{\\}

% Ensure proper grouping and formatting for compatibility with lists
\newenvironment*{thm}[1]{%
  \begin{tcolorbox}%
  \textbf{Theorem \thesubsection.\stepcounter{index}\theindex\ {#1}}%
  \par\noindent%
}{%
  \end{tcolorbox}%
}

\newenvironment*{cor}[1]{%
  \par\noindent\textbf{Corollary \thesubsection.\stepcounter{index}\theindex\ {#1}:}%
  \par\noindent%
}{%
  \par%
}

\newenvironment*{lem}[1]{%
  \par\noindent\textbf{Lemma \thesubsection.\stepcounter{index}\theindex\ {#1}:}%
  \par\noindent%
}{%
  \par%
}

\newenvironment*{ax}[1]{%
  \par\noindent\textbf{Axiom \thesubsection.\stepcounter{index}\theindex\ {#1}:}%
  \par\noindent%
}{%
  \par%
}

\newenvironment*{prop}[1]{%
  \par\noindent\textbf{Proposition \thesubsection.\stepcounter{index}\theindex\ {#1}:}%
  \par\noindent%
}{%
  \par%
}

\newenvironment*{conj}[1]{%
  \par\noindent\textbf{Conjecture \thesubsection.\stepcounter{index}\theindex\ {#1}:}%
  \par\noindent%
}{%
  \par%
}

\newenvironment*{nota}{%
  \par\noindent\textbf{Notation \thesubsection.\stepcounter{index}\theindex:}%
  \par\noindent%
}{%
  \par%
}

\newenvironment*{clm}{%
  \par\noindent\textbf{Claim \thesubsection.\stepcounter{index}\theindex:}%
  \par\noindent%
}{%
  \par%
}


\newcounter{nprf}[subsection]
\setcounter{nprf}{0}
\newenvironment*{prf}{\noindent\textbf{\textit{Proof \stepcounter{nprf}\thenprf.}}}{\hfill$\blacksquare$\\}
\newenvironment*{dis}{\indent\textbf{\textit{Disproof \stepcounter{nprf}\thenprf.}}}{\hfill$\blacksquare$\\}
\newenvironment*{sol}{\indent\textbf{\textit{Solution \stepcounter{nprf}\thenprf.}}\\}{\hfill{$\square$}\\}

\newenvironment*{prf*}{\noindent\textit{Proof.}\ }{$\qquad\square$\\}
\newenvironment*{dis*}{\indent\textit{Disproof.}\ }{$\qquad\square$\\}
\newenvironment*{sol*}{\indent\textit{Solution.}\ }{$\qquad\square$\\}

\newtheorem{hint}{Hint}[section]
\newtheorem{rmk}{Remark}[section]
\newtheorem{ext}{Extension}[section]

\newtheorem*{df*}{Definition}
\newtheorem*{thm*}{Theorem}
\newtheorem*{clm*}{Claim}
\newtheorem*{cor*}{Corollary}
\newtheorem*{lem*}{Lemma}
\newtheorem*{ax*}{Axiom}
\newtheorem*{prop*}{Proposition}
\newtheorem*{conj*}{Conjecture}
\newtheorem*{nota*}{Notation}

\linespread{1.25}

\newcommand{\inprod}[2]{\left\langle #1, #2 \right\rangle}

\def\Z{{\mathbb{Z}}}
\def\H{{\mathcal{H}}}
\def\M{{\mathcal{M}}}
\def\R{{\mathbb{R}}}
\def\C{{\mathbb{C}}}
\def\Q{{\mathbb{Q}}}
\def\d{{\mathrm{d}}}
\def\i{{\mathrm{i}}}
\def\ep{{\varepsilon}}
\def\N{\mathbb{N}}
\def\1{\mathds{1}}
\def\bigO{\mathcal{O}}
\def\sp{\operatorname{span}}
\def\epsilon{\varepsilon}
\def\emptyset{\varnothing}
\def\phi{\varphi}
\def\dsst{\displaystyle}
\def\st{\ s.t.\ }
\def\wrt{\ w.r.t.\ }
\def\bar{\overline}
\def\tilde{\widetilde}
\def\E{\vb{E}}
\def\B{\vb{B}}
\def\L{\vb{L}}
\def\I{\vb{I}}
\def\Var{\vb{Var}}
\def\V{\vb{Var}}
\def\Cov{\vb{Cov}}
\def\MSE{\vb{MSE}}
\def\P{\vb{P}}
\def\M{\vb{M}}
\def\iid{i.i.d.}
\def\argmax{\arg\max}
\def\argmin{\arg\min}
\def\l{\ell}
\def\hat{\widehat}
\def\independ{\perp\!\!\!\perp}
\def\depend{\leftrightsquigarrow}
\def\residual{\varepsilon}
\def\sd{\mathrm{sd}}
\def\LI{\mathrm{L.I.}}
\def\range{\operatorname{range}}
\def\Null{\operatorname{null}}
\def\nullity{\operatorname{nullity}}
\def\A{A^{-1}}
\def\alg{\operatorname{alg}}
\def\fl{\operatorname{fl}}
\def\algmult{\operatorname{alg. mult.}}
\def\geomult{\operatorname{geo. mult.}}
\def\diag{\operatorname{diag}}
\def\gap{\operatorname{gap}}
\def\pqde{\quad\square}
\def\lub{\operatorname{lub}}
\def\Int{\operatorname{int}}
\def\ac{\operatorname{ac}}
\def\cl{\operatorname{cl}}
\def\bd{\operatorname{bd}}
\DeclareMathOperator*{\plim}{plim}
\def\upint{\mathchoice%
    {\mkern13mu\overline{\vphantom{\intop}\mkern7mu}\mkern-20mu}%
    {\mkern7mu\overline{\vphantom{\intop}\mkern7mu}\mkern-14mu}%
    {\mkern7mu\overline{\vphantom{\intop}\mkern7mu}\mkern-14mu}%
    {\mkern7mu\overline{\vphantom{\intop}\mkern7mu}\mkern-14mu}%
  \int}
\def\lowint{\mkern3mu\underline{\vphantom{\intop}\mkern7mu}\mkern-10mu\int}




\title{\textbf{% ECON 620\\
               Probability and Statistical Inference}}
\author{Tianqi Zhang\\
Emory University}
\date{Apr 17th 2025}

\begin{document}
\maketitle
\setcounter{tocdepth}{1} % Only show sections in the table of contents

\setcounter{section}{3}
\section{Lebesgue Measure}
\subsection{Motivation (Number Theory Version)}
Focus on half closed interval. Random draws $\omega \in [0,1]$: $\mathbb{P}(\omega \in [0,0.47)) = ?$
\begin{itemize}
    \item Pick a number: \(0.xy\ldots\)
    \item Either \(x \in \{0,1,2,3\}\) or \(x = 4 \cap y \in \{0, 1, \ldots, 0.6\}\).
\end{itemize}
\[
\#\text{ possibilities} = (4 \cdot 10) + (1 \cdot 7) = 47 \text{ out of } (10 \cdot 10) \text{ possibilities}.
\]
\[
\mathbb{P}(w \in [0,0.47]) = \frac{47}{100} = 0.47 \quad \text{\textcolor{teal}{Some shape of Uniform CDF}}.
\]

\subsection{Construction}
\begin{df}{Lebesgue Measure on $\R\cap [0,1]$}\\
More Generally, for \(a, b \in \mathbb{R} \cap [0,1]\), \(a < b\), Lebesgue Measure is a measure defined on the power set of $\mathbb{R} \cap [0,1]$ \textcolor{red}{(Warning! this is ideal but is not true. reason to be discussed in the next chapter.)} onto $\R$ such that the mapping is non-negative and sigma-additive. It is defined by the following: 
\[
\mathbb{P}([0,a)) = a, \quad \mathbb{P}([0,b)) = b.
\]
\[
\mathbb{P}([a,b)) = \mathbb{P}([0,b)) - \mathbb{P}([0,a)) = b - a.
\]
\end{df}

\begin{prop}{Additional Features on Lebesgue Measure}
\begin{itemize}
        \item Unity: $\mathbb{P}([0, 1]) = 1$ 
    	\item Translational Invariant: $\mathbb{P}(x + A) = \mathbb{P}(A), \ \forall x \in \mathbb{R}$
\end{itemize}
\end{prop}


\begin{prop}{}
\begin{enumerate}
    \item \textbf{Open interval: } \(\mathbb{P}((a,b)) = \lim_{n \to 0} \mathbb{P}((a+\frac{1}{n},b)) = \lim_{n \to 0} b-(a+\frac{1}{n}) = b-a\).
    \item \textbf{Single element: } \(\mathbb{P}(\qty{a}) = \mathbb{P}([a,b]) - \mathbb{P}((a,b)) = b-a - (b-a) = 0 \).
    \item \textbf{Closed interval: } \(\mathbb{P}([a,b]) = \mathbb{P}([a,b)) + \mathbb{P}(\{b\}) = \mathbb{P}([a,b]) = b - a.\)
\end{enumerate}
\end{prop}

\begin{rmk}
Using Kolmogorov axioms for measures, \(A \subset [0,1]\), finite or countable, \(\mathbb{P}(A) = 0\).
As a result, \(\mathbb{P}(\mathbb{Q} \cap [0,1]) = 0.\)
\end{rmk}

%
%\newpage
%\section*{Outer Measure}
%\subsection*{Motivation and Definition of Outer Measure}
%Recall the definition of upper and lower Darboux sum in Riemann integral setting: 
%$$U(f) = \sum_{i=1}^N \sup_{x\in [x_i, x_{i+1}]}{f(x)} \cdot \underbrace{(x_{i+1} - x_i)}_{\text{length}}$$
%Definition of Riemann non-integrable involves the upper integral and the lower integral as two limits do not agree, which likely boils down to some partitions $[x_i, x_{i+1}]$ not well-defined. Therefore, to propose a fix, we are motivated to properly define the "length" of any general subset of $\R$. \\
%
%\begin{df}{Length}
%    The length $\ell(I)$ of some open interval $I\subset \R$ is a function defined by
%    $$\ell(I) =
%    \begin{cases}
%         b-a, & I = (b-a), a<b, a, b\in \R\\
%         0,  & I = \emptyset, \\
%         \infty & I = (-\infty, a), a\in \R\\
%         \infty & I = (a, \infty), a\in \R
%    \end{cases}$$
%\end{df}
%Then suppose $A\subset \R$. The size of $A$ can at most be the \textbf{sum of lengths of a sequence of open intervals $I$ whose union contains} $A$. Taking the infimum of such sums over all possible sequences of $I$, we obtain the outer measure of $A$, i.e.\\
%
%\begin{df}{Outer Measure, $\abs{A}$}
%    For $A\subset \R$,
%    $$\abs{A} \equiv \inf\qty{\sum_{k=1}^\infty \ell(I_k)\mid I_k \text{ open, } A\subset \bigcup_{k=1}^\infty I_k }$$
%\end{df}
%
%\begin{prop}{Finite sets have outer measure 0} 
%\end{prop}
%
%\subsubsection{Properties of Outer Measure}
%\begin{prop}{Countable subsets of $\R$ have outer measure 0} 
%\end{prop}
%
%\begin{prop}{Order preserving of outer measure}\\
%If $A\subset B\subset \R$, then $\abs{A}\leq\abs{B}$
%\end{prop}
%
%\begin{df}{Translation}
%    For any $A\subset \R, t\in \R$, the translation $t+A$ is defined by
%    \[t+A = \qty{t+a \mid a\in A}\]
%\end{df}
%Note that the length function should be translation invariant. Therefore, we obtain the proposition that outer measure is translation invariant. \\
%
%\begin{prop}{Outer measure is translation invariant}\\
%Suppose $t\in \R$ and $A\subset \R$, then $\abs{t+A} = \abs{A}$.
%\end{prop}
%
%\begin{prop}{Countable Sub-additivity of outer measure}
%Suppose $A_1, A_2, \dots, \subset \R$. Then 
%$$\abs{\bigcup_{k=1}^\infty A_k} \leq \sum_{k=1}^\infty \abs{A_k}$$	
%Note that this implies finite sub-additivity which could come handy in proof techniques: 
%$$\abs{A_1\cup \dots \cup A_n} \leq \abs{A_1} + \dots + \abs{A_n}$$
%\end{prop}
%
%\subsection*{Outer Measure of Closed Bounded Interval}
%It is apparent for any closed interval $[a, b]$, we can construct a sequence of open cover $(a-\ep, b+\ep)$ and arbitrarily shrink $\ep$. We obtain $\abs{[a, b]} \leq b-a$. However, the other direction requires completeness of $\R$. \\
%
%\begin{prop}{$\abs{[a, b]} = b-a$}\\
%Suppose $a, b\in \R, a<b$. Then $\abs{[a, b]} = b-a$. 
%Proof: \ref{prof:closed interval}
%\end{prop}
%
%\begin{prop}{Non-trivial intervals are uncountable}\\
%	Every \textbf{interval} in $\R$ that contains at least two distinct terms is uncountable. 
%\end{prop}
%
%\begin{prop}{Non-additivity}\\
%	$\exists A, B\subset \R$ disjoint such that $\abs{A\cup B} \neq \abs{A} + \abs{B}$. 
%\end{prop}
\end{document}