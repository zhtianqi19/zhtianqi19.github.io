\documentclass[12pt, letterpaper]{article}

\usepackage[utf8]{inputenc}
\usepackage[framemethod=TikZ]{mdframed}
\usepackage[hidelinks]{hyperref}
\usepackage{mathtools, amssymb, amsmath, cleveref, fancyhdr, geometry, tcolorbox, graphicx, float, subfigure, arydshln, url, setspace, framed, pifont, physics, ntheorem, cancel, mathrsfs}


%%% for coding %%%
\usepackage{listings}
\usepackage[ruled, vlined, linesnumbered]{algorithm2e}
\SetKwComment{Comment}{/* }{ */}
\newcommand\mycommfont[1]{\small\ttfamily\textcolor{mygreen}{#1}}
\SetCommentSty{mycommfont}

\geometry{letterpaper, left=2cm, right=2cm, bottom=2cm, top=2cm}

\pagestyle{fancy}
\fancyhead{}
\fancyhead[L]{\leftmark}
\fancyhead[R]{\rightmark}
\fancyfoot{}
\fancyfoot[C]{\thepage}
%\rfoot{\footnotesize  Tianqi Zhang}


%\renewcommand{\headrulewidth}{0pt}
\renewcommand{\footrulewidth}{0pt}

\hypersetup{
	colorlinks = true,
	bookmarks = true,
	bookmarksnumbered = true,
	pdfborder = 001,
	linkcolor = blue
}

\definecolor{emoryblue}{RGB}{1, 33, 105} 
\definecolor{lightblue}{RGB}{0, 125, 186}
\definecolor{mediumblue}{RGB}{ 0, 51, 160}
\definecolor{darkblue}{RGB}{12, 35, 64}
\definecolor{red}{RGB}{185, 58, 38}
\definecolor{green}{RGB}{72, 127, 132}
\definecolor{gray1}{RGB}{217, 217, 214}
\definecolor{gray5}{RGB}{177, 179, 179}
\definecolor{gray3}{RGB}{208, 208, 206}

\definecolor{grey}{rgb}{0.49,0.38,0.29}
\definecolor{mygreen}{rgb}{0,0.6,0}
\definecolor{grey}{rgb}{0.49,0.38,0.29}
\definecolor{mygreen}{rgb}{0,0.6,0}


%%% for coding %%%
\lstset{basicstyle = \ttfamily\small,commentstyle = \color{mygreen}\textit, deletekeywords = {...}, escapeinside = {\%*}{*)}, frame = single, framesep = 0.5em, keywordstyle = \bfseries\color{blue}, morekeywords = {*}, emph = {self}, emphstyle=\bfseries\color{red}, numbers = left, numbersep = 1.5em, numberstyle = \ttfamily\small\color{grey},  rulecolor = \color{black}, showstringspaces = false, stringstyle = \ttfamily\color{purple}, tabsize = 4, columns = flexible}


\newcounter{index}[subsection]
\setcounter{index}{0}
\newenvironment*{df}[1]{\noindent\textbf{Definition \thesubsection.\stepcounter{index}\theindex\ (#1).}}{\\}

%\newenvironment*{eg}[1]{\begin{framed}\\\noindent\textbf{Example \thesubsection.\stepcounter{index}\theindex\ #1}\\ }{\\\end{framed}}

\newenvironment*{eg}[1]{
    \refstepcounter{index} % Increment the example counter
    \begin{framed}
    \noindent\textbf{Example \thesubsection.\theindex\ #1}
}{
    \end{framed}
}
%\newenvironment*{thm}[1]{\begin{tcolorbox}{\textbf{Theorem \thesubsection.\stepcounter{index}\theindex\ {#1}}}\\}{\\\end{tcolorbox}}
%\newenvironment*{cor}[1]{\noindent\textbf{Corollary \thesubsection.\stepcounter{index}\theindex\ #1:}}{\\}
%\newenvironment*{lem}[1]{\noindent\textbf{Lemma \thesubsection.\stepcounter{index}\theindex\ #1:}}{\\}
%\newenvironment*{ax}[1]{\noindent\textbf{Axiom \thesubsection.\stepcounter{index}\theindex\ #1:}}{\\}
%\newenvironment*{prop}[1]{\noindent\textbf{Proposition \thesubsection.\stepcounter{index}\theindex\ #1:}}{\\}
%\newenvironment*{conj}[1]{\noindent\textbf{Conjecture \thesubsection.\stepcounter{index}\theindex\ #1:}}{\\}
%\newenvironment*{nota}{\noindent\textbf{Notation \thesubsection.\stepcounter{index}\theindex.}}{\\}
%\newenvironment*{clm}{\noindent\textbf{Claim \thesubsection.\stepcounter{index}\theindex}}{\\}

% Ensure proper grouping and formatting for compatibility with lists
\newenvironment*{thm}[1]{%
  \begin{tcolorbox}%
  \textbf{Theorem \thesubsection.\stepcounter{index}\theindex\ {#1}}%
  \par\noindent%
}{%
  \end{tcolorbox}%
}

\newenvironment*{cor}[1]{%
  \par\noindent\textbf{Corollary \thesubsection.\stepcounter{index}\theindex\ {#1}:}%
  \par\noindent%
}{%
  \par%
}

\newenvironment*{lem}[1]{%
  \par\noindent\textbf{Lemma \thesubsection.\stepcounter{index}\theindex\ {#1}:}%
  \par\noindent%
}{%
  \par%
}

\newenvironment*{ax}[1]{%
  \par\noindent\textbf{Axiom \thesubsection.\stepcounter{index}\theindex\ {#1}:}%
  \par\noindent%
}{%
  \par%
}

\newenvironment*{prop}[1]{%
  \par\noindent\textbf{Proposition \thesubsection.\stepcounter{index}\theindex\ {#1}:}%
  \par\noindent%
}{%
  \par%
}

\newenvironment*{conj}[1]{%
  \par\noindent\textbf{Conjecture \thesubsection.\stepcounter{index}\theindex\ {#1}:}%
  \par\noindent%
}{%
  \par%
}

\newenvironment*{nota}{%
  \par\noindent\textbf{Notation \thesubsection.\stepcounter{index}\theindex:}%
  \par\noindent%
}{%
  \par%
}

\newenvironment*{clm}{%
  \par\noindent\textbf{Claim \thesubsection.\stepcounter{index}\theindex:}%
  \par\noindent%
}{%
  \par%
}


\newcounter{nprf}[subsection]
\setcounter{nprf}{0}
\newenvironment*{prf}{\noindent\textbf{\textit{Proof \stepcounter{nprf}\thenprf.}}}{\hfill$\blacksquare$\\}
\newenvironment*{dis}{\indent\textbf{\textit{Disproof \stepcounter{nprf}\thenprf.}}}{\hfill$\blacksquare$\\}
\newenvironment*{sol}{\indent\textbf{\textit{Solution \stepcounter{nprf}\thenprf.}}\\}{\hfill{$\square$}\\}

\newenvironment*{prf*}{\noindent\textit{Proof.}\ }{$\qquad\square$\\}
\newenvironment*{dis*}{\indent\textit{Disproof.}\ }{$\qquad\square$\\}
\newenvironment*{sol*}{\indent\textit{Solution.}\ }{$\qquad\square$\\}

\newtheorem{hint}{Hint}[section]
\newtheorem{rmk}{Remark}[section]
\newtheorem{ext}{Extension}[section]

\newtheorem*{df*}{Definition}
\newtheorem*{thm*}{Theorem}
\newtheorem*{clm*}{Claim}
\newtheorem*{cor*}{Corollary}
\newtheorem*{lem*}{Lemma}
\newtheorem*{ax*}{Axiom}
\newtheorem*{prop*}{Proposition}
\newtheorem*{conj*}{Conjecture}
\newtheorem*{nota*}{Notation}

\linespread{1.25}

\newcommand{\inprod}[2]{\left\langle #1, #2 \right\rangle}

\def\Z{{\mathbb{Z}}}
\def\H{{\mathcal{H}}}
\def\M{{\mathcal{M}}}
\def\R{{\mathbb{R}}}
\def\C{{\mathbb{C}}}
\def\Q{{\mathbb{Q}}}
\def\d{{\mathrm{d}}}
\def\i{{\mathrm{i}}}
\def\ep{{\varepsilon}}
\def\N{\mathbb{N}}
\def\1{\mathds{1}}
\def\bigO{\mathcal{O}}
\def\sp{\operatorname{span}}
\def\epsilon{\varepsilon}
\def\emptyset{\varnothing}
\def\phi{\varphi}
\def\dsst{\displaystyle}
\def\st{\ s.t.\ }
\def\wrt{\ w.r.t.\ }
\def\bar{\overline}
\def\tilde{\widetilde}
\def\E{\vb{E}}
\def\B{\vb{B}}
\def\L{\vb{L}}
\def\I{\vb{I}}
\def\Var{\vb{Var}}
\def\V{\vb{Var}}
\def\Cov{\vb{Cov}}
\def\MSE{\vb{MSE}}
\def\P{\vb{P}}
\def\M{\vb{M}}
\def\iid{i.i.d.}
\def\argmax{\arg\max}
\def\argmin{\arg\min}
\def\l{\ell}
\def\hat{\widehat}
\def\independ{\perp\!\!\!\perp}
\def\depend{\leftrightsquigarrow}
\def\residual{\varepsilon}
\def\sd{\mathrm{sd}}
\def\LI{\mathrm{L.I.}}
\def\range{\operatorname{range}}
\def\Null{\operatorname{null}}
\def\nullity{\operatorname{nullity}}
\def\A{A^{-1}}
\def\alg{\operatorname{alg}}
\def\fl{\operatorname{fl}}
\def\algmult{\operatorname{alg. mult.}}
\def\geomult{\operatorname{geo. mult.}}
\def\diag{\operatorname{diag}}
\def\gap{\operatorname{gap}}
\def\pqde{\quad\square}
\def\lub{\operatorname{lub}}
\def\Int{\operatorname{int}}
\def\ac{\operatorname{ac}}
\def\cl{\operatorname{cl}}
\def\bd{\operatorname{bd}}
\DeclareMathOperator*{\plim}{plim}
\def\upint{\mathchoice%
    {\mkern13mu\overline{\vphantom{\intop}\mkern7mu}\mkern-20mu}%
    {\mkern7mu\overline{\vphantom{\intop}\mkern7mu}\mkern-14mu}%
    {\mkern7mu\overline{\vphantom{\intop}\mkern7mu}\mkern-14mu}%
    {\mkern7mu\overline{\vphantom{\intop}\mkern7mu}\mkern-14mu}%
  \int}
\def\lowint{\mkern3mu\underline{\vphantom{\intop}\mkern7mu}\mkern-10mu\int}




\title{\textbf{% ECON 620\\
               Probability and Statistical Inference}}
\author{Tianqi Zhang\\
Emory University}
\date{Apr 17th 2025}

\begin{document}
\maketitle
\setcounter{section}{2}
\section{Discrete Probability Measures}
\subsection{Discrete Probability Measures}
We start with discrete, countable latent space $\Omega$. \\

\begin{df}{Discrete Probability Measure}
A discrete probability measure on sample space \(\Omega\), finite or countable, is a sequence of \(\qty{p_\omega}_{\omega \in \Omega}\) of non-negative real numbers such that:
\begin{enumerate}
	\item \(p_\omega \geq 0\), \(\forall \omega \in \Omega\),
	\item \(\sum_{\omega \in \Omega} p_\omega = 1\)
\end{enumerate}
A general definition that works not only for finite $\Omega$ but also for countable $\Omega$ since it allows in both cases to compute for any random event $A\subseteq \Omega$. 
$$P(A) = \sum_{\omega\in A}P_\omega$$
\end{df}

\begin{df}{Measure/Distribution}\\
A measure \(P\) on \(\Omega\) is a mapping from the power set of the latent space $\Omega$. 
$$P: \mathscr{P}(\Omega) \to [0, \infty]$$
such that the following two axioms are satisfied:
\begin{enumerate}
	\item \textbf{Non-negativity:} $\forall A \subseteq \Omega, P(A)\geq 0$
	\item \textbf{Countable additivity:} (Or sigma additivity) For disjoint $A_n \subset \Omega$, 
	$$P\qty(\bigcup_{n\in\N} A_n) = \sum_k P(A_n)$$
	\item \textbf{Empty set measurability: } $P(\emptyset) = 0$. 
\end{enumerate}
 Note that for $P$ to be a probability measure: $P(A) \in [0, 1] \ \forall A\in \mathscr{P}(\Omega)$, with $P(\Omega) = 1$.
\end{df}

\begin{thm}{Komolgrov Axioms}
	$P:\mathscr{P}(\Omega) \rightarrow [0, 1]$ is a probability measure if the above conditions are true. 
\end{thm}

\begin{rmk}{inclusion of infinity:}
In most measure-theoretic contexts, it is permissible for certain subsets  $\mathcal{A}\subset \Omega$ to have infinite measure. Consequently, the codomain of a measure is typically extended to include infinity. This is commonly represented as the set of positive real numbers together with infinity, denoted by $[0, \infty)\cup\qty{\infty}$. For simplicity, this notation is often abbreviated as $[0, \infty]$. 
\end{rmk}


\begin{eg}{Common Measures}
\begin{itemize}
    \item \textbf{Counting Measure:} \(\mu(A) = |A|\).
    \item \textbf{Dirac Measure at \(p \in \Omega\):}
    \[
    \delta_p(A) = 
    \begin{cases} 
    1, & \text{if } p \in A, \\
    0, & \text{otherwise}.
    \end{cases}
    \]
    \item \textbf{Lebesgue Measure on $\R$:} For simple intervals $[a, b)\subset \R$ with $a\leq b$. Lebesgue measure $\mu$ is defined to be $b-a$. 
\end{itemize}
\end{eg}

\begin{prop}{Properties of Komolgrov axioms}
\begin{enumerate}
	\item $P(\overline A) = 1-P(A)$ where $\overline A \equiv \Omega\backslash A$.
	\item Define $A+B$ as the disjoint union of $A, B$, i.e., $A\cap B = \varnothing$, then $P(A+B) = P(A) + P(B)$. 
	\item If $B\subset A$, define $A-B \equiv A\cap \overline B$  then $P(A-B) = P(A) - P(B)$. 
	\item Partition $\Omega$ into disjoint $\qty{H_i}_{i\in \N}$, i.e. $H_i\cap H_j = \varnothing \ \forall i, j$ and $\bigcup_i H_i = \Omega$. Then $\forall A\subset \Omega, P(A) = \sum_i P(A\cap H_i)$. 
\end{enumerate}
\end{prop}

\begin{cor}{Monotonicity}
If $A\subseteq B$, then $0\leq P(A) \leq P(B)\leq 1$. 	
\end{cor}\


\begin{cor}{Sylvester Formula}
For any collection of subsets $\qty{A_i}$ of $\Omega$
$$P\qty(\bigcup_{i=1}^n) = \sum_{i=1}^nP(A_i) - \sum_{i<j}P(A_i\cap A_j) + \sum_{i<j<k}P(A_i\cap A_j\cap A_k) + \dots + (-1)^{n-1}P\qty(\bigcap_{i=1}^nA_i)$$
Or a more intuitive version, for any $A, B\subseteq \Omega$, 
$$P(A\cup B) = P(A)+P(B)-P(A\cap B)$$
\end{cor}

\begin{rmk}{Banach-Tarski Paradox:}
If \(\Omega\) is countable, we can assign a finite measure to all subsets of \(\Omega\) satisfying the Kolmogorov axioms. However, if \(\Omega\) is uncountable, these axioms can lead to paradoxes.
\end{rmk}

\noindent To address these issues, we may need to restrict our measure $P: \mathscr{P}(\Omega)\rightarrow [0, \infty]$ to a carefully chosen collection of subsets $\mathcal{F}\subset \Omega$. This restriction sets the foundation for further discussion on the role of $\sigma$-algebras in measure theory.

\subsection{Results and Properties}

\begin{prop}{Boole's Inequality}
For \(A_i \subseteq \Omega\), not necessarily disjoint:
\[
P\left(\bigcup_i A_i\right) \leq \sum_i P(A_i).
\]
Proof: \ref{proof: boole}
\end{prop}

\begin{prop}{Bonferroni's Inequality}
For \(A_i \subseteq \Omega\):
\[
P\left(\bigcap_i A_i\right) \geq \sum_i P(A_i) - (n-1).
\]
This result is useful for multiple hypothesis testing.\\
Proof: \ref{proof: bonferroni}
\end{prop}

\begin{prop}{De Morgan's Laws}
\[
P\left(\bigcap_i A_i\right) = P\left(\bigcup_i \overline{A_i}\right).
\]
\end{prop}

\noindent Some notations for limits of sets:  
\begin{itemize}
	\item \textbf{Increasing Sequence: }\((A_n)\) is called an increasing sequence if (\(A_n \subseteq A_{n+1}\)), and 
	$$\lim_{n\to \infty} \uparrow A_n \equiv \bigcup_{k=1}^\infty A_n$$
	\item \textbf{Decreasing Sequence: }\((B_n)\) is called an increasing sequence if (\(B_n \supseteq B_{n+1}\)), and 
	$$\lim_{n\to \infty} \downarrow B_n \equiv \bigcap_{k=1}^\infty B_n$$
\end{itemize}

\begin{prop}{Continuity of Measures}\\
The continuity of measures is preserved under increasing and decreasing set limits. \\

\noindent Let \((A_n)\) be an increasing sequence of sets:\\
\textbf{Increasing continuity:}
    \[
    \lim_{n \to \infty} P(A_n) = P\qty(\lim_{n \to \infty} \uparrow A_n) \equiv P\left(\bigcup_{n=1}^\infty A_n\right).
    \]
 Let \((B_n)\) be a decreasing sequence of sets:\\
 \textbf{Decreasing Sequence:}
    \[
    \lim_{n \to \infty} P(B_n) = P\qty(\lim_{n \to \infty} \downarrow B_n) \equiv P\left(\bigcap_{n=1}^\infty B_n\right).
    \]
\end{prop}
\end{document}