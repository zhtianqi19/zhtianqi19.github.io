\usepackage[utf8]{inputenc}
\usepackage[framemethod=TikZ]{mdframed}
\usepackage[hidelinks]{hyperref}
\usepackage{mathtools, amssymb, amsmath, cleveref, fancyhdr, geometry, tcolorbox, graphicx, float, subfigure, arydshln, url, setspace, framed, pifont, physics, ntheorem, cancel}
%%% for coding %%%
\usepackage{listings}
\usepackage[ruled, vlined, linesnumbered]{algorithm2e}
\SetKwComment{Comment}{/* }{ */}
\newcommand\mycommfont[1]{\small\ttfamily\textcolor{mygreen}{#1}}
\SetCommentSty{mycommfont}

\geometry{letterpaper, left=2cm, right=2cm, bottom=2cm, top=2cm}

\pagestyle{fancy}
\fancyhead{}
\fancyhead[L]{\leftmark}
\fancyhead[R]{\rightmark}
\fancyfoot{}
\fancyfoot[C]{\thepage}
%\renewcommand{\headrulewidth}{0pt}
\renewcommand{\footrulewidth}{0pt}

\hypersetup{
	colorlinks = true,
	bookmarks = true,
	bookmarksnumbered = true,
	pdfborder = 001,
	linkcolor = blue
}

\definecolor{emoryblue}{RGB}{1, 33, 105} 
\definecolor{lightblue}{RGB}{0, 125, 186}
\definecolor{mediumblue}{RGB}{ 0, 51, 160}
\definecolor{darkblue}{RGB}{12, 35, 64}
\definecolor{red}{RGB}{185, 58, 38}
\definecolor{green}{RGB}{72, 127, 132}
\definecolor{gray1}{RGB}{217, 217, 214}
\definecolor{gray5}{RGB}{177, 179, 179}
\definecolor{gray3}{RGB}{208, 208, 206}

\definecolor{grey}{rgb}{0.49,0.38,0.29}
\definecolor{mygreen}{rgb}{0,0.6,0}

%%% for coding %%%
\lstset{basicstyle = \ttfamily\small,commentstyle = \color{mygreen}\textit, deletekeywords = {...}, escapeinside = {\%*}{*)}, frame = single, framesep = 0.5em, keywordstyle = \bfseries\color{blue}, morekeywords = {*}, emph = {self}, emphstyle=\bfseries\color{red}, numbers = left, numbersep = 1.5em, numberstyle = \ttfamily\small\color{grey},  rulecolor = \color{black}, showstringspaces = false, stringstyle = \ttfamily\color{purple}, tabsize = 4, columns = flexible}

\newcounter{index}[section]
\renewcommand{\theindex}{\thesection.\arabic{index}}

\newenvironment*{df}[1]{%
  \stepcounter{index}%
  \noindent\textbf{Definition \theindex\ (#1).}%
}{\\}

\newenvironment*{eg}[1]{%
  \stepcounter{index}%
  \begin{framed}\noindent\textbf{Example \theindex\ #1}\\ 
}{%
  \end{framed}
}

\newenvironment*{thm}[1]{%
  \stepcounter{index}%
  \begin{tcolorbox}[colback=gray!5, colframe=gray!40!black,
    title=Theorem \theindex\ (#1)]%
}{%
  \end{tcolorbox}
}

\newenvironment*{cor}[1]{\stepcounter{index}\noindent\textbf{Corollary \theindex\ (#1):}}{\\}
\newenvironment*{lem}[1]{\stepcounter{index}\noindent\textbf{Lemma \theindex\ (#1):}}{\\}
\newenvironment*{ax}[1]{\stepcounter{index}\noindent\textbf{Axiom \theindex\ (#1):}}{\\}
\newenvironment*{prop}[1]{\stepcounter{index}\noindent\textbf{Proposition \theindex\ (#1):}}{\\}
\newenvironment*{conj}[1]{\stepcounter{index}\noindent\textbf{Conjecture \theindex\ (#1):}}{\\}
\newenvironment*{nota}{\stepcounter{index}\noindent\textbf{Notation \theindex.}}{\\}
\newenvironment*{clm}{\stepcounter{index}\noindent\textbf{Claim \theindex}}{\\}

\newcounter{nprf}[section]
\setcounter{nprf}{0}
\newenvironment*{prf}{\noindent\textbf{\textit{Proof \stepcounter{nprf}\thenprf.}}}{\hfill$\blacksquare$\\}
\newenvironment*{dis}{\indent\textbf{\textit{Disproof \stepcounter{nprf}\thenprf.}}}{\hfill$\blacksquare$\\}
\newenvironment*{sol}{\indent\textbf{\textit{Solution \stepcounter{nprf}\thenprf.}}\\}{\hfill{$\square$}\\}

\newenvironment*{prf*}{\noindent\textit{Proof.}\ }{$\qquad\square$\\}
\newenvironment*{dis*}{\indent\textit{Disproof.}\ }{$\qquad\square$\\}
\newenvironment*{sol*}{\indent\textit{Solution.}\ }{$\qquad\square$\\}

\newtheorem{hint}{Hint}[section]
\newtheorem{rmk}{Remark}[section]
\newtheorem{ext}{Extension}[section]

\newtheorem*{df*}{Definition}
\newtheorem*{thm*}{Theorem}
\newtheorem*{clm*}{Claim}
\newtheorem*{cor*}{Corollary}
\newtheorem*{lem*}{Lemma}
\newtheorem*{ax*}{Axiom}
\newtheorem*{prop*}{Proposition}
\newtheorem*{conj*}{Conjecture}
\newtheorem*{nota*}{Notation}

\linespread{1.25}

\newcommand{\inprod}[2]{\left\langle #1, #2 \right\rangle}

\def\Z{{\mathbb{Z}}}
\def\R{{\mathbb{R}}}
\def\C{{\mathbb{C}}}
\def\Q{{\mathbb{Q}}}
\def\d{{\mathrm{d}}}
\def\i{{\mathrm{i}}}
\def\ep{{\varepsilon}}
\def\N{\mathbb{N}}
\def\1{\mathds{1}}
\def\bigO{\mathcal{O}}
\def\sp{\operatorname{span}}
\def\epsilon{\varepsilon}
\def\emptyset{\varnothing}
\def\phi{\varphi}
\def\dsst{\displaystyle}
\def\st{\ s.t.\ }
\def\wrt{\ w.r.t.\ }
\def\bar{\overline}
\def\tilde{\widetilde}
\def\E{\mathbb{E}}
\def\B{\vb{B}}
\def\L{\vb{L}}
\def\I{\vb{I}}
\def\Var{\vb{Var}}
\def\V{\vb{Var}}
\def\Cov{\vb{Cov}}
\def\MSE{\vb{MSE}}
\def\P{\vb{P}}
\def\M{\vb{M}}
\def\iid{i.i.d.}
\def\argmax{\arg\max}
\def\argmin{\arg\min}
\def\l{\ell}
\def\hat{\widehat}
\def\independ{\perp\!\!\!\perp}
\def\depend{\leftrightsquigarrow}
\def\residual{\varepsilon}
\def\sd{\mathrm{sd}}
\def\LI{\mathrm{L.I.}}
\def\range{\operatorname{range}}
\def\Null{\operatorname{null}}
\def\nullity{\operatorname{nullity}}
\def\A{A^{-1}}
\def\alg{\operatorname{alg}}
\def\fl{\operatorname{fl}}
\def\algmult{\operatorname{alg. mult.}}
\def\geomult{\operatorname{geo. mult.}}
\def\diag{\operatorname{diag}}
\def\gap{\operatorname{gap}}
\def\pqde{\quad\square}
\def\lub{\operatorname{lub}}
\def\Int{\operatorname{int}}
\def\ac{\operatorname{ac}}
\def\cl{\operatorname{cl}}
\def\bd{\operatorname{bd}}
\def\upint{\mathchoice%
    {\mkern13mu\overline{\vphantom{\intop}\mkern7mu}\mkern-20mu}%
    {\mkern7mu\overline{\vphantom{\intop}\mkern7mu}\mkern-14mu}%
    {\mkern7mu\overline{\vphantom{\intop}\mkern7mu}\mkern-14mu}%
    {\mkern7mu\overline{\vphantom{\intop}\mkern7mu}\mkern-14mu}%
  \int}
\def\lowint{\mkern3mu\underline{\vphantom{\intop}\mkern7mu}\mkern-10mu\int}


\title{\textbf{% ECON 620\\
               Probability and Statistical Inference}}
\author{Tianqi Zhang\\
Emory University}
\date{Apr 17th 2025}

\begin{document}
\maketitle
\setcounter{section}{2}
\section{Discrete Probability Measures}
\subsection{Discrete Probability Measures}
We start with discrete, countable latent space $\Omega$. \\

\begin{df}{Discrete Probability Measure}
A discrete probability measure on sample space \(\Omega\), finite or countable, is a sequence of \(\qty{p_\omega}_{\omega \in \Omega}\) of non-negative real numbers such that:
\begin{enumerate}
	\item \(p_\omega \geq 0\), \(\forall \omega \in \Omega\),
	\item \(\sum_{\omega \in \Omega} p_\omega = 1\)
\end{enumerate}
A general definition that works not only for finite $\Omega$ but also for countable $\Omega$ since it allows in both cases to compute for any random event $A\subseteq \Omega$. 
$$P(A) = \sum_{\omega\in A}P_\omega$$
\end{df}
such that it satisfies the \textbf{Komolgrov axioms:}

\begin{thm}{Kolmogorov Axioms for Discrete Probability}
Let \( \Omega \) be a finite or countable set. A function \( P: \mathscr{P}(\Omega) \to [0,1] \) is a probability measure if:
\begin{enumerate}
    \item \textbf{Non-negativity:} \( P(A) \geq 0 \) for all \( A \subseteq \Omega \),
    \item \textbf{Normalization:} \( P(\Omega) = 1 \),
    \item \textbf{Countable additivity:} For any sequence of disjoint events $A_1, A_2, \dots \subseteq \Omega$:
$$P\left(\bigcup_{i=1}^{\infty} A_i\right) = \sum_{i=1}^{\infty} P(A_i)$$
\end{enumerate}
\end{thm}



Let’s consider some common examples of discrete probability measures.

\begin{eg}{Common Measures}
\begin{itemize}
    \item \textbf{Counting Measure:} \(\mu(A) = |A|\).
    Define $\Omega = \mathbb{N}$ and let
	\[
	P(A) = \text{number of elements in } A.
	\]
	This is called the \textbf{counting measure}. It’s not a probability measure since the total mass is infinite, but it satisfies many of the same properties and helps build intuition.
    \item \textbf{Dirac Measure at \(p \in \Omega\):}
    \[
    \delta_p(A) = 
    \begin{cases} 
    1, & \text{if } p \in A, \\
    0, & \text{otherwise}.
    \end{cases}
    \]
    This is called the \textbf{Dirac measure} at $p$. It concentrates all probability mass at a single point.
    
	\item \textbf{Uniform Distribution on a Finite Set}
	Let $\Omega = \{1, 2, \dots, n\}$ and define
	\[
	P(\omega) = \frac{1}{n} \quad \text{for all } \omega \in \Omega.
	\]
	This assigns equal weight to each element — a classic finite uniform distribution.
\end{itemize}
\end{eg}

\subsection{Basic Properties of Discrete Probability}

Let $P$ be a discrete probability measure on $\Omega$ and let $A, B \subseteq \Omega$.

\begin{prop}{Basic Rules}
\begin{enumerate}
    \item \textbf{Empty and full space:} \( P(\emptyset) = 0 \), \quad \( P(\Omega) = 1 \)
    
    \item \textbf{Complement rule:} \( P(A^c) = 1 - P(A) \)
    \item \textbf{Difference rule:} If \( B \subseteq A \), then
    \[
    P(A \setminus B) = P(A) - P(B)
    \]
\end{enumerate}
\end{prop}

\begin{cor}{Monotonicity} If $A \subseteq B$, then $P(A) \leq P(B)$
\end{cor}

\begin{cor}{Partition Formula}
Suppose $\Omega$ is partitioned into disjoint subsets $\{H_i\}$ such that $\bigcup_i H_i = \Omega$. Then for any $A \subseteq \Omega$,
\[
P(A) = \sum_i P(A \cap H_i).
\]
\end{cor}


\begin{cor}{Sylvester Formula (Inclusion-Exclusion)}
For any collection of subsets $\qty{A_i}$ of $\Omega$
$$P\qty(\bigcup_{i=1}^n A_i) = \sum_{i=1}^nP(A_i) - \sum_{i<j}P(A_i\cap A_j) + \sum_{i<j<k}P(A_i\cap A_j\cap A_k) + \dots + (-1)^{n-1}P\qty(\bigcap_{i=1}^nA_i)$$
Or a more intuitive version, for any $A, B\subseteq \Omega$, 
$$P(A\cup B) = P(A)+P(B)-P(A\cap B)$$
\end{cor}


\subsection{Further Properties of Discrete Probability}
Probability satisfies a number of important inequalities, logical identities, and limit properties. These results often appear in theoretical proofs and practical applications.

\subsubsection{Inequalities and Bounds}

\begin{prop}{Boole's Inequality}
For any (finite or countable) collection of events \( A_i \subseteq \Omega \), not necessarily disjoint:
\[
P\left(\bigcup_i A_i\right) \leq \sum_i P(A_i).
\]
\end{prop}

\begin{rmk}
This inequality reflects that when events overlap, simply summing their individual probabilities can overestimate the probability of their union. In general measure theory, Boole’s inequality expresses the principle of \textbf{countable subadditivity}.
\end{rmk}


\begin{prop}{Bonferroni's Inequality}
Let \( A_1, A_2, \dots, A_n \subseteq \Omega \). Then:
\[
P\left(\bigcap_{i=1}^n A_i\right) \geq \sum_{i=1}^n P(A_i) - (n - 1).
\]
\end{prop}
\begin{rmk}
Bonferroni’s inequality provides a lower bound on the probability that *all* events occur. It’s commonly used in multiple hypothesis testing.
\end{rmk}



\subsubsection{Set-Theoretic Identities}
\begin{prop}{De Morgan's Law}
For any collection \( \{A_i\} \),
\[
\left( \bigcap_i A_i \right)^C = \bigcup_i A_i^C.
\]
Taking probabilities on both sides:
\[
P\left( \bigcap_i A_i \right) = 1 - P\left( \bigcup_i A_i^C \right).
\]
where \( A^C \equiv \Omega \setminus A \) denotes the complement of \( A \).
\end{prop}

%\begin{prop}{Continuity of Probability}
%Let \( P \) be a discrete probability measure.
%
%\begin{enumerate}
%    \item \textbf{Increasing continuity:} If \( A_n \uparrow A \), then
%    \[
%    \lim_{n \to \infty} P(A_n) = P\left( \bigcup_{n=1}^\infty A_n \right).
%    \]
%    
%    \item \textbf{Decreasing continuity:} If \( B_n \downarrow B \), then
%    \[
%    \lim_{n \to \infty} P(B_n) = P\left( \bigcap_{n=1}^\infty B_n \right).
%    \]
%\end{enumerate}
%\end{prop}


%
%
%\subsection{Results and Properties}
%
%\begin{prop}{Boole's Inequality}
%For \(A_i \subseteq \Omega\), not necessarily disjoint:
%\[
%P\left(\bigcup_i A_i\right) \leq \sum_i P(A_i).
%\]
%\end{prop}
%
%\begin{prop}{Bonferroni's Inequality}
%For \(A_i \subseteq \Omega\):
%\[
%P\left(\bigcap_i A_i\right) \geq \sum_i P(A_i) - (n-1).
%\]
%This result is useful for multiple hypothesis testing.
%\end{prop}
%
%\begin{prop}{De Morgan's Laws}
%\[
%P\left(\bigcap_i A_i\right) = P\left(\bigcup_i \overline{A_i}\right).
%\]
%\end{prop}
%
%\noindent Some notations for limits of sets:  
%\begin{itemize}
%	\item \textbf{Increasing Sequence: }\((A_n)\) is called an increasing sequence if (\(A_n \subseteq A_{n+1}\)), and 
%	$$\lim_{n\to \infty} \uparrow A_n \equiv \bigcup_{k=1}^\infty A_n$$
%	\item \textbf{Decreasing Sequence: }\((B_n)\) is called an increasing sequence if (\(B_n \supseteq B_{n+1}\)), and 
%	$$\lim_{n\to \infty} \downarrow B_n \equiv \bigcap_{k=1}^\infty B_n$$
%\end{itemize}
%
%\begin{prop}{Continuity of Measures}\\
%The continuity of measures is preserved under increasing and decreasing set limits. \\
%
%\noindent Let \((A_n)\) be an increasing sequence of sets:\\
%\textbf{Increasing continuity:}
%    \[
%    \lim_{n \to \infty} P(A_n) = P\qty(\lim_{n \to \infty} \uparrow A_n) \equiv P\left(\bigcup_{n=1}^\infty A_n\right).
%    \]
% Let \((B_n)\) be a decreasing sequence of sets:\\
% \textbf{Decreasing Sequence:}
%    \[
%    \lim_{n \to \infty} P(B_n) = P\qty(\lim_{n \to \infty} \downarrow B_n) \equiv P\left(\bigcap_{n=1}^\infty B_n\right).
%    \]
%\end{prop}
\end{document}