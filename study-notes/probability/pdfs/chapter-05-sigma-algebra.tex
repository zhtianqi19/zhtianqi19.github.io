\documentclass[12pt, letterpaper]{article}

\usepackage[utf8]{inputenc}
\usepackage[framemethod=TikZ]{mdframed}
\usepackage[hidelinks]{hyperref}
\usepackage{mathtools, amssymb, amsmath, cleveref, fancyhdr, geometry, tcolorbox, graphicx, float, subfigure, arydshln, url, setspace, framed, pifont, physics, ntheorem, cancel, mathrsfs}


%%% for coding %%%
\usepackage{listings}
\usepackage[ruled, vlined, linesnumbered]{algorithm2e}
\SetKwComment{Comment}{/* }{ */}
\newcommand\mycommfont[1]{\small\ttfamily\textcolor{mygreen}{#1}}
\SetCommentSty{mycommfont}

\geometry{letterpaper, left=2cm, right=2cm, bottom=2cm, top=2cm}

\pagestyle{fancy}
\fancyhead{}
\fancyhead[L]{\leftmark}
\fancyhead[R]{\rightmark}
\fancyfoot{}
\fancyfoot[C]{\thepage}
%\rfoot{\footnotesize  Tianqi Zhang}


%\renewcommand{\headrulewidth}{0pt}
\renewcommand{\footrulewidth}{0pt}

\hypersetup{
	colorlinks = true,
	bookmarks = true,
	bookmarksnumbered = true,
	pdfborder = 001,
	linkcolor = blue
}

\definecolor{emoryblue}{RGB}{1, 33, 105} 
\definecolor{lightblue}{RGB}{0, 125, 186}
\definecolor{mediumblue}{RGB}{ 0, 51, 160}
\definecolor{darkblue}{RGB}{12, 35, 64}
\definecolor{red}{RGB}{185, 58, 38}
\definecolor{green}{RGB}{72, 127, 132}
\definecolor{gray1}{RGB}{217, 217, 214}
\definecolor{gray5}{RGB}{177, 179, 179}
\definecolor{gray3}{RGB}{208, 208, 206}

\definecolor{grey}{rgb}{0.49,0.38,0.29}
\definecolor{mygreen}{rgb}{0,0.6,0}
\definecolor{grey}{rgb}{0.49,0.38,0.29}
\definecolor{mygreen}{rgb}{0,0.6,0}


%%% for coding %%%
\lstset{basicstyle = \ttfamily\small,commentstyle = \color{mygreen}\textit, deletekeywords = {...}, escapeinside = {\%*}{*)}, frame = single, framesep = 0.5em, keywordstyle = \bfseries\color{blue}, morekeywords = {*}, emph = {self}, emphstyle=\bfseries\color{red}, numbers = left, numbersep = 1.5em, numberstyle = \ttfamily\small\color{grey},  rulecolor = \color{black}, showstringspaces = false, stringstyle = \ttfamily\color{purple}, tabsize = 4, columns = flexible}


\newcounter{index}[subsection]
\setcounter{index}{0}
\newenvironment*{df}[1]{\noindent\textbf{Definition \thesubsection.\stepcounter{index}\theindex\ (#1).}}{\\}

%\newenvironment*{eg}[1]{\begin{framed}\\\noindent\textbf{Example \thesubsection.\stepcounter{index}\theindex\ #1}\\ }{\\\end{framed}}

\newenvironment*{eg}[1]{
    \refstepcounter{index} % Increment the example counter
    \begin{framed}
    \noindent\textbf{Example \thesubsection.\theindex\ #1}
}{
    \end{framed}
}
%\newenvironment*{thm}[1]{\begin{tcolorbox}{\textbf{Theorem \thesubsection.\stepcounter{index}\theindex\ {#1}}}\\}{\\\end{tcolorbox}}
%\newenvironment*{cor}[1]{\noindent\textbf{Corollary \thesubsection.\stepcounter{index}\theindex\ #1:}}{\\}
%\newenvironment*{lem}[1]{\noindent\textbf{Lemma \thesubsection.\stepcounter{index}\theindex\ #1:}}{\\}
%\newenvironment*{ax}[1]{\noindent\textbf{Axiom \thesubsection.\stepcounter{index}\theindex\ #1:}}{\\}
%\newenvironment*{prop}[1]{\noindent\textbf{Proposition \thesubsection.\stepcounter{index}\theindex\ #1:}}{\\}
%\newenvironment*{conj}[1]{\noindent\textbf{Conjecture \thesubsection.\stepcounter{index}\theindex\ #1:}}{\\}
%\newenvironment*{nota}{\noindent\textbf{Notation \thesubsection.\stepcounter{index}\theindex.}}{\\}
%\newenvironment*{clm}{\noindent\textbf{Claim \thesubsection.\stepcounter{index}\theindex}}{\\}

% Ensure proper grouping and formatting for compatibility with lists
\newenvironment*{thm}[1]{%
  \begin{tcolorbox}%
  \textbf{Theorem \thesubsection.\stepcounter{index}\theindex\ {#1}}%
  \par\noindent%
}{%
  \end{tcolorbox}%
}

\newenvironment*{cor}[1]{%
  \par\noindent\textbf{Corollary \thesubsection.\stepcounter{index}\theindex\ {#1}:}%
  \par\noindent%
}{%
  \par%
}

\newenvironment*{lem}[1]{%
  \par\noindent\textbf{Lemma \thesubsection.\stepcounter{index}\theindex\ {#1}:}%
  \par\noindent%
}{%
  \par%
}

\newenvironment*{ax}[1]{%
  \par\noindent\textbf{Axiom \thesubsection.\stepcounter{index}\theindex\ {#1}:}%
  \par\noindent%
}{%
  \par%
}

\newenvironment*{prop}[1]{%
  \par\noindent\textbf{Proposition \thesubsection.\stepcounter{index}\theindex\ {#1}:}%
  \par\noindent%
}{%
  \par%
}

\newenvironment*{conj}[1]{%
  \par\noindent\textbf{Conjecture \thesubsection.\stepcounter{index}\theindex\ {#1}:}%
  \par\noindent%
}{%
  \par%
}

\newenvironment*{nota}{%
  \par\noindent\textbf{Notation \thesubsection.\stepcounter{index}\theindex:}%
  \par\noindent%
}{%
  \par%
}

\newenvironment*{clm}{%
  \par\noindent\textbf{Claim \thesubsection.\stepcounter{index}\theindex:}%
  \par\noindent%
}{%
  \par%
}


\newcounter{nprf}[subsection]
\setcounter{nprf}{0}
\newenvironment*{prf}{\noindent\textbf{\textit{Proof \stepcounter{nprf}\thenprf.}}}{\hfill$\blacksquare$\\}
\newenvironment*{dis}{\indent\textbf{\textit{Disproof \stepcounter{nprf}\thenprf.}}}{\hfill$\blacksquare$\\}
\newenvironment*{sol}{\indent\textbf{\textit{Solution \stepcounter{nprf}\thenprf.}}\\}{\hfill{$\square$}\\}

\newenvironment*{prf*}{\noindent\textit{Proof.}\ }{$\qquad\square$\\}
\newenvironment*{dis*}{\indent\textit{Disproof.}\ }{$\qquad\square$\\}
\newenvironment*{sol*}{\indent\textit{Solution.}\ }{$\qquad\square$\\}

\newtheorem{hint}{Hint}[section]
\newtheorem{rmk}{Remark}[section]
\newtheorem{ext}{Extension}[section]

\newtheorem*{df*}{Definition}
\newtheorem*{thm*}{Theorem}
\newtheorem*{clm*}{Claim}
\newtheorem*{cor*}{Corollary}
\newtheorem*{lem*}{Lemma}
\newtheorem*{ax*}{Axiom}
\newtheorem*{prop*}{Proposition}
\newtheorem*{conj*}{Conjecture}
\newtheorem*{nota*}{Notation}

\linespread{1.25}

\newcommand{\inprod}[2]{\left\langle #1, #2 \right\rangle}

\def\Z{{\mathbb{Z}}}
\def\H{{\mathcal{H}}}
\def\M{{\mathcal{M}}}
\def\R{{\mathbb{R}}}
\def\C{{\mathbb{C}}}
\def\Q{{\mathbb{Q}}}
\def\d{{\mathrm{d}}}
\def\i{{\mathrm{i}}}
\def\ep{{\varepsilon}}
\def\N{\mathbb{N}}
\def\1{\mathds{1}}
\def\bigO{\mathcal{O}}
\def\sp{\operatorname{span}}
\def\epsilon{\varepsilon}
\def\emptyset{\varnothing}
\def\phi{\varphi}
\def\dsst{\displaystyle}
\def\st{\ s.t.\ }
\def\wrt{\ w.r.t.\ }
\def\bar{\overline}
\def\tilde{\widetilde}
\def\E{\vb{E}}
\def\B{\vb{B}}
\def\L{\vb{L}}
\def\I{\vb{I}}
\def\Var{\vb{Var}}
\def\V{\vb{Var}}
\def\Cov{\vb{Cov}}
\def\MSE{\vb{MSE}}
\def\P{\vb{P}}
\def\M{\vb{M}}
\def\iid{i.i.d.}
\def\argmax{\arg\max}
\def\argmin{\arg\min}
\def\l{\ell}
\def\hat{\widehat}
\def\independ{\perp\!\!\!\perp}
\def\depend{\leftrightsquigarrow}
\def\residual{\varepsilon}
\def\sd{\mathrm{sd}}
\def\LI{\mathrm{L.I.}}
\def\range{\operatorname{range}}
\def\Null{\operatorname{null}}
\def\nullity{\operatorname{nullity}}
\def\A{A^{-1}}
\def\alg{\operatorname{alg}}
\def\fl{\operatorname{fl}}
\def\algmult{\operatorname{alg. mult.}}
\def\geomult{\operatorname{geo. mult.}}
\def\diag{\operatorname{diag}}
\def\gap{\operatorname{gap}}
\def\pqde{\quad\square}
\def\lub{\operatorname{lub}}
\def\Int{\operatorname{int}}
\def\ac{\operatorname{ac}}
\def\cl{\operatorname{cl}}
\def\bd{\operatorname{bd}}
\DeclareMathOperator*{\plim}{plim}
\def\upint{\mathchoice%
    {\mkern13mu\overline{\vphantom{\intop}\mkern7mu}\mkern-20mu}%
    {\mkern7mu\overline{\vphantom{\intop}\mkern7mu}\mkern-14mu}%
    {\mkern7mu\overline{\vphantom{\intop}\mkern7mu}\mkern-14mu}%
    {\mkern7mu\overline{\vphantom{\intop}\mkern7mu}\mkern-14mu}%
  \int}
\def\lowint{\mkern3mu\underline{\vphantom{\intop}\mkern7mu}\mkern-10mu\int}




\title{\textbf{% ECON 620\\
               Probability and Statistical Inference}}
\author{Tianqi Zhang\\
Emory University}
\date{Apr 17th 2025}

\begin{document}
\maketitle
\setcounter{tocdepth}{1} % Only show sections in the table of contents

\setcounter{section}{4}
\section{Sigma-Algebra}
\subsection{Motivation}
The Banach-Tarski paradox introduces the fundamental problem in measure theory with uncountable latent space $\Omega$. To be more specific, it is our inability to properly define a measure with the aforementioned axioms. \\

\begin{prop}{Non-existence of extension of length to \text{all} subsets of $\R$}\\
	There does not exist a function $\mu$ with all the following properties:
	\begin{enumerate}
		\item [a).] $\mu: \mathscr{P}(\R)\rightarrow [0, \infty]$
		\item [b).]$\mu(I) = \ell(I) \ \forall$ open interval $I$ on $\R$
		\item [c).] Countable additivity: $\mu\qty(\bigcup_k A_k) = \sum_k \mu(A_k)$ for all disjoint $A_k\subset \R$
		\item [d).]Translation invariant: $\mu(t+A) = \mu(A) \ \forall A\subset \R$ and $t\in \R$. 
	\end{enumerate}
Proof: \ref{proof:non-existence}
\end{prop}

\noindent The only condition we can relax is a). Instead of the entire power set, we define the measure to be only on a subset of the power set, defined as "$\sigma$-algebra: 

\subsection{Setup}
\begin{thm}{Sigma Algebra}
A subset \(\mathscr{F} \subseteq \mathscr{P}(X)\) is called a \textbf{sigma algebra} if:
\begin{enumerate}
    \item \(\varnothing, X \in \mathscr{F}\),
    \item If \(A \in \mathscr{F}\), then \(\overline A \in \mathscr{F}\),
    \item If \((A_i)_{i \in \mathbb{N}} \subseteq \mathscr{F}\), then \(\bigcup_{i=1}^\infty A_i \in \mathscr{F}\).
\end{enumerate}
\end{thm}

\begin{df}{Measurable Set}
If \(A \in \mathscr{F}\), then \(A\) is called an \textbf{\(\mathscr{F}\)-measurable set}.
\end{df}

\begin{rmk}
Examples of sigma algebras:
\begin{itemize}
    \item \textbf{Trivial sigma algebra:} \(\mathscr{F} = \{\varnothing, X\}\).
    \item \textbf{Full power set:} \(\mathscr{F} = \mathscr{P}(X)\).
\end{itemize}
\end{rmk}



\subsection{Properties}
\begin{prop}{Intersection of Sigma Algebras}
The countable intersection of sigma algebras is a sigma algebra. If \(\mathscr{F}_i\) is a sigma algebra on \(X\) for \(i \in I\), then:
\[
\bigcap_{i \in I} \mathscr{F}_i \text{ is also a sigma algebra.}
\]
\end{prop}

\begin{df}{Sigma Algebra Generated by a Set}\\
For any \(\mathcal{M} \subseteq \mathscr{P}(X)\), the smallest sigma algebra containing \(\mathcal{M}\) is denoted \(\sigma(\mathcal{M})\) and is called the \textbf{sigma algebra generated by \(\mathcal{M}\)}.
\begin{enumerate}
    \item Collect all large \(\mathscr{F}\) as sigma algebras such that \(\mathcal{M} \subseteq \mathscr{F}\).
    \item Take their intersection:
\end{enumerate}
    \[
    \sigma(\mathcal{M}) = \bigcap_{\mathcal{M} \subseteq \mathscr{F}} \mathscr{F}.
    \]
\end{df}

\begin{eg}{}
Let \(X = \{a, b, c, d\}\) and \(\mathcal{M} = \{\{a\}, \{b\}\}\). Then:
\[
\sigma(\mathcal{M}) = \{\varnothing, X, \{a\}, \{b\}, \{a, b\}, \{c, d\}, \{a, b, c\}, \{a, b, d\}\}.
\]
\end{eg}

\begin{thm}{closure property of $\sigma$-algebras}
	For a sigma-algebra $\mathscr{F}$, if \( A \in \mathscr{F} \), then \(\sigma(A) \subseteq \mathscr{F}\).	
\end{thm}



\begin{prop}{Sigma Algebra on subsets}
\noindent If a $\sigma$-algebra on a larger space is defined, then any subset of that space has a $\sigma$-algebra by intersecting everything with that subset.	
\end{prop}



\subsection{Borel $\sigma$-field on $\R$}
\begin{df}{Borel Sigma Algebra}\\
Let \(X\) be a topological space. \\
The Borel sigma algebra \(\mathscr{B}(X)\) is the sigma algebra \textbf{generated by all open sets} of $X$.
\end{df}

\begin{lem}{Compactness}
	All compact subset has a finite measure w.r.t. Borel sigma algebra. 
\end{lem}

\begin{rmk}
The triplet "(Set, $\sigma$-algebra, measure)": \((X, \mathscr{F}, \mu)\) is called a \textbf{measure space}.
\end{rmk}

\begin{rmk}
The Borel sigma algebra is particularly useful in analysis and probability theory:
\begin{itemize}
    \item For continuous random variables, the pre-image of an open set under a continuous mapping \( f: \mathbb{R} \to \mathbb{R} \) is measurable since it belongs to \( \mathscr{B}(\mathbb{R}) \).
    \item \( \mathscr{B}(\mathbb{R}) \) is the natural sigma algebra for defining measures, such as the Lebesgue measure.
\end{itemize}
\end{rmk}

\begin{df}{Borel $\sigma$-field on $\mathbb{R}^d$}
The Borel $\sigma$-field on $\mathbb{R}^d$, denoted $\mathscr{B}^d$, is the smallest $\sigma$-algebra on $\mathbb{R}^d$ containing all Cartesian products of univariate Borel sets:
\[
\prod_{i=1}^d (a_i, b_i).
\]
\end{df}


\subsection{Proof on Proposition 5.1.1:}
\label{proof:non-existence}
\begin{prf*}\\
\textbf{Construction:}\\
Define the interval \( I = (0, 1] \) with an equivalence relation \( x \sim y \) if \( x - y \in \mathbb{Q} \). That is:
\[
[x] = \{x + r \mid r \in \mathbb{Q}, x \in I\}.
\]
This partitions \( I \) into disjoint sets.

\noindent Pick \( A \subseteq I \) with:
\begin{enumerate}
    \item[i)] \(\forall x, y \in A, \, x \sim y \implies x = y\),
    \item[ii)] For each \( x \in I \), if \( x \in [x] \) for some \( x \), then \( x + r \in A \), where \( A_i = x + A \).
\end{enumerate}

\begin{clm}Disjointness of Shifts: If \( A_n = A + r_n \), where \( r_n \) is an enumeration of \( \mathbb{Q} \cap (-1, 1) \), then:
\[
A_n \cap A_m = \varnothing \quad \text{for } n \neq m.
\]
\end{clm}

\noindent Suppose \( x \in A_n \cap A_m \). Then:
\[
x \in A + r_n \quad \text{and} \quad x \in A + r_m.
\]
This implies:
\[
x = a + r_n \quad \text{and} \quad x = a' + r_m \implies r_n - r_m \in \mathbb{Q}.
\]
By the construction of \( A \), this forces \( r_n = r_m \), which is a contradiction. Thus \( A_n \cap A_m = \varnothing \) for \( n \neq m \).

\begin{clm}[Covering of \((0, 1]\)]
\[
(0, 1] \subseteq \bigcup_{n \in \mathbb{N}} A_n \subseteq (-1, 2).
\]
\end{clm}

\begin{enumerate}
    \item[(i)] The first inclusion is given since all \( A_n \) serve as a partition of \((0, 1]\),
    \item[(ii)] By construction, for all \( x \in \bigcup_{n \in \mathbb{N}} A_n \), there exists \( i \in \mathbb{N} \) such that \( x \in A + r_i \). Since \( A \subseteq (0, 1] \), we conclude:
    \[
    x \in \mathbb{R} \cap (-1, 2).
    \]
\end{enumerate}

By property (ii), \( \mu(x + A) = \mu(A) \) for all \( x \in \mathbb{R} \). By Claim 2:
\[
\mu((0, 1]) \leq \mu\left(\bigcup_{n \in \mathbb{N}} A_n\right) \leq \mu((-1, 2)).
\]
We know \(\mu((0, 1]) = C < \infty\). Then:
\[
\mu((-1, 2)) = \mu((-1, 0]) + \mu((0, 1]) + \mu((1, 2]) = 3C.
\]
Thus:
\[
C \leq \sum_{n=1}^\infty \mu(A_n) \leq 3C.
\]
If \( \mu(A) > 0 \), this leads to a contradiction such that an item bounded above by a finite value $3C$ diverges to infinity. Therefore:
\[
\mu(A) = 0.
\]

\end{prf*}

\end{document}