\usepackage[utf8]{inputenc}
\usepackage[framemethod=TikZ]{mdframed}
\usepackage[hidelinks]{hyperref}
\usepackage{mathtools, amssymb, amsmath, cleveref, fancyhdr, geometry, tcolorbox, graphicx, float, subfigure, arydshln, url, setspace, framed, pifont, physics, ntheorem, cancel}
%%% for coding %%%
\usepackage{listings}
\usepackage[ruled, vlined, linesnumbered]{algorithm2e}
\SetKwComment{Comment}{/* }{ */}
\newcommand\mycommfont[1]{\small\ttfamily\textcolor{mygreen}{#1}}
\SetCommentSty{mycommfont}

\geometry{letterpaper, left=2cm, right=2cm, bottom=2cm, top=2cm}

\pagestyle{fancy}
\fancyhead{}
\fancyhead[L]{\leftmark}
\fancyhead[R]{\rightmark}
\fancyfoot{}
\fancyfoot[C]{\thepage}
%\renewcommand{\headrulewidth}{0pt}
\renewcommand{\footrulewidth}{0pt}

\hypersetup{
	colorlinks = true,
	bookmarks = true,
	bookmarksnumbered = true,
	pdfborder = 001,
	linkcolor = blue
}

\definecolor{emoryblue}{RGB}{1, 33, 105} 
\definecolor{lightblue}{RGB}{0, 125, 186}
\definecolor{mediumblue}{RGB}{ 0, 51, 160}
\definecolor{darkblue}{RGB}{12, 35, 64}
\definecolor{red}{RGB}{185, 58, 38}
\definecolor{green}{RGB}{72, 127, 132}
\definecolor{gray1}{RGB}{217, 217, 214}
\definecolor{gray5}{RGB}{177, 179, 179}
\definecolor{gray3}{RGB}{208, 208, 206}

\definecolor{grey}{rgb}{0.49,0.38,0.29}
\definecolor{mygreen}{rgb}{0,0.6,0}

%%% for coding %%%
\lstset{basicstyle = \ttfamily\small,commentstyle = \color{mygreen}\textit, deletekeywords = {...}, escapeinside = {\%*}{*)}, frame = single, framesep = 0.5em, keywordstyle = \bfseries\color{blue}, morekeywords = {*}, emph = {self}, emphstyle=\bfseries\color{red}, numbers = left, numbersep = 1.5em, numberstyle = \ttfamily\small\color{grey},  rulecolor = \color{black}, showstringspaces = false, stringstyle = \ttfamily\color{purple}, tabsize = 4, columns = flexible}

\newcounter{index}[section]
\renewcommand{\theindex}{\thesection.\arabic{index}}

\newenvironment*{df}[1]{%
  \stepcounter{index}%
  \noindent\textbf{Definition \theindex\ (#1).}%
}{\\}

\newenvironment*{eg}[1]{%
  \stepcounter{index}%
  \begin{framed}\noindent\textbf{Example \theindex\ #1}\\ 
}{%
  \end{framed}
}

\newenvironment*{thm}[1]{%
  \stepcounter{index}%
  \begin{tcolorbox}[colback=gray!5, colframe=gray!40!black,
    title=Theorem \theindex\ (#1)]%
}{%
  \end{tcolorbox}
}

\newenvironment*{cor}[1]{\stepcounter{index}\noindent\textbf{Corollary \theindex\ (#1):}}{\\}
\newenvironment*{lem}[1]{\stepcounter{index}\noindent\textbf{Lemma \theindex\ (#1):}}{\\}
\newenvironment*{ax}[1]{\stepcounter{index}\noindent\textbf{Axiom \theindex\ (#1):}}{\\}
\newenvironment*{prop}[1]{\stepcounter{index}\noindent\textbf{Proposition \theindex\ (#1):}}{\\}
\newenvironment*{conj}[1]{\stepcounter{index}\noindent\textbf{Conjecture \theindex\ (#1):}}{\\}
\newenvironment*{nota}{\stepcounter{index}\noindent\textbf{Notation \theindex.}}{\\}
\newenvironment*{clm}{\stepcounter{index}\noindent\textbf{Claim \theindex}}{\\}

\newcounter{nprf}[section]
\setcounter{nprf}{0}
\newenvironment*{prf}{\noindent\textbf{\textit{Proof \stepcounter{nprf}\thenprf.}}}{\hfill$\blacksquare$\\}
\newenvironment*{dis}{\indent\textbf{\textit{Disproof \stepcounter{nprf}\thenprf.}}}{\hfill$\blacksquare$\\}
\newenvironment*{sol}{\indent\textbf{\textit{Solution \stepcounter{nprf}\thenprf.}}\\}{\hfill{$\square$}\\}

\newenvironment*{prf*}{\noindent\textit{Proof.}\ }{$\qquad\square$\\}
\newenvironment*{dis*}{\indent\textit{Disproof.}\ }{$\qquad\square$\\}
\newenvironment*{sol*}{\indent\textit{Solution.}\ }{$\qquad\square$\\}

\newtheorem{hint}{Hint}[section]
\newtheorem{rmk}{Remark}[section]
\newtheorem{ext}{Extension}[section]

\newtheorem*{df*}{Definition}
\newtheorem*{thm*}{Theorem}
\newtheorem*{clm*}{Claim}
\newtheorem*{cor*}{Corollary}
\newtheorem*{lem*}{Lemma}
\newtheorem*{ax*}{Axiom}
\newtheorem*{prop*}{Proposition}
\newtheorem*{conj*}{Conjecture}
\newtheorem*{nota*}{Notation}

\linespread{1.25}

\newcommand{\inprod}[2]{\left\langle #1, #2 \right\rangle}

\def\Z{{\mathbb{Z}}}
\def\R{{\mathbb{R}}}
\def\C{{\mathbb{C}}}
\def\Q{{\mathbb{Q}}}
\def\d{{\mathrm{d}}}
\def\i{{\mathrm{i}}}
\def\ep{{\varepsilon}}
\def\N{\mathbb{N}}
\def\1{\mathds{1}}
\def\bigO{\mathcal{O}}
\def\sp{\operatorname{span}}
\def\epsilon{\varepsilon}
\def\emptyset{\varnothing}
\def\phi{\varphi}
\def\dsst{\displaystyle}
\def\st{\ s.t.\ }
\def\wrt{\ w.r.t.\ }
\def\bar{\overline}
\def\tilde{\widetilde}
\def\E{\mathbb{E}}
\def\B{\vb{B}}
\def\L{\vb{L}}
\def\I{\vb{I}}
\def\Var{\vb{Var}}
\def\V{\vb{Var}}
\def\Cov{\vb{Cov}}
\def\MSE{\vb{MSE}}
\def\P{\vb{P}}
\def\M{\vb{M}}
\def\iid{i.i.d.}
\def\argmax{\arg\max}
\def\argmin{\arg\min}
\def\l{\ell}
\def\hat{\widehat}
\def\independ{\perp\!\!\!\perp}
\def\depend{\leftrightsquigarrow}
\def\residual{\varepsilon}
\def\sd{\mathrm{sd}}
\def\LI{\mathrm{L.I.}}
\def\range{\operatorname{range}}
\def\Null{\operatorname{null}}
\def\nullity{\operatorname{nullity}}
\def\A{A^{-1}}
\def\alg{\operatorname{alg}}
\def\fl{\operatorname{fl}}
\def\algmult{\operatorname{alg. mult.}}
\def\geomult{\operatorname{geo. mult.}}
\def\diag{\operatorname{diag}}
\def\gap{\operatorname{gap}}
\def\pqde{\quad\square}
\def\lub{\operatorname{lub}}
\def\Int{\operatorname{int}}
\def\ac{\operatorname{ac}}
\def\cl{\operatorname{cl}}
\def\bd{\operatorname{bd}}
\def\upint{\mathchoice%
    {\mkern13mu\overline{\vphantom{\intop}\mkern7mu}\mkern-20mu}%
    {\mkern7mu\overline{\vphantom{\intop}\mkern7mu}\mkern-14mu}%
    {\mkern7mu\overline{\vphantom{\intop}\mkern7mu}\mkern-14mu}%
    {\mkern7mu\overline{\vphantom{\intop}\mkern7mu}\mkern-14mu}%
  \int}
\def\lowint{\mkern3mu\underline{\vphantom{\intop}\mkern7mu}\mkern-10mu\int}


\title{\textbf{Measure Theory\\
               Study Note}}
\author{Tianqi Zhang}
\date{\today}

\begin{document}
\maketitle

\tableofcontents

\newpage
\section{Preface}
Welcome to my personal study notes on measure theory, with a particular emphasis on its applications in probability theory. To fully benefit from these notes, it is recommended that readers have a solid foundation in basic real analysis, including the following topics:
\begin{itemize}
    \item Sequence and function convergence
    \item Point-set topology
    \item Integration analysis
\end{itemize}
This document aims to provide a structured and concise exploration of measure theory, blending rigorous mathematical concepts with practical insights to support further studies in probability theory and related fields.


\begin{flushright}
-- Tianqi Zhang\\
Dec 14th, 2024
\end{flushright}


\newpage
\section{Measure}

\subsection{Motivation and Definition of Outer Measure}
Recall the definition of upper and lower Darboux sum in Riemann integral setting: 
$$U(f) = \sum_{i=1}^N \sup_{x\in [x_i, x_{i+1}]}{f(x)} \cdot \underbrace{(x_{i+1} - x_i)}_{\text{length}}$$
Definition of Riemann non-integrable involves the upper integral and the lower integral as two limits do not agree, which likely boils down to some partitions $[x_i, x_{i+1}]$ not well-defined. Therefore, to propose a fix, we are motivated to properly define the "length" of any general subset of $\R$. \\

\begin{df}{Length}
    The length $\ell(I)$ of some open interval $I\subset \R$ is a function defined by
    $$\ell(I) =
    \begin{cases}
         b-a, & I = (b-a), a<b, a, b\in \R\\
         0,  & I = \emptyset, \\
         \infty & I = (-\infty, a), a\in \R\\
         \infty & I = (a, \infty), a\in \R
    \end{cases}$$
\end{df}
Then suppose $A\subset \R$. The size of $A$ can at most be the \textbf{sum of lengths of a sequence of open intervals $I$ whose union contains} $A$. Taking the infimum of such sums over all possible sequences of $I$, we obtain the outer measure of $A$, i.e.\\

\begin{df}{Outer Measure, $\abs{A}$}
    For $A\subset \R$,
    $$\abs{A} \equiv \inf\qty{\sum_{k=1}^\infty \ell(I_k)\mid I_k \text{ open, } A\subset \bigcup_{k=1}^\infty I_k }$$
\end{df}

\begin{prop}{Finite sets have outer measure 0} 
Proof: \ref{prof:finite set}
\end{prop}

\subsubsection{Properties of Outer Measure}
\begin{prop}{Countable subsets of $\R$ have outer measure 0} 
Proof: \ref{prof:countable set}
\end{prop}

\begin{prop}{Order preserving of outer measure}\\
If $A\subset B\subset \R$, then $\abs{A}\leq\abs{B}$
Proof: \ref{prof:order preserving}
\end{prop}

\begin{df}{Translation}
    For any $A\subset \R, t\in \R$, the translation $t+A$ is defined by
    \[t+A = \qty{t+a \mid a\in A}\]
\end{df}
Note that the length function should be translation invariant. Therefore, we obtain the proposition that outer measure is translation invariant. \\

\begin{prop}{Outer measure is translation invariant}\\
Suppose $t\in \R$ and $A\subset \R$, then $\abs{t+A} = \abs{A}$.
\end{prop}

\begin{prop}{Countable Sub-additivity of outer measure}
Suppose $A_1, A_2, \dots, \subset \R$. Then 
$$\abs{\bigcup_{k=1}^\infty A_k} \leq \sum_{k=1}^\infty \abs{A_k}$$	
Note that this implies finite sub-additivity which could come handy in proof techniques: 
$$\abs{A_1\cup \dots \cup A_n} \leq \abs{A_1} + \dots + \abs{A_n}$$
\end{prop}

\subsection{Outer Measure of Closed Bounded Interval}
It is apparent for any closed interval $[a, b]$, we can construct a sequence of open cover $(a-\ep, b+\ep)$ and arbitrarily shrink $\ep$. We obtain $\abs{[a, b]} \leq b-a$. However, the other direction requires completeness of $\R$. \\

\begin{prop}{$\abs{[a, b]} = b-a$}\\
Suppose $a, b\in \R, a<b$. Then $\abs{[a, b]} = b-a$. 
Proof: \ref{prof:closed interval}
\end{prop}

\begin{prop}{Non-trivial intervals are uncountable}\\
	Every \textbf{interval} in $\R$ that contains at least two distinct terms is uncountable. 
\end{prop}

\begin{prop}{Non-additivity}\\
	$\exists A, B\subset \R$ disjoint such that $\abs{A\cup B} \neq \abs{A} + \abs{B}$. 
\end{prop}

\subsection{Measurable Space and Functions}
Countable additivity is desired for many scenario since we are mostly interested in proving theorems on limits. However, there is a fundamental problem of measure theory such as follows: \\

\begin{prop}{Non-existence of extension of length to \text{all} subsets of $\R$}\\
	There does not exist a function $\mu$ with all the following properties:\\
	a). $\mu: \mathcal{P}(\R)\rightarrow [0, \infty]$\\
	b). $\mu(I) = \ell(I) \forall$ open interval $I$ on $\R$\\
	c). Countable additivity: $\mu\qty(\bigcup_k A_k) = \sum_k \mu(A_k)$ for all disjoint $A_k\subset \R$\\
	d). Translation invariant: $\mu(t+A) = \mu(A) \ \forall A\subset \R$ and $t\in \R$. 
\end{prop}

\noindent The only relaxation we can make is $a).$ Instead of defining the function on the entirety of power set of $\R$, we consider a more practice subset of the power set such that the subset is closed under complementation and countable unions. We make the following definition: \\

\begin{df}{Sigma algebra} \\
	Suppose $X$ is a set and $S\subset \mathcal{P}(X)$. Then $S$ is a $\sigma$-algebra if the following conditions are satisfied:\\
	1. $\emptyset \in S$\\
	2. if $E\in S$, then $\overline E\in S$\\
	3. if $\qty{E_k}_k\subset S$, then $\bigcup_k E_k \in S$
\end{df}



%%%%%%%%%%%%%%%%%% PROOF %%%%%%%%%%%%%%%%%%%%%%%%
\newpage
\appendix

\section{Proofs}
\setcounter{subsection}{1}
\subsection{Measure}

\subsubsection{Finite sets have outer measure 0}\label{prof:finite set}

\begin{prf*}
Suppose $A = \qty{a_1, a_2, \dots, a_N}\subset \R$. It is equivalent to prove that $\forall \ep$, $\exists \qty{I_k}$ such that $A\subset \bigcup I_k$ and $\sum \ell(I_k) <\ep$. Therefore, let 
$$I_k = \begin{cases}
    \qty(a_k - \frac{\ep}{4N}, a_k + \frac{\ep}{4N}), & k\leq N\\
    \emptyset, & k>n
\end{cases}$$
we have $A\subset \bigcup I_k$, and we have the sum $\sum_{k=1}^\infty I_k = \sum_{k=1}^n \frac{\ep}{2n} = \frac{\ep}{2} < \ep$. Therefore, $\abs{A} = 0$.
\end{prf*}

\subsubsection{Countable sets have outer measure 0}\label{prof:countable set}

\begin{prf*}
Suppose $A = \qty{a_1, a_2, \dots}\subset \R$. 
With similar idea, for any $\ep$, construct $I_k$ to be
$$I_k = \qty(a_k -\frac{\ep}{2^{k+2}}, a_k + \frac{\ep}{2^{k+2}})$$
Then each $a_k\in I_k$, which implies $A\subset \bigcup I_k$. Now consider the infinite sum:
\begin{align*}
    \sum_{k=1}^\infty I_k &= \sum_{k=1}^\infty\frac{\ep}{2^{k+1}}\\
                          &= \frac{\ep}{2}\\
                          &< \ep
\end{align*}
Therefore, $\abs{A} = 0$.
\end{prf*}


\subsubsection{Outer measure is order preserving}\label{prof:order preserving}
\begin{prf*}
Since $A\subset B$, then any open cover for $A$ would be an open cover for $B$. Taking infimum over all sequences of open covers of $B$, we have $\abs{A}\leq \abs{B}$. 
\end{prf*}

\subsubsection{Outer Measure is translation invariant}
\label{prof:translation invariant}
\begin{prf*}
Suppose $t\in \R, A\subset \R$ and suppose $\qty{I_k}$ covers $A$. Then $\qty{t+I_k}$ should be a set of cover for $t+A$, we obtain the following inequality: 
$$\abs{t+A} \leq \sum_{k} \ell(t+I_k)$$
Since $\ell$ is translational invariant, we have $\sum_{k} \ell(t+I_k) = \sum_{k} \ell(I_k)$. Taking the infimum on both sides: 
$$\abs{t+A} \leq \abs{A}$$

Now consider $\qty{t+I_k}$ to be a set of open cover for $t+A$ and $A = -t+(t+A)$, we have 

\begin{align*}
	\abs{A} &\leq \sum_{k} \ell(-t + (t+I_k)) = \sum_{k} \ell (t+I_k)\\
	\inf_{t+I_k} \abs{A} &\leq	\inf_{t+I_k}  \sum_{k} \ell (t+I_k)\\
	\abs{A} &\leq \abs{t+A}
\end{align*}
We have $\abs{A} = \abs{t+A}$. 
\end{prf*}


\subsubsection{Sub-additivity}
\label{prof:subadditivity}
\begin{prf*}
Suppose for each $A_k$, we have a sequence of open cover $\qty{I_{ik}}_i$ such that it is close enough above to the outer measure. 
$$\sum_{i} \ell(I_{ik}) \leq \frac{\ep}{2^k} + \abs{A_k}$$
Then summing both sides over $k$, we have
\begin{align*}
	\sum_{k}\sum_{i} \ell(I_{ik}) &\leq \sum_{k} \frac{\ep}{2^k} + \abs{A_k}\\
	\sum_{k}\sum_{i} \ell(I_{ik}) &\leq \ep + \sum_{k}\abs{A_k}
\end{align*}
Observe that the left hand side $\cup_{i, k} I_{ik}$ is an open cover for the entire union $\cup_k A_k$, then we have
$$ \abs{\bigcup_k A_k} \leq \sum_{k}\sum_{i} \ell(I_{ik}) \leq \ep + \sum_{k}\abs{A_k}$$
$$\abs{\bigcup_k A_k} < \abs{A_k}$$
\end{prf*}

\subsubsection{Closed Interval Length}
\label{prof:closed interval}

\begin{prf*}
One direction is trivially proven. Now for $\abs{[a, b]} \leq b - a$, recall Heine-Borel theorem. $\exists \qty{I_k}_{k=1}^K$ such that $[a, b]\subset \qty{I_k}_{k=1}^K$, which implies that 
$$\abs{[a,b]} \leq \sum_{k=1}^K \ell(I_k)$$
We now need to show that the sum is bounded below by $b-a$, by induction. Note that the statement is true when $K = 1$. Now assume that above holds for $K$, now consider a new set of open cover $\qty{I_k}_{k=1}^{K+1}$ for $[a, b]$. Without loss of generality, we assume that $b\in I_{K+1} \equiv (c, d)$ with $c, d\in \R$. If $c<a$, $I_{K+1}$ alone covers $[a, b]$ and the proof is finished. So consider $a<c<b<d$.

One immediate conclusion is that the sub-interval $[a, c]$ is covered by $\qty{I_k}_{k=1}^{K}$, and by our induction hypothesis, 
\begin{align*}
	\sum_{k=1}^K I_k &\geq c-a\\
	\sum_{k=1}^{K+1} I_k &\geq c-a +\ell(I_{K+1})\\
	&\geq c-a+d-c = d-a\\
\abs{[a, b]} \geq \sum_{k=1}^{K+1} I_k &\geq b-a
\end{align*}
Therefore, we have both directions such that $\abs{[a, b]} = b-a$. 
\end{prf*}

%%%%%%%%%%%%%%%%%% PRACTICES %%%%%%%%%%%%%%%%%%%%%%%%

\section{Practices}
\setcounter{subsection}{1}
\subsection{Measure}


%\newpage
%\section{This is the section name}
%\subsection{This is the subsection name}
%
%\begin{thm}{Name of thm}
%	This is a theorem. 
%\end{thm}
%\begin{thm*}
%	
%\end{thm*}
%
%\begin{df}{Name for df}
%	This is a definition. 
%\end{df}
%\begin{df*}
%	
%\end{df*}
%
%\begin{eg}{}
%	This is an example
%	\begin{sol}
%		
%	\end{sol}
%	\begin{sol*}
%		
%	\end{sol*}
%\end{eg}
%
%\begin{rmk}
%	This is a remark
%\end{rmk}
%
%\begin{prf}
%	This is a proof
%\end{prf}
%\begin{prf*}
%	
%\end{prf*}
%\begin{dis}
%	
%\end{dis}
%\begin{dis*}
%	
%\end{dis*}
%
%\begin{cor}{}
%	This is a corollary
%\end{cor}
%
%\begin{lem}{}
%	This is a lemma
%\end{lem}
%
%\begin{prop}{}
%	This is a proposition
%\end{prop}
%
%\begin{conj}{}
%	This is a conjecture
%\end{conj}
%
%\begin{ax}{}
%	This is an axiom.
%\end{ax}
%
%\begin{clm}
%	
%\end{clm}
%\begin{clm*}
%	
%\end{clm*}

%\begin{lstlisting}[language = Matlab, title = {Answer.m}]
%% Plot function f(x) = 2*x^3 - x - 2
%ezplot('2*x^3-x-2', [0, 2])
%hold on
%plot([0, 2], [0, 0], 'r')
%\end{lstlisting}
%
%\begin{lstlisting}[language = Python]
%def __self__(self):
%for i in range(10):
%	print(f"This number is {i}.")
%\end{lstlisting}
%
%\begin{lstlisting}[language = java]
%public static void main(String[] args) {
%	System.out.println("Hello World!"); // comment
%}
%\end{lstlisting}

%\begin{algorithm}
%\caption{Bisection Algorithm}
%\SetKwData{In}{\rmfamily\textbf{in}}\SetKwData{To}{\rmfamily\textbf{to}}\SetKwData{And}{\rmfamily{\textbf{and}}}\SetKwData{Or}{\rmfamily{\textbf{or}}}\SetKwData{Stop}{\rmfamily{\textbf{stop}}}\SetKwData{Break}{\rmfamily{\textbf{break}}}
%\SetKwComment{Comment}{/* }{ */}
%\DontPrintSemicolon
%\KwIn{$a,b,M,\delta,\epsilon$\;$u\leftarrow f(a)$\;$b\leftarrow f(b)$\;$e\leftarrow b-a$}
%\KwOut{output}
%\BlankLine
%\Begin{
%	\If{$\text{sign}(u)=\text{sign}(v)$}{\Stop}
%	\For{k=1 \To M}{
%		$e\leftarrow e/2$\;
%		$c\leftarrow a+e$\;
%		$w\leftarrow f(c)$\;
%		\Return $k,c,w,e$\;
%		\If{$|e|<\delta$\Or$|w|<\epsilon$}{\Stop}
%		\eIf{$\text{sign}(u)\neq=\text{sign}(v)$}{
%			$b\leftarrow c$\;
%			$v\leftarrow w$\;
%		} {
%			$a\leftarrow c$\;
%			$u\leftarrow w$\;
%		}
%	}
%}
%\end{algorithm}
\end{document}